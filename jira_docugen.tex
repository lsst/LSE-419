% generated from JIRA project LVV
% using template at /var/jenkins_home/.local/lib/python3.7/site-packages/docsteady/templates/dm-spec.latex.jinja2.
% Collecting ATM data from folder: "/Project Systems Engineering/Commissioning Science Verification"
% using docsteady version 1.2rc1
% Please do not edit -- update information in Jira instead

\section{Test Cases Summary}\label{test-cases-summary}

\begin{longtable}[]{p{3cm}p{13cm}}
\toprule
Test Id & Test Name\tabularnewline
\midrule
\endhead
    \hyperref[lvv-t293]{LVV-T293} &
    \href{https://jira.lsstcorp.org/secure/Tests.jspa\#/testCase/LVV-T293}{On-sky Observations: Single-visit Key Performance Metrics} \tabularnewline
    \hyperref[lvv-t294]{LVV-T294} &
    \href{https://jira.lsstcorp.org/secure/Tests.jspa\#/testCase/LVV-T294}{On-sky Observations: Full-survey Key Performance Metrics} \tabularnewline
    \hyperref[lvv-t295]{LVV-T295} &
    \href{https://jira.lsstcorp.org/secure/Tests.jspa\#/testCase/LVV-T295}{Data Processing Campaign: Single-visit Key Performance Metrics} \tabularnewline
    \hyperref[lvv-t296]{LVV-T296} &
    \href{https://jira.lsstcorp.org/secure/Tests.jspa\#/testCase/LVV-T296}{Data Processing Campaign: Full-survey Key Performance Metrics} \tabularnewline
    \hyperref[lvv-t297]{LVV-T297} &
    \href{https://jira.lsstcorp.org/secure/Tests.jspa\#/testCase/LVV-T297}{Absolute Astrometric Performance} \tabularnewline
    \hyperref[lvv-t298]{LVV-T298} &
    \href{https://jira.lsstcorp.org/secure/Tests.jspa\#/testCase/LVV-T298}{Cross-band Astrometric Performance} \tabularnewline
    \hyperref[lvv-t299]{LVV-T299} &
    \href{https://jira.lsstcorp.org/secure/Tests.jspa\#/testCase/LVV-T299}{Relative Astrometric Performance} \tabularnewline
    \hyperref[lvv-t360]{LVV-T360} &
    \href{https://jira.lsstcorp.org/secure/Tests.jspa\#/testCase/LVV-T360}{Off Zenith Image Quality Degradation} \tabularnewline
    \hyperref[lvv-t361]{LVV-T361} &
    \href{https://jira.lsstcorp.org/secure/Tests.jspa\#/testCase/LVV-T361}{Verify 10-year Ellipticity Residuals w/ On-Sky Data} \tabularnewline
    \hyperref[lvv-t389]{LVV-T389} &
    \href{https://jira.lsstcorp.org/secure/Tests.jspa\#/testCase/LVV-T389}{Single Visit Photometric Repeatability} \tabularnewline
    \hyperref[lvv-t434]{LVV-T434} &
    \href{https://jira.lsstcorp.org/secure/Tests.jspa\#/testCase/LVV-T434}{Acquire Filter Response Verification Data (with flat field screen)} \tabularnewline
    \hyperref[lvv-t442]{LVV-T442} &
    \href{https://jira.lsstcorp.org/secure/Tests.jspa\#/testCase/LVV-T442}{Control Ghosts in Coadds} \tabularnewline
    \hyperref[lvv-t445]{LVV-T445} &
    \href{https://jira.lsstcorp.org/secure/Tests.jspa\#/testCase/LVV-T445}{Acquire Filter Response Verification Data (with CBP)} \tabularnewline
    \hyperref[lvv-t446]{LVV-T446} &
    \href{https://jira.lsstcorp.org/secure/Tests.jspa\#/testCase/LVV-T446}{Process Filter Response Verification Data (with flat field screen)} \tabularnewline
    \hyperref[lvv-t447]{LVV-T447} &
    \href{https://jira.lsstcorp.org/secure/Tests.jspa\#/testCase/LVV-T447}{Process Filter Response Verification Data (with CBP)} \tabularnewline
    \hyperref[lvv-t448]{LVV-T448} &
    \href{https://jira.lsstcorp.org/secure/Tests.jspa\#/testCase/LVV-T448}{Analyze Filter Response Uniformity} \tabularnewline
    \hyperref[lvv-t452]{LVV-T452} &
    \href{https://jira.lsstcorp.org/secure/Tests.jspa\#/testCase/LVV-T452}{Analyze in-band ripple} \tabularnewline
    \hyperref[lvv-t453]{LVV-T453} &
    \href{https://jira.lsstcorp.org/secure/Tests.jspa\#/testCase/LVV-T453}{Analyze Filter Response Envelope} \tabularnewline
    \hyperref[lvv-t461]{LVV-T461} &
    \href{https://jira.lsstcorp.org/secure/Tests.jspa\#/testCase/LVV-T461}{Filter Out of Band Constraints} \tabularnewline
    \hyperref[lvv-t532]{LVV-T532} &
    \href{https://jira.lsstcorp.org/secure/Tests.jspa\#/testCase/LVV-T532}{MOPS completeness threshold} \tabularnewline
    \hyperref[lvv-t533]{LVV-T533} &
    \href{https://jira.lsstcorp.org/secure/Tests.jspa\#/testCase/LVV-T533}{MOPS purity threshold} \tabularnewline
    \hyperref[lvv-t543]{LVV-T543} &
    \href{https://jira.lsstcorp.org/secure/Tests.jspa\#/testCase/LVV-T543}{Astrometric error -- level 1 processing -- simulations} \tabularnewline
    \hyperref[lvv-t544]{LVV-T544} &
    \href{https://jira.lsstcorp.org/secure/Tests.jspa\#/testCase/LVV-T544}{Astrometric error -- level 1 processing -- on-sky data} \tabularnewline
    \hyperref[lvv-t545]{LVV-T545} &
    \href{https://jira.lsstcorp.org/secure/Tests.jspa\#/testCase/LVV-T545}{Astrometric error -- level 1 processing -- reference catalog} \tabularnewline
    \hyperref[lvv-t546]{LVV-T546} &
    \href{https://jira.lsstcorp.org/secure/Tests.jspa\#/testCase/LVV-T546}{Photometric error -- level 1 processing -- simulations} \tabularnewline
    \hyperref[lvv-t547]{LVV-T547} &
    \href{https://jira.lsstcorp.org/secure/Tests.jspa\#/testCase/LVV-T547}{Photometric errors -- level 1 processing -- on-sky data} \tabularnewline
    \hyperref[lvv-t548]{LVV-T548} &
    \href{https://jira.lsstcorp.org/secure/Tests.jspa\#/testCase/LVV-T548}{Photometric errors -- level 1 processing -- reference catalog} \tabularnewline
    \hyperref[lvv-t549]{LVV-T549} &
    \href{https://jira.lsstcorp.org/secure/Tests.jspa\#/testCase/LVV-T549}{Zeropoint consistency} \tabularnewline
    \hyperref[lvv-t550]{LVV-T550} &
    \href{https://jira.lsstcorp.org/secure/Tests.jspa\#/testCase/LVV-T550}{MOPS -- orbit association completeness} \tabularnewline
    \hyperref[lvv-t551]{LVV-T551} &
    \href{https://jira.lsstcorp.org/secure/Tests.jspa\#/testCase/LVV-T551}{WCS accuracy -- simulations} \tabularnewline
    \hyperref[lvv-t554]{LVV-T554} &
    \href{https://jira.lsstcorp.org/secure/Tests.jspa\#/testCase/LVV-T554}{Single exposure dynamic range} \tabularnewline
    \hyperref[lvv-t587]{LVV-T587} &
    \href{https://jira.lsstcorp.org/secure/Tests.jspa\#/testCase/LVV-T587}{PSF size in pixels} \tabularnewline
    \hyperref[lvv-t588]{LVV-T588} &
    \href{https://jira.lsstcorp.org/secure/Tests.jspa\#/testCase/LVV-T588}{Median image quality at 0.44 arcsecond seeing} \tabularnewline
    \hyperref[lvv-t589]{LVV-T589} &
    \href{https://jira.lsstcorp.org/secure/Tests.jspa\#/testCase/LVV-T589}{Median image quality at 0.6 arcsecond seeing} \tabularnewline
    \hyperref[lvv-t590]{LVV-T590} &
    \href{https://jira.lsstcorp.org/secure/Tests.jspa\#/testCase/LVV-T590}{Median image quality at 0.8 arcsecond seeing} \tabularnewline
    \hyperref[lvv-t591]{LVV-T591} &
    \href{https://jira.lsstcorp.org/secure/Tests.jspa\#/testCase/LVV-T591}{Flux-enclosing radius} \tabularnewline
    \hyperref[lvv-t592]{LVV-T592} &
    \href{https://jira.lsstcorp.org/secure/Tests.jspa\#/testCase/LVV-T592}{Image quality - maximum system contribution} \tabularnewline
    \hyperref[lvv-t593]{LVV-T593} &
    \href{https://jira.lsstcorp.org/secure/Tests.jspa\#/testCase/LVV-T593}{Image quality at zenith} \tabularnewline
    \hyperref[lvv-t594]{LVV-T594} &
    \href{https://jira.lsstcorp.org/secure/Tests.jspa\#/testCase/LVV-T594}{Image quality - degradation from zenith} \tabularnewline
    \hyperref[lvv-t595]{LVV-T595} &
    \href{https://jira.lsstcorp.org/secure/Tests.jspa\#/testCase/LVV-T595}{PSF ellipticity} \tabularnewline
    \hyperref[lvv-t596]{LVV-T596} &
    \href{https://jira.lsstcorp.org/secure/Tests.jspa\#/testCase/LVV-T596}{Depth: r-band} \tabularnewline
    \hyperref[lvv-t597]{LVV-T597} &
    \href{https://jira.lsstcorp.org/secure/Tests.jspa\#/testCase/LVV-T597}{Depth variation over field of view} \tabularnewline
    \hyperref[lvv-t938]{LVV-T938} &
    \href{https://jira.lsstcorp.org/secure/Tests.jspa\#/testCase/LVV-T938}{Level 2 reproducibility (same computer hardware)} \tabularnewline
    \hyperref[lvv-t939]{LVV-T939} &
    \href{https://jira.lsstcorp.org/secure/Tests.jspa\#/testCase/LVV-T939}{Level 1 reproducibility (same computer hardware)} \tabularnewline
    \hyperref[lvv-t940]{LVV-T940} &
    \href{https://jira.lsstcorp.org/secure/Tests.jspa\#/testCase/LVV-T940}{Level 2 reproducibility (different computer hardware)} \tabularnewline
    \hyperref[lvv-t941]{LVV-T941} &
    \href{https://jira.lsstcorp.org/secure/Tests.jspa\#/testCase/LVV-T941}{Level 1 reproducibility (different computer hardware)} \tabularnewline
    \hyperref[lvv-t942]{LVV-T942} &
    \href{https://jira.lsstcorp.org/secure/Tests.jspa\#/testCase/LVV-T942}{Provenance on Level 2 catalogs} \tabularnewline
    \hyperref[lvv-t943]{LVV-T943} &
    \href{https://jira.lsstcorp.org/secure/Tests.jspa\#/testCase/LVV-T943}{Provenance in Level 1 catalogs} \tabularnewline
    \hyperref[lvv-t950]{LVV-T950} &
    \href{https://jira.lsstcorp.org/secure/Tests.jspa\#/testCase/LVV-T950}{DIASource misassociation rate} \tabularnewline
\bottomrule
\end{longtable}

\newpage

\section{Test Cases}

\subsection{\href{https://jira.lsstcorp.org/secure/Tests.jspa\#/testCase/LVV-T293}{LVV-T293}
    - On-sky Observations: Single-visit Key Performance Metrics}\label{lvv-t293}

\begin{longtable}[]{llllll}
\toprule
Version & Status & Priority & Verification Type & Owner
\\\midrule
1 & Draft & Normal &
Demonstration & Keith Bechtol
\\\bottomrule
\end{longtable}

\subsubsection{Verification Elements}
\begin{itemize}
\item \href{https://jira.lsstcorp.org/browse/LVV-1544}{LVV-1544} - OSS-REQ-0228-V-02: Image Size in Pixels

\end{itemize}

\subsubsection{Test Items}


\subsubsection{Predecessors}

\subsubsection{Environment Needs}

\paragraph{Software}

\paragraph{Hardware}

\subsubsection{Input Specification}

\subsubsection{Output Specification}

\subsubsection{Test Procedure}
    \begin{longtable}[]{p{1.3cm}p{2cm}p{13cm}}
    %\toprule
    Step & \multicolumn{2}{@{}l}{Description, Input Data and Expected Result} \\ \toprule
    \endhead

            \multirow{3}{*}{ 1 } & Description &
            \begin{minipage}[t]{13cm}{\footnotesize
            
            \vspace{\dp0}
            } \end{minipage} \\ \cline{2-3}
            & Test Data &
            \begin{minipage}[t]{13cm}{\footnotesize
                No data.
                \vspace{\dp0}
            } \end{minipage} \\ \cline{2-3}
            & Expected Result &
        \\ \midrule
    \end{longtable}

\subsection{\href{https://jira.lsstcorp.org/secure/Tests.jspa\#/testCase/LVV-T294}{LVV-T294}
    - On-sky Observations: Full-survey Key Performance Metrics}\label{lvv-t294}

\begin{longtable}[]{llllll}
\toprule
Version & Status & Priority & Verification Type & Owner
\\\midrule
1 & Draft & Normal &
Demonstration & Keith Bechtol
\\\bottomrule
\end{longtable}

\subsubsection{Verification Elements}
\begin{itemize}
\item \href{https://jira.lsstcorp.org/browse/LVV-1545}{LVV-1545} - OSS-REQ-0228-V-03: Image Quality vs Field

\item \href{https://jira.lsstcorp.org/browse/LVV-7213}{LVV-7213} - OSS-REQ-0228-V-04: Image Quality in Encircled Energy

\item \href{https://jira.lsstcorp.org/browse/LVV-7214}{LVV-7214} - OSS-REQ-0228-V-05: System Contribution to Image Quality

\end{itemize}

\subsubsection{Test Items}


\subsubsection{Predecessors}

\subsubsection{Environment Needs}

\paragraph{Software}

\paragraph{Hardware}

\subsubsection{Input Specification}

\subsubsection{Output Specification}

\subsubsection{Test Procedure}
    \begin{longtable}[]{p{1.3cm}p{2cm}p{13cm}}
    %\toprule
    Step & \multicolumn{2}{@{}l}{Description, Input Data and Expected Result} \\ \toprule
    \endhead

            \multirow{3}{*}{ 1 } & Description &
            \begin{minipage}[t]{13cm}{\footnotesize
            
            \vspace{\dp0}
            } \end{minipage} \\ \cline{2-3}
            & Test Data &
            \begin{minipage}[t]{13cm}{\footnotesize
                No data.
                \vspace{\dp0}
            } \end{minipage} \\ \cline{2-3}
            & Expected Result &
        \\ \midrule
    \end{longtable}

\subsection{\href{https://jira.lsstcorp.org/secure/Tests.jspa\#/testCase/LVV-T295}{LVV-T295}
    - Data Processing Campaign: Single-visit Key Performance Metrics}\label{lvv-t295}

\begin{longtable}[]{llllll}
\toprule
Version & Status & Priority & Verification Type & Owner
\\\midrule
1 & Draft & Normal &
Demonstration & Keith Bechtol
\\\bottomrule
\end{longtable}

\subsubsection{Verification Elements}
\begin{itemize}
\item \href{https://jira.lsstcorp.org/browse/LVV-1544}{LVV-1544} - OSS-REQ-0228-V-02: Image Size in Pixels

\end{itemize}

\subsubsection{Test Items}


\subsubsection{Predecessors}

\subsubsection{Environment Needs}

\paragraph{Software}

\paragraph{Hardware}

\subsubsection{Input Specification}

\subsubsection{Output Specification}

\subsubsection{Test Procedure}
    \begin{longtable}[]{p{1.3cm}p{2cm}p{13cm}}
    %\toprule
    Step & \multicolumn{2}{@{}l}{Description, Input Data and Expected Result} \\ \toprule
    \endhead

            \multirow{3}{*}{ 1 } & Description &
            \begin{minipage}[t]{13cm}{\footnotesize
            
            \vspace{\dp0}
            } \end{minipage} \\ \cline{2-3}
            & Test Data &
            \begin{minipage}[t]{13cm}{\footnotesize
                No data.
                \vspace{\dp0}
            } \end{minipage} \\ \cline{2-3}
            & Expected Result &
        \\ \midrule
    \end{longtable}

\subsection{\href{https://jira.lsstcorp.org/secure/Tests.jspa\#/testCase/LVV-T296}{LVV-T296}
    - Data Processing Campaign: Full-survey Key Performance Metrics}\label{lvv-t296}

\begin{longtable}[]{llllll}
\toprule
Version & Status & Priority & Verification Type & Owner
\\\midrule
1 & Draft & Normal &
Demonstration & Keith Bechtol
\\\bottomrule
\end{longtable}

\subsubsection{Verification Elements}
\begin{itemize}
\item \href{https://jira.lsstcorp.org/browse/LVV-1545}{LVV-1545} - OSS-REQ-0228-V-03: Image Quality vs Field

\item \href{https://jira.lsstcorp.org/browse/LVV-7213}{LVV-7213} - OSS-REQ-0228-V-04: Image Quality in Encircled Energy

\item \href{https://jira.lsstcorp.org/browse/LVV-7214}{LVV-7214} - OSS-REQ-0228-V-05: System Contribution to Image Quality

\end{itemize}

\subsubsection{Test Items}


\subsubsection{Predecessors}

\subsubsection{Environment Needs}

\paragraph{Software}

\paragraph{Hardware}

\subsubsection{Input Specification}

\subsubsection{Output Specification}

\subsubsection{Test Procedure}
    \begin{longtable}[]{p{1.3cm}p{2cm}p{13cm}}
    %\toprule
    Step & \multicolumn{2}{@{}l}{Description, Input Data and Expected Result} \\ \toprule
    \endhead

            \multirow{3}{*}{ 1 } & Description &
            \begin{minipage}[t]{13cm}{\footnotesize
            
            \vspace{\dp0}
            } \end{minipage} \\ \cline{2-3}
            & Test Data &
            \begin{minipage}[t]{13cm}{\footnotesize
                No data.
                \vspace{\dp0}
            } \end{minipage} \\ \cline{2-3}
            & Expected Result &
        \\ \midrule
    \end{longtable}

\subsection{\href{https://jira.lsstcorp.org/secure/Tests.jspa\#/testCase/LVV-T297}{LVV-T297}
    - Absolute Astrometric Performance}\label{lvv-t297}

\begin{longtable}[]{llllll}
\toprule
Version & Status & Priority & Verification Type & Owner
\\\midrule
1 & Draft & Normal &
Test & Keith Bechtol
\\\bottomrule
\end{longtable}

\subsubsection{Verification Elements}
\begin{itemize}
\item \href{https://jira.lsstcorp.org/browse/LVV-238}{LVV-238} - LSR-REQ-0094-V-01: Astrometric Performance1

\item \href{https://jira.lsstcorp.org/browse/LVV-1544}{LVV-1544} - OSS-REQ-0228-V-02: Image Size in Pixels

\end{itemize}

\subsubsection{Test Items}
Note: Probably measured with respect to Gaia as external reference



\subsubsection{Predecessors}

\subsubsection{Environment Needs}

\paragraph{Software}

\paragraph{Hardware}

\subsubsection{Input Specification}

\subsubsection{Output Specification}

\subsubsection{Test Procedure}
    \begin{longtable}[]{p{1.3cm}p{2cm}p{13cm}}
    %\toprule
    Step & \multicolumn{2}{@{}l}{Description, Input Data and Expected Result} \\ \toprule
    \endhead

            \multirow{3}{*}{ 1 } & Description &
            \begin{minipage}[t]{13cm}{\footnotesize
            
            \vspace{\dp0}
            } \end{minipage} \\ \cline{2-3}
            & Test Data &
            \begin{minipage}[t]{13cm}{\footnotesize
                No data.
                \vspace{\dp0}
            } \end{minipage} \\ \cline{2-3}
            & Expected Result &
        \\ \midrule
    \end{longtable}

\subsection{\href{https://jira.lsstcorp.org/secure/Tests.jspa\#/testCase/LVV-T298}{LVV-T298}
    - Cross-band Astrometric Performance}\label{lvv-t298}

\begin{longtable}[]{llllll}
\toprule
Version & Status & Priority & Verification Type & Owner
\\\midrule
1 & Draft & Normal &
Test & Keith Bechtol
\\\bottomrule
\end{longtable}

\subsubsection{Verification Elements}
\begin{itemize}
\item \href{https://jira.lsstcorp.org/browse/LVV-1544}{LVV-1544} - OSS-REQ-0228-V-02: Image Size in Pixels

\end{itemize}

\subsubsection{Test Items}


\subsubsection{Predecessors}

\subsubsection{Environment Needs}

\paragraph{Software}

\paragraph{Hardware}

\subsubsection{Input Specification}

\subsubsection{Output Specification}

\subsubsection{Test Procedure}
    \begin{longtable}[]{p{1.3cm}p{2cm}p{13cm}}
    %\toprule
    Step & \multicolumn{2}{@{}l}{Description, Input Data and Expected Result} \\ \toprule
    \endhead

            \multirow{3}{*}{ 1 } & Description &
            \begin{minipage}[t]{13cm}{\footnotesize
            
            \vspace{\dp0}
            } \end{minipage} \\ \cline{2-3}
            & Test Data &
            \begin{minipage}[t]{13cm}{\footnotesize
                No data.
                \vspace{\dp0}
            } \end{minipage} \\ \cline{2-3}
            & Expected Result &
        \\ \midrule
    \end{longtable}

\subsection{\href{https://jira.lsstcorp.org/secure/Tests.jspa\#/testCase/LVV-T299}{LVV-T299}
    - Relative Astrometric Performance}\label{lvv-t299}

\begin{longtable}[]{llllll}
\toprule
Version & Status & Priority & Verification Type & Owner
\\\midrule
1 & Draft & Normal &
Test & Keith Bechtol
\\\bottomrule
\end{longtable}

\subsubsection{Verification Elements}
\begin{itemize}
\item \href{https://jira.lsstcorp.org/browse/LVV-1544}{LVV-1544} - OSS-REQ-0228-V-02: Image Size in Pixels

\end{itemize}

\subsubsection{Test Items}


\subsubsection{Predecessors}

\subsubsection{Environment Needs}

\paragraph{Software}

\paragraph{Hardware}

\subsubsection{Input Specification}

\subsubsection{Output Specification}

\subsubsection{Test Procedure}
    \begin{longtable}[]{p{1.3cm}p{2cm}p{13cm}}
    %\toprule
    Step & \multicolumn{2}{@{}l}{Description, Input Data and Expected Result} \\ \toprule
    \endhead

            \multirow{3}{*}{ 1 } & Description &
            \begin{minipage}[t]{13cm}{\footnotesize
            
            \vspace{\dp0}
            } \end{minipage} \\ \cline{2-3}
            & Test Data &
            \begin{minipage}[t]{13cm}{\footnotesize
                No data.
                \vspace{\dp0}
            } \end{minipage} \\ \cline{2-3}
            & Expected Result &
        \\ \midrule
    \end{longtable}

\subsection{\href{https://jira.lsstcorp.org/secure/Tests.jspa\#/testCase/LVV-T360}{LVV-T360}
    - Off Zenith Image Quality Degradation}\label{lvv-t360}

\begin{longtable}[]{llllll}
\toprule
Version & Status & Priority & Verification Type & Owner
\\\midrule
1 & Draft & Normal &
Analysis & Keith Bechtol
\\\bottomrule
\end{longtable}

\subsubsection{Verification Elements}
\begin{itemize}
\item \href{https://jira.lsstcorp.org/browse/LVV-1543}{LVV-1543} - OSS-REQ-0228-V-01: Image Quality Off-Zenith Degredation

\end{itemize}

\subsubsection{Test Items}


\subsubsection{Predecessors}

\subsubsection{Environment Needs}

\paragraph{Software}

\paragraph{Hardware}

\subsubsection{Input Specification}

\subsubsection{Output Specification}

\subsubsection{Test Procedure}
    \begin{longtable}[]{p{1.3cm}p{2cm}p{13cm}}
    %\toprule
    Step & \multicolumn{2}{@{}l}{Description, Input Data and Expected Result} \\ \toprule
    \endhead

            \multirow{3}{*}{ 1 } & Description &
            \begin{minipage}[t]{13cm}{\footnotesize
            
            \vspace{\dp0}
            } \end{minipage} \\ \cline{2-3}
            & Test Data &
            \begin{minipage}[t]{13cm}{\footnotesize
                No data.
                \vspace{\dp0}
            } \end{minipage} \\ \cline{2-3}
            & Expected Result &
        \\ \midrule
    \end{longtable}

\subsection{\href{https://jira.lsstcorp.org/secure/Tests.jspa\#/testCase/LVV-T361}{LVV-T361}
    - Verify 10-year Ellipticity Residuals w/ On-Sky Data}\label{lvv-t361}

\begin{longtable}[]{llllll}
\toprule
Version & Status & Priority & Verification Type & Owner
\\\midrule
1 & Draft & Normal &
Analysis & Keith Bechtol
\\\bottomrule
\end{longtable}

\subsubsection{Verification Elements}
\begin{itemize}
\item \href{https://jira.lsstcorp.org/browse/LVV-1546}{LVV-1546} - OSS-REQ-0234-V-01: 10-year Ellipticity Residuals

\item \href{https://jira.lsstcorp.org/browse/LVV-1366}{LVV-1366} - OSS-REQ-0390-V-01: Ellipticity Correlations

\end{itemize}

\subsubsection{Test Items}
This is closely related to DMS-REQ-0362. It is likely that we can
recycle some parts of analysis code.



\subsubsection{Predecessors}

\subsubsection{Environment Needs}

\paragraph{Software}

\paragraph{Hardware}

\subsubsection{Input Specification}

\subsubsection{Output Specification}

\subsubsection{Test Procedure}
    \begin{longtable}[]{p{1.3cm}p{2cm}p{13cm}}
    %\toprule
    Step & \multicolumn{2}{@{}l}{Description, Input Data and Expected Result} \\ \toprule
    \endhead

            \multirow{3}{*}{ 1 } & Description &
            \begin{minipage}[t]{13cm}{\footnotesize
            Partition the sky into many smaller regions. We are estimating that
these might be \textasciitilde{}1 deg\^{}2 in solid angle. We need these
to be large enough that we have enough stars to compute an ellipticity
correlation with meaningful statistics in each region.

            \vspace{\dp0}
            } \end{minipage} \\ \cline{2-3}
            & Test Data &
            \begin{minipage}[t]{13cm}{\footnotesize
                No data.
                \vspace{\dp0}
            } \end{minipage} \\ \cline{2-3}
            & Expected Result &
        \\ \midrule

            \multirow{3}{*}{ 2 } & Description &
            \begin{minipage}[t]{13cm}{\footnotesize
            For each region, extract the good quality bright, isolated, point
sources from the database. Columns needed include position, measured
ellipticity e1 and e2 values, PSF model ellipticity e1 and e2 values,
and flag for whether the star was used for PSF modeling.

            \vspace{\dp0}
            } \end{minipage} \\ \cline{2-3}
            & Test Data &
            \begin{minipage}[t]{13cm}{\footnotesize
                No data.
                \vspace{\dp0}
            } \end{minipage} \\ \cline{2-3}
            & Expected Result &
        \\ \midrule

            \multirow{3}{*}{ 3 } & Description &
            \begin{minipage}[t]{13cm}{\footnotesize
            Consider separately stars that were included in PSF modeling and stars
that were not included in PSF modeling. The fiduciual result for this
test should use the stars that were not included in PSF modeling. The
stars that were included in PSF modeling will still be useful for
diagnostic purposes.

            \vspace{\dp0}
            } \end{minipage} \\ \cline{2-3}
            & Test Data &
            \begin{minipage}[t]{13cm}{\footnotesize
                No data.
                \vspace{\dp0}
            } \end{minipage} \\ \cline{2-3}
            & Expected Result &
        \\ \midrule

            \multirow{3}{*}{ 4 } & Description &
            \begin{minipage}[t]{13cm}{\footnotesize
            For each stellar sample in each region, compute the correlation function
of ellipticity residuals over angular scales ranging from less than 1
arcmin to greater than 5 arcmin.

            \vspace{\dp0}
            } \end{minipage} \\ \cline{2-3}
            & Test Data &
            \begin{minipage}[t]{13cm}{\footnotesize
                No data.
                \vspace{\dp0}
            } \end{minipage} \\ \cline{2-3}
            & Expected Result &
        \\ \midrule

            \multirow{3}{*}{ 5 } & Description &
            \begin{minipage}[t]{13cm}{\footnotesize
            Extract the amplitude of the residual ellipticity correlation at angular
scale of 1 arcmin and 5 arcmin in each region.

            \vspace{\dp0}
            } \end{minipage} \\ \cline{2-3}
            & Test Data &
            \begin{minipage}[t]{13cm}{\footnotesize
                No data.
                \vspace{\dp0}
            } \end{minipage} \\ \cline{2-3}
            & Expected Result &
        \\ \midrule

            \multirow{3}{*}{ 6 } & Description &
            \begin{minipage}[t]{13cm}{\footnotesize
            For each angular scale (1 arcmin and 5 arcmin), produce a histogram of
the distribution of residual ellipticity correlation function
amplitudes.

            \vspace{\dp0}
            } \end{minipage} \\ \cline{2-3}
            & Test Data &
            \begin{minipage}[t]{13cm}{\footnotesize
                No data.
                \vspace{\dp0}
            } \end{minipage} \\ \cline{2-3}
            & Expected Result &
        \\ \midrule

            \multirow{3}{*}{ 7 } & Description &
            \begin{minipage}[t]{13cm}{\footnotesize
            Extract the requirement parameters (median and outlier fraction) from
the histrograms corresponding to each angular scale.

            \vspace{\dp0}
            } \end{minipage} \\ \cline{2-3}
            & Test Data &
            \begin{minipage}[t]{13cm}{\footnotesize
                No data.
                \vspace{\dp0}
            } \end{minipage} \\ \cline{2-3}
            & Expected Result &
        \\ \midrule
    \end{longtable}

\subsection{\href{https://jira.lsstcorp.org/secure/Tests.jspa\#/testCase/LVV-T389}{LVV-T389}
    - Single Visit Photometric Repeatability}\label{lvv-t389}

\begin{longtable}[]{llllll}
\toprule
Version & Status & Priority & Verification Type & Owner
\\\midrule
1 & Draft & Normal &
Analysis & Imram Hasan
\\\bottomrule
\end{longtable}

\subsubsection{Verification Elements}
\begin{itemize}
\item \href{https://jira.lsstcorp.org/browse/LVV-273}{LVV-273} - LSR-REQ-0093-V-01: Photometric Performance1

\item \href{https://jira.lsstcorp.org/browse/LVV-1372}{LVV-1372} - OSS-REQ-0387-V-01: Photometric Performance\_1

\end{itemize}

\subsubsection{Test Items}
The rm sof the unresolved source magnitude distribution around the mean
value (repeatability) will not exceed PA1 millimag. No more than PF1 \%
of the measurements will deviate by more than PA2 millimag from the
mean.~



\subsubsection{Predecessors}

\subsubsection{Environment Needs}

\paragraph{Software}

\paragraph{Hardware}

\subsubsection{Input Specification}

\subsubsection{Output Specification}

\subsubsection{Test Procedure}
    \begin{longtable}[]{p{1.3cm}p{2cm}p{13cm}}
    %\toprule
    Step & \multicolumn{2}{@{}l}{Description, Input Data and Expected Result} \\ \toprule
    \endhead

            \multirow{3}{*}{ 1 } & Description &
            \begin{minipage}[t]{13cm}{\footnotesize
            Define sample data of stars to be used in subsequent tests. ~Columns
needed are camera rotation angle, magnitude in all bands, RA, Dec,
detector position.

            \vspace{\dp0}
            } \end{minipage} \\ \cline{2-3}
            & Test Data &
            \begin{minipage}[t]{13cm}{\footnotesize
                LSST Photometry from main sequence, variable, bright, non-saturated
non-resolved point sources. Observations need to span sky position,
camera position, chip position, object color, brightness, airmass, water
vapor content, and time of observation.\\
~\\
Per The LSST SRD, bright means 1-4 magnitudes fainter than saturation
limits.\\
~\\
We estimate this will require at \textasciitilde{} 50,000 objects in
order to overcome statistical noise. This will require about 50**2
degrees of imaging, assuming there are 1000 stars per square degree.
This roughly corresponds to \textasciitilde{}100 ComCam pointing.~

                \vspace{\dp0}
            } \end{minipage} \\ \cline{2-3}
            & Expected Result &
        \\ \midrule

            \multirow{3}{*}{ 2 } & Description &
            \begin{minipage}[t]{13cm}{\footnotesize
            For each star, measure the RMS in each filter. This yields a
distribution of RMS values in each filter. Calculate the median of the
distributions.

            \vspace{\dp0}
            } \end{minipage} \\ \cline{2-3}
            & Test Data &
            \begin{minipage}[t]{13cm}{\footnotesize
                No data.
                \vspace{\dp0}
            } \end{minipage} \\ \cline{2-3}
            & Expected Result &
        \\ \midrule

            \multirow{3}{*}{ 3 } & Description &
            \begin{minipage}[t]{13cm}{\footnotesize
            For each star, calculate the mean magnitude. Calculate the number of
observations which deviate by more than PA2gri (15) millimags for
magnitudes in the g, r, and i bands and PA2uzy (22.5) millimags ~for
magnitudes in the u, z, and y bands.

            \vspace{\dp0}
            } \end{minipage} \\ \cline{2-3}
            & Test Data &
            \begin{minipage}[t]{13cm}{\footnotesize
                No data.
                \vspace{\dp0}
            } \end{minipage} \\ \cline{2-3}
            & Expected Result &
        \\ \midrule

            \multirow{3}{*}{ 4 } & Description &
            \begin{minipage}[t]{13cm}{\footnotesize
            Check the median RMS values for u, z, ~are below PA1uzy (7.5)
millimagniutes

            \vspace{\dp0}
            } \end{minipage} \\ \cline{2-3}
            & Test Data &
            \begin{minipage}[t]{13cm}{\footnotesize
                No data.
                \vspace{\dp0}
            } \end{minipage} \\ \cline{2-3}
            & Expected Result &
        \\ \midrule

            \multirow{3}{*}{ 5 } & Description &
            \begin{minipage}[t]{13cm}{\footnotesize
            Check the median RMS values for g, r, i are below PA1gri (5)
millimagnitudes~

            \vspace{\dp0}
            } \end{minipage} \\ \cline{2-3}
            & Test Data &
            \begin{minipage}[t]{13cm}{\footnotesize
                No data.
                \vspace{\dp0}
            } \end{minipage} \\ \cline{2-3}
            & Expected Result &
        \\ \midrule

            \multirow{3}{*}{ 6 } & Description &
            \begin{minipage}[t]{13cm}{\footnotesize
            Check that less than PF1 (10\%) of measurements deviate from their means
by more than PA2 from step 3

            \vspace{\dp0}
            } \end{minipage} \\ \cline{2-3}
            & Test Data &
            \begin{minipage}[t]{13cm}{\footnotesize
                No data.
                \vspace{\dp0}
            } \end{minipage} \\ \cline{2-3}
            & Expected Result &
        \\ \midrule
    \end{longtable}

\subsection{\href{https://jira.lsstcorp.org/secure/Tests.jspa\#/testCase/LVV-T434}{LVV-T434}
    - Acquire Filter Response Verification Data (with flat field screen)}\label{lvv-t434}

\begin{longtable}[]{llllll}
\toprule
Version & Status & Priority & Verification Type & Owner
\\\midrule
1 & Draft & Normal &
Test & Brian Stalder
\\\bottomrule
\end{longtable}

\subsubsection{Verification Elements}
\begin{itemize}
\item \href{https://jira.lsstcorp.org/browse/LVV-1555}{LVV-1555} - OSS-REQ-0238-V-01: Filter Response Uniformity u-band blue edge

\item \href{https://jira.lsstcorp.org/browse/LVV-1570}{LVV-1570} - OSS-REQ-0239-V-01: In-band Ripple u-band

\item \href{https://jira.lsstcorp.org/browse/LVV-1582}{LVV-1582} - OSS-REQ-0240-V-01: u-band Response Envelope

\item \href{https://jira.lsstcorp.org/browse/LVV-1579}{LVV-1579} - OSS-REQ-0366-V-01: u-band not-to-exceed envelope

\item \href{https://jira.lsstcorp.org/browse/LVV-1561}{LVV-1561} - OSS-REQ-0241-V-01: g-band Response Envelope

\item \href{https://jira.lsstcorp.org/browse/LVV-1558}{LVV-1558} - OSS-REQ-0367-V-01: g-band not-to-exceed envelope

\item \href{https://jira.lsstcorp.org/browse/LVV-1576}{LVV-1576} - OSS-REQ-0242-V-01: r-band Response Envelope

\item \href{https://jira.lsstcorp.org/browse/LVV-1573}{LVV-1573} - OSS-REQ-0368-V-01: r-band not-to-exceed envelope

\item \href{https://jira.lsstcorp.org/browse/LVV-1567}{LVV-1567} - OSS-REQ-0243-V-01: i-band Response Envelope

\item \href{https://jira.lsstcorp.org/browse/LVV-1564}{LVV-1564} - OSS-REQ-0369-V-01: i-band not-to-exceed envelope

\item \href{https://jira.lsstcorp.org/browse/LVV-1594}{LVV-1594} - OSS-REQ-0244-V-01: z-band Response Envelope

\item \href{https://jira.lsstcorp.org/browse/LVV-1591}{LVV-1591} - OSS-REQ-0370-V-01: z-band not-to-exceed envelope

\item \href{https://jira.lsstcorp.org/browse/LVV-1588}{LVV-1588} - OSS-REQ-0245-V-01: y-band Response Envelope

\item \href{https://jira.lsstcorp.org/browse/LVV-1585}{LVV-1585} - OSS-REQ-0371-V-01: y-band not-to-exceed envelope

\item \href{https://jira.lsstcorp.org/browse/LVV-1556}{LVV-1556} - OSS-REQ-0238-V-02: Filter Response Uniformity u-band red edge

\item \href{https://jira.lsstcorp.org/browse/LVV-1557}{LVV-1557} - OSS-REQ-0238-V-03: Filter Response Uniformity g-band blue edge

\item \href{https://jira.lsstcorp.org/browse/LVV-8678}{LVV-8678} - OSS-REQ-0238-V-04: Filter Response Uniformity g-band red edge

\item \href{https://jira.lsstcorp.org/browse/LVV-8681}{LVV-8681} - OSS-REQ-0238-V-05: Filter Response Uniformity r-band blue edge

\item \href{https://jira.lsstcorp.org/browse/LVV-8683}{LVV-8683} - OSS-REQ-0238-V-06: Filter Response Uniformity r-band red edge

\item \href{https://jira.lsstcorp.org/browse/LVV-8686}{LVV-8686} - OSS-REQ-0238-V-07: Filter Response Uniformity i-band blue edge

\item \href{https://jira.lsstcorp.org/browse/LVV-8689}{LVV-8689} - OSS-REQ-0238-V-08: Filter Response Uniformity i-band red edge

\item \href{https://jira.lsstcorp.org/browse/LVV-8691}{LVV-8691} - OSS-REQ-0238-V-09: Filter Response Uniformity z-band blue edge

\item \href{https://jira.lsstcorp.org/browse/LVV-8694}{LVV-8694} - OSS-REQ-0238-V-10: Filter Response Uniformity z-band red edge

\item \href{https://jira.lsstcorp.org/browse/LVV-8696}{LVV-8696} - OSS-REQ-0238-V-11: Filter Response Uniformity y-band blue edge

\item \href{https://jira.lsstcorp.org/browse/LVV-8698}{LVV-8698} - OSS-REQ-0238-V-12: Filter Response Uniformity y-band red edge

\item \href{https://jira.lsstcorp.org/browse/LVV-1572}{LVV-1572} - OSS-REQ-0239-V-03: In-band Ripple r-band

\item \href{https://jira.lsstcorp.org/browse/LVV-1571}{LVV-1571} - OSS-REQ-0239-V-02: In-band Ripple g-band

\end{itemize}

\subsubsection{Test Items}


\subsubsection{Predecessors}

\subsubsection{Environment Needs}

\paragraph{Software}

\paragraph{Hardware}

\subsubsection{Input Specification}

\subsubsection{Output Specification}

\subsubsection{Test Procedure}
    \begin{longtable}[]{p{1.3cm}p{2cm}p{13cm}}
    %\toprule
    Step & \multicolumn{2}{@{}l}{Description, Input Data and Expected Result} \\ \toprule
    \endhead

            \multirow{3}{*}{ 1 } & Description &
            \begin{minipage}[t]{13cm}{\footnotesize
            Prepare relevant calibration products (master bias, dark).

            \vspace{\dp0}
            } \end{minipage} \\ \cline{2-3}
            & Test Data &
            \begin{minipage}[t]{13cm}{\footnotesize
                No data.
                \vspace{\dp0}
            } \end{minipage} \\ \cline{2-3}
            & Expected Result &
        \\ \midrule

            \multirow{3}{*}{ 2 } & Description &
            \begin{minipage}[t]{13cm}{\footnotesize
            Using monochromatic laser on flatfield screen. ~Start with no filter.
~For each wavelength (300-1100nm), ~Take N flat field images.

            \vspace{\dp0}
            } \end{minipage} \\ \cline{2-3}
            & Test Data &
            \begin{minipage}[t]{13cm}{\footnotesize
                No data.
                \vspace{\dp0}
            } \end{minipage} \\ \cline{2-3}
            & Expected Result &
        \\ \midrule

            \multirow{3}{*}{ 3 } & Description &
            \begin{minipage}[t]{13cm}{\footnotesize
            Repeat with same setup with each filter.

            \vspace{\dp0}
            } \end{minipage} \\ \cline{2-3}
            & Test Data &
            \begin{minipage}[t]{13cm}{\footnotesize
                No data.
                \vspace{\dp0}
            } \end{minipage} \\ \cline{2-3}
            & Expected Result &
        \\ \midrule
    \end{longtable}

\subsection{\href{https://jira.lsstcorp.org/secure/Tests.jspa\#/testCase/LVV-T442}{LVV-T442}
    - Control Ghosts in Coadds}\label{lvv-t442}

\begin{longtable}[]{llllll}
\toprule
Version & Status & Priority & Verification Type & Owner
\\\midrule
1 & Draft & Normal &
Analysis & Keith Bechtol
\\\bottomrule
\end{longtable}

\subsubsection{Verification Elements}
    None.

\subsubsection{Test Items}


\subsubsection{Predecessors}

\subsubsection{Environment Needs}

\paragraph{Software}

\paragraph{Hardware}

\subsubsection{Input Specification}

\subsubsection{Output Specification}

\subsubsection{Test Procedure}
    \begin{longtable}[]{p{1.3cm}p{2cm}p{13cm}}
    %\toprule
    Step & \multicolumn{2}{@{}l}{Description, Input Data and Expected Result} \\ \toprule
    \endhead

            \multirow{3}{*}{ 1 } & Description &
            \begin{minipage}[t]{13cm}{\footnotesize
            
            \vspace{\dp0}
            } \end{minipage} \\ \cline{2-3}
            & Test Data &
            \begin{minipage}[t]{13cm}{\footnotesize
                No data.
                \vspace{\dp0}
            } \end{minipage} \\ \cline{2-3}
            & Expected Result &
        \\ \midrule
    \end{longtable}

\subsection{\href{https://jira.lsstcorp.org/secure/Tests.jspa\#/testCase/LVV-T445}{LVV-T445}
    - Acquire Filter Response Verification Data (with CBP)}\label{lvv-t445}

\begin{longtable}[]{llllll}
\toprule
Version & Status & Priority & Verification Type & Owner
\\\midrule
1 & Draft & Normal &
Test & Brian Stalder
\\\bottomrule
\end{longtable}

\subsubsection{Verification Elements}
\begin{itemize}
\item \href{https://jira.lsstcorp.org/browse/LVV-1555}{LVV-1555} - OSS-REQ-0238-V-01: Filter Response Uniformity u-band blue edge

\item \href{https://jira.lsstcorp.org/browse/LVV-1556}{LVV-1556} - OSS-REQ-0238-V-02: Filter Response Uniformity u-band red edge

\item \href{https://jira.lsstcorp.org/browse/LVV-1557}{LVV-1557} - OSS-REQ-0238-V-03: Filter Response Uniformity g-band blue edge

\item \href{https://jira.lsstcorp.org/browse/LVV-8678}{LVV-8678} - OSS-REQ-0238-V-04: Filter Response Uniformity g-band red edge

\item \href{https://jira.lsstcorp.org/browse/LVV-8681}{LVV-8681} - OSS-REQ-0238-V-05: Filter Response Uniformity r-band blue edge

\item \href{https://jira.lsstcorp.org/browse/LVV-8683}{LVV-8683} - OSS-REQ-0238-V-06: Filter Response Uniformity r-band red edge

\item \href{https://jira.lsstcorp.org/browse/LVV-8686}{LVV-8686} - OSS-REQ-0238-V-07: Filter Response Uniformity i-band blue edge

\item \href{https://jira.lsstcorp.org/browse/LVV-8689}{LVV-8689} - OSS-REQ-0238-V-08: Filter Response Uniformity i-band red edge

\item \href{https://jira.lsstcorp.org/browse/LVV-8691}{LVV-8691} - OSS-REQ-0238-V-09: Filter Response Uniformity z-band blue edge

\item \href{https://jira.lsstcorp.org/browse/LVV-8694}{LVV-8694} - OSS-REQ-0238-V-10: Filter Response Uniformity z-band red edge

\item \href{https://jira.lsstcorp.org/browse/LVV-8696}{LVV-8696} - OSS-REQ-0238-V-11: Filter Response Uniformity y-band blue edge

\item \href{https://jira.lsstcorp.org/browse/LVV-8698}{LVV-8698} - OSS-REQ-0238-V-12: Filter Response Uniformity y-band red edge

\item \href{https://jira.lsstcorp.org/browse/LVV-1570}{LVV-1570} - OSS-REQ-0239-V-01: In-band Ripple u-band

\item \href{https://jira.lsstcorp.org/browse/LVV-1572}{LVV-1572} - OSS-REQ-0239-V-03: In-band Ripple r-band

\item \href{https://jira.lsstcorp.org/browse/LVV-1571}{LVV-1571} - OSS-REQ-0239-V-02: In-band Ripple g-band

\item \href{https://jira.lsstcorp.org/browse/LVV-1582}{LVV-1582} - OSS-REQ-0240-V-01: u-band Response Envelope

\item \href{https://jira.lsstcorp.org/browse/LVV-1579}{LVV-1579} - OSS-REQ-0366-V-01: u-band not-to-exceed envelope

\item \href{https://jira.lsstcorp.org/browse/LVV-1561}{LVV-1561} - OSS-REQ-0241-V-01: g-band Response Envelope

\item \href{https://jira.lsstcorp.org/browse/LVV-1558}{LVV-1558} - OSS-REQ-0367-V-01: g-band not-to-exceed envelope

\item \href{https://jira.lsstcorp.org/browse/LVV-1576}{LVV-1576} - OSS-REQ-0242-V-01: r-band Response Envelope

\item \href{https://jira.lsstcorp.org/browse/LVV-1573}{LVV-1573} - OSS-REQ-0368-V-01: r-band not-to-exceed envelope

\item \href{https://jira.lsstcorp.org/browse/LVV-1567}{LVV-1567} - OSS-REQ-0243-V-01: i-band Response Envelope

\item \href{https://jira.lsstcorp.org/browse/LVV-1564}{LVV-1564} - OSS-REQ-0369-V-01: i-band not-to-exceed envelope

\item \href{https://jira.lsstcorp.org/browse/LVV-1594}{LVV-1594} - OSS-REQ-0244-V-01: z-band Response Envelope

\item \href{https://jira.lsstcorp.org/browse/LVV-1591}{LVV-1591} - OSS-REQ-0370-V-01: z-band not-to-exceed envelope

\item \href{https://jira.lsstcorp.org/browse/LVV-1588}{LVV-1588} - OSS-REQ-0245-V-01: y-band Response Envelope

\item \href{https://jira.lsstcorp.org/browse/LVV-1585}{LVV-1585} - OSS-REQ-0371-V-01: y-band not-to-exceed envelope

\end{itemize}

\subsubsection{Test Items}


\subsubsection{Predecessors}

\subsubsection{Environment Needs}

\paragraph{Software}

\paragraph{Hardware}

\subsubsection{Input Specification}

\subsubsection{Output Specification}

\subsubsection{Test Procedure}
    \begin{longtable}[]{p{1.3cm}p{2cm}p{13cm}}
    %\toprule
    Step & \multicolumn{2}{@{}l}{Description, Input Data and Expected Result} \\ \toprule
    \endhead

            \multirow{3}{*}{ 1 } & Description &
            \begin{minipage}[t]{13cm}{\footnotesize
            Prepare relevant calibration products (master bias, dark).

            \vspace{\dp0}
            } \end{minipage} \\ \cline{2-3}
            & Test Data &
            \begin{minipage}[t]{13cm}{\footnotesize
                No data.
                \vspace{\dp0}
            } \end{minipage} \\ \cline{2-3}
            & Expected Result &
        \\ \midrule

            \multirow{3}{*}{ 2 } & Description &
            \begin{minipage}[t]{13cm}{\footnotesize
            Using monochromatic laser on CBP, wide (FP level) mask at a nominal
position and angle. ~Start with no filter. ~For each wavelength
(300-1200nm), ~Take N flat field images.

            \vspace{\dp0}
            } \end{minipage} \\ \cline{2-3}
            & Test Data &
            \begin{minipage}[t]{13cm}{\footnotesize
                No data.
                \vspace{\dp0}
            } \end{minipage} \\ \cline{2-3}
            & Expected Result &
        \\ \midrule

            \multirow{3}{*}{ 3 } & Description &
            \begin{minipage}[t]{13cm}{\footnotesize
            Repeat with same setup with each filter.

            \vspace{\dp0}
            } \end{minipage} \\ \cline{2-3}
            & Test Data &
            \begin{minipage}[t]{13cm}{\footnotesize
                No data.
                \vspace{\dp0}
            } \end{minipage} \\ \cline{2-3}
            & Expected Result &
        \\ \midrule

            \multirow{3}{*}{ 4 } & Description &
            \begin{minipage}[t]{13cm}{\footnotesize
            Repeat 1-3 at N angles of incident.

            \vspace{\dp0}
            } \end{minipage} \\ \cline{2-3}
            & Test Data &
            \begin{minipage}[t]{13cm}{\footnotesize
                No data.
                \vspace{\dp0}
            } \end{minipage} \\ \cline{2-3}
            & Expected Result &
        \\ \midrule
    \end{longtable}

\subsection{\href{https://jira.lsstcorp.org/secure/Tests.jspa\#/testCase/LVV-T446}{LVV-T446}
    - Process Filter Response Verification Data (with flat field screen)}\label{lvv-t446}

\begin{longtable}[]{llllll}
\toprule
Version & Status & Priority & Verification Type & Owner
\\\midrule
1 & Draft & Normal &
Test & Brian Stalder
\\\bottomrule
\end{longtable}

\subsubsection{Verification Elements}
\begin{itemize}
\item \href{https://jira.lsstcorp.org/browse/LVV-1555}{LVV-1555} - OSS-REQ-0238-V-01: Filter Response Uniformity u-band blue edge

\item \href{https://jira.lsstcorp.org/browse/LVV-1556}{LVV-1556} - OSS-REQ-0238-V-02: Filter Response Uniformity u-band red edge

\item \href{https://jira.lsstcorp.org/browse/LVV-1557}{LVV-1557} - OSS-REQ-0238-V-03: Filter Response Uniformity g-band blue edge

\item \href{https://jira.lsstcorp.org/browse/LVV-8678}{LVV-8678} - OSS-REQ-0238-V-04: Filter Response Uniformity g-band red edge

\item \href{https://jira.lsstcorp.org/browse/LVV-8681}{LVV-8681} - OSS-REQ-0238-V-05: Filter Response Uniformity r-band blue edge

\item \href{https://jira.lsstcorp.org/browse/LVV-8683}{LVV-8683} - OSS-REQ-0238-V-06: Filter Response Uniformity r-band red edge

\item \href{https://jira.lsstcorp.org/browse/LVV-8686}{LVV-8686} - OSS-REQ-0238-V-07: Filter Response Uniformity i-band blue edge

\item \href{https://jira.lsstcorp.org/browse/LVV-8689}{LVV-8689} - OSS-REQ-0238-V-08: Filter Response Uniformity i-band red edge

\item \href{https://jira.lsstcorp.org/browse/LVV-8691}{LVV-8691} - OSS-REQ-0238-V-09: Filter Response Uniformity z-band blue edge

\item \href{https://jira.lsstcorp.org/browse/LVV-8694}{LVV-8694} - OSS-REQ-0238-V-10: Filter Response Uniformity z-band red edge

\item \href{https://jira.lsstcorp.org/browse/LVV-8696}{LVV-8696} - OSS-REQ-0238-V-11: Filter Response Uniformity y-band blue edge

\item \href{https://jira.lsstcorp.org/browse/LVV-8698}{LVV-8698} - OSS-REQ-0238-V-12: Filter Response Uniformity y-band red edge

\item \href{https://jira.lsstcorp.org/browse/LVV-1570}{LVV-1570} - OSS-REQ-0239-V-01: In-band Ripple u-band

\item \href{https://jira.lsstcorp.org/browse/LVV-1572}{LVV-1572} - OSS-REQ-0239-V-03: In-band Ripple r-band

\item \href{https://jira.lsstcorp.org/browse/LVV-1571}{LVV-1571} - OSS-REQ-0239-V-02: In-band Ripple g-band

\item \href{https://jira.lsstcorp.org/browse/LVV-1582}{LVV-1582} - OSS-REQ-0240-V-01: u-band Response Envelope

\item \href{https://jira.lsstcorp.org/browse/LVV-1579}{LVV-1579} - OSS-REQ-0366-V-01: u-band not-to-exceed envelope

\item \href{https://jira.lsstcorp.org/browse/LVV-1561}{LVV-1561} - OSS-REQ-0241-V-01: g-band Response Envelope

\item \href{https://jira.lsstcorp.org/browse/LVV-1558}{LVV-1558} - OSS-REQ-0367-V-01: g-band not-to-exceed envelope

\item \href{https://jira.lsstcorp.org/browse/LVV-1576}{LVV-1576} - OSS-REQ-0242-V-01: r-band Response Envelope

\item \href{https://jira.lsstcorp.org/browse/LVV-1573}{LVV-1573} - OSS-REQ-0368-V-01: r-band not-to-exceed envelope

\item \href{https://jira.lsstcorp.org/browse/LVV-1567}{LVV-1567} - OSS-REQ-0243-V-01: i-band Response Envelope

\item \href{https://jira.lsstcorp.org/browse/LVV-1564}{LVV-1564} - OSS-REQ-0369-V-01: i-band not-to-exceed envelope

\item \href{https://jira.lsstcorp.org/browse/LVV-1594}{LVV-1594} - OSS-REQ-0244-V-01: z-band Response Envelope

\item \href{https://jira.lsstcorp.org/browse/LVV-1591}{LVV-1591} - OSS-REQ-0370-V-01: z-band not-to-exceed envelope

\item \href{https://jira.lsstcorp.org/browse/LVV-1588}{LVV-1588} - OSS-REQ-0245-V-01: y-band Response Envelope

\item \href{https://jira.lsstcorp.org/browse/LVV-1585}{LVV-1585} - OSS-REQ-0371-V-01: y-band not-to-exceed envelope

\end{itemize}

\subsubsection{Test Items}


\subsubsection{Predecessors}

\subsubsection{Environment Needs}

\paragraph{Software}

\paragraph{Hardware}

\subsubsection{Input Specification}

\subsubsection{Output Specification}

\subsubsection{Test Procedure}
    \begin{longtable}[]{p{1.3cm}p{2cm}p{13cm}}
    %\toprule
    Step & \multicolumn{2}{@{}l}{Description, Input Data and Expected Result} \\ \toprule
    \endhead

            \multirow{3}{*}{ 1 } & Description &
            \begin{minipage}[t]{13cm}{\footnotesize
            For each flat, normalize by photodiode output and take ratio (filter
in/out). ~This is the filter throughput realized on the focal plane at
each wavelength. ~

            \vspace{\dp0}
            } \end{minipage} \\ \cline{2-3}
            & Test Data &
            \begin{minipage}[t]{13cm}{\footnotesize
                No data.
                \vspace{\dp0}
            } \end{minipage} \\ \cline{2-3}
            & Expected Result &
        \\ \midrule
    \end{longtable}

\subsection{\href{https://jira.lsstcorp.org/secure/Tests.jspa\#/testCase/LVV-T447}{LVV-T447}
    - Process Filter Response Verification Data (with CBP)}\label{lvv-t447}

\begin{longtable}[]{llllll}
\toprule
Version & Status & Priority & Verification Type & Owner
\\\midrule
1 & Draft & Normal &
Test & Brian Stalder
\\\bottomrule
\end{longtable}

\subsubsection{Verification Elements}
\begin{itemize}
\item \href{https://jira.lsstcorp.org/browse/LVV-1555}{LVV-1555} - OSS-REQ-0238-V-01: Filter Response Uniformity u-band blue edge

\item \href{https://jira.lsstcorp.org/browse/LVV-1556}{LVV-1556} - OSS-REQ-0238-V-02: Filter Response Uniformity u-band red edge

\item \href{https://jira.lsstcorp.org/browse/LVV-1557}{LVV-1557} - OSS-REQ-0238-V-03: Filter Response Uniformity g-band blue edge

\item \href{https://jira.lsstcorp.org/browse/LVV-8678}{LVV-8678} - OSS-REQ-0238-V-04: Filter Response Uniformity g-band red edge

\item \href{https://jira.lsstcorp.org/browse/LVV-8681}{LVV-8681} - OSS-REQ-0238-V-05: Filter Response Uniformity r-band blue edge

\item \href{https://jira.lsstcorp.org/browse/LVV-8683}{LVV-8683} - OSS-REQ-0238-V-06: Filter Response Uniformity r-band red edge

\item \href{https://jira.lsstcorp.org/browse/LVV-8686}{LVV-8686} - OSS-REQ-0238-V-07: Filter Response Uniformity i-band blue edge

\item \href{https://jira.lsstcorp.org/browse/LVV-8689}{LVV-8689} - OSS-REQ-0238-V-08: Filter Response Uniformity i-band red edge

\item \href{https://jira.lsstcorp.org/browse/LVV-8691}{LVV-8691} - OSS-REQ-0238-V-09: Filter Response Uniformity z-band blue edge

\item \href{https://jira.lsstcorp.org/browse/LVV-8694}{LVV-8694} - OSS-REQ-0238-V-10: Filter Response Uniformity z-band red edge

\item \href{https://jira.lsstcorp.org/browse/LVV-8696}{LVV-8696} - OSS-REQ-0238-V-11: Filter Response Uniformity y-band blue edge

\item \href{https://jira.lsstcorp.org/browse/LVV-8698}{LVV-8698} - OSS-REQ-0238-V-12: Filter Response Uniformity y-band red edge

\item \href{https://jira.lsstcorp.org/browse/LVV-1570}{LVV-1570} - OSS-REQ-0239-V-01: In-band Ripple u-band

\item \href{https://jira.lsstcorp.org/browse/LVV-1572}{LVV-1572} - OSS-REQ-0239-V-03: In-band Ripple r-band

\item \href{https://jira.lsstcorp.org/browse/LVV-1571}{LVV-1571} - OSS-REQ-0239-V-02: In-band Ripple g-band

\item \href{https://jira.lsstcorp.org/browse/LVV-1582}{LVV-1582} - OSS-REQ-0240-V-01: u-band Response Envelope

\item \href{https://jira.lsstcorp.org/browse/LVV-1579}{LVV-1579} - OSS-REQ-0366-V-01: u-band not-to-exceed envelope

\item \href{https://jira.lsstcorp.org/browse/LVV-1561}{LVV-1561} - OSS-REQ-0241-V-01: g-band Response Envelope

\item \href{https://jira.lsstcorp.org/browse/LVV-1558}{LVV-1558} - OSS-REQ-0367-V-01: g-band not-to-exceed envelope

\item \href{https://jira.lsstcorp.org/browse/LVV-1576}{LVV-1576} - OSS-REQ-0242-V-01: r-band Response Envelope

\item \href{https://jira.lsstcorp.org/browse/LVV-1573}{LVV-1573} - OSS-REQ-0368-V-01: r-band not-to-exceed envelope

\item \href{https://jira.lsstcorp.org/browse/LVV-1567}{LVV-1567} - OSS-REQ-0243-V-01: i-band Response Envelope

\item \href{https://jira.lsstcorp.org/browse/LVV-1564}{LVV-1564} - OSS-REQ-0369-V-01: i-band not-to-exceed envelope

\item \href{https://jira.lsstcorp.org/browse/LVV-1594}{LVV-1594} - OSS-REQ-0244-V-01: z-band Response Envelope

\item \href{https://jira.lsstcorp.org/browse/LVV-1591}{LVV-1591} - OSS-REQ-0370-V-01: z-band not-to-exceed envelope

\item \href{https://jira.lsstcorp.org/browse/LVV-1588}{LVV-1588} - OSS-REQ-0245-V-01: y-band Response Envelope

\item \href{https://jira.lsstcorp.org/browse/LVV-1585}{LVV-1585} - OSS-REQ-0371-V-01: y-band not-to-exceed envelope

\end{itemize}

\subsubsection{Test Items}


\subsubsection{Predecessors}

\subsubsection{Environment Needs}

\paragraph{Software}

\paragraph{Hardware}

\subsubsection{Input Specification}

\subsubsection{Output Specification}

\subsubsection{Test Procedure}
    \begin{longtable}[]{p{1.3cm}p{2cm}p{13cm}}
    %\toprule
    Step & \multicolumn{2}{@{}l}{Description, Input Data and Expected Result} \\ \toprule
    \endhead

            \multirow{3}{*}{ 1 } & Description &
            \begin{minipage}[t]{13cm}{\footnotesize
            For each image, normalize by photodiode output and take ratio (filter
in/out). ~This is the filter throughput at the individual points on the
focal plane at each wavelength. ~

            \vspace{\dp0}
            } \end{minipage} \\ \cline{2-3}
            & Test Data &
            \begin{minipage}[t]{13cm}{\footnotesize
                No data.
                \vspace{\dp0}
            } \end{minipage} \\ \cline{2-3}
            & Expected Result &
        \\ \midrule
    \end{longtable}

\subsection{\href{https://jira.lsstcorp.org/secure/Tests.jspa\#/testCase/LVV-T448}{LVV-T448}
    - Analyze Filter Response Uniformity}\label{lvv-t448}

\begin{longtable}[]{llllll}
\toprule
Version & Status & Priority & Verification Type & Owner
\\\midrule
1 & Draft & Normal &
Test & Brian Stalder
\\\bottomrule
\end{longtable}

\subsubsection{Verification Elements}
\begin{itemize}
\item \href{https://jira.lsstcorp.org/browse/LVV-1555}{LVV-1555} - OSS-REQ-0238-V-01: Filter Response Uniformity u-band blue edge

\item \href{https://jira.lsstcorp.org/browse/LVV-1556}{LVV-1556} - OSS-REQ-0238-V-02: Filter Response Uniformity u-band red edge

\item \href{https://jira.lsstcorp.org/browse/LVV-1557}{LVV-1557} - OSS-REQ-0238-V-03: Filter Response Uniformity g-band blue edge

\item \href{https://jira.lsstcorp.org/browse/LVV-8678}{LVV-8678} - OSS-REQ-0238-V-04: Filter Response Uniformity g-band red edge

\item \href{https://jira.lsstcorp.org/browse/LVV-8681}{LVV-8681} - OSS-REQ-0238-V-05: Filter Response Uniformity r-band blue edge

\item \href{https://jira.lsstcorp.org/browse/LVV-8683}{LVV-8683} - OSS-REQ-0238-V-06: Filter Response Uniformity r-band red edge

\item \href{https://jira.lsstcorp.org/browse/LVV-8686}{LVV-8686} - OSS-REQ-0238-V-07: Filter Response Uniformity i-band blue edge

\item \href{https://jira.lsstcorp.org/browse/LVV-8689}{LVV-8689} - OSS-REQ-0238-V-08: Filter Response Uniformity i-band red edge

\item \href{https://jira.lsstcorp.org/browse/LVV-8691}{LVV-8691} - OSS-REQ-0238-V-09: Filter Response Uniformity z-band blue edge

\item \href{https://jira.lsstcorp.org/browse/LVV-8694}{LVV-8694} - OSS-REQ-0238-V-10: Filter Response Uniformity z-band red edge

\item \href{https://jira.lsstcorp.org/browse/LVV-8696}{LVV-8696} - OSS-REQ-0238-V-11: Filter Response Uniformity y-band blue edge

\item \href{https://jira.lsstcorp.org/browse/LVV-8698}{LVV-8698} - OSS-REQ-0238-V-12: Filter Response Uniformity y-band red edge

\end{itemize}

\subsubsection{Test Items}


\subsubsection{Predecessors}

\subsubsection{Environment Needs}

\paragraph{Software}

\paragraph{Hardware}

\subsubsection{Input Specification}

\subsubsection{Output Specification}

\subsubsection{Test Procedure}
    \begin{longtable}[]{p{1.3cm}p{2cm}p{13cm}}
    %\toprule
    Step & \multicolumn{2}{@{}l}{Description, Input Data and Expected Result} \\ \toprule
    \endhead

            \multirow{3}{*}{ 1 } & Description &
            \begin{minipage}[t]{13cm}{\footnotesize
            Measure the wavelength of 50\% maximum throughput edge for each position
sampled. ~This wavelength shall not deviate from spatially-weighted mean
by more than \textbf{filtUniformity\_u} for u-band (blue and red edges)
and \textbf{filtUniformity\_grizy} for all other filters.

            \vspace{\dp0}
            } \end{minipage} \\ \cline{2-3}
            & Test Data &
            \begin{minipage}[t]{13cm}{\footnotesize
                No data.
                \vspace{\dp0}
            } \end{minipage} \\ \cline{2-3}
            & Expected Result &
        \\ \midrule

            \multirow{3}{*}{ 2 } & Description &
            \begin{minipage}[t]{13cm}{\footnotesize
            Repeat for each filter/edge.

            \vspace{\dp0}
            } \end{minipage} \\ \cline{2-3}
            & Test Data &
            \begin{minipage}[t]{13cm}{\footnotesize
                No data.
                \vspace{\dp0}
            } \end{minipage} \\ \cline{2-3}
            & Expected Result &
        \\ \midrule
    \end{longtable}

\subsection{\href{https://jira.lsstcorp.org/secure/Tests.jspa\#/testCase/LVV-T452}{LVV-T452}
    - Analyze in-band ripple}\label{lvv-t452}

\begin{longtable}[]{llllll}
\toprule
Version & Status & Priority & Verification Type & Owner
\\\midrule
1 & Draft & Normal &
Test & Brian Stalder
\\\bottomrule
\end{longtable}

\subsubsection{Verification Elements}
\begin{itemize}
\item \href{https://jira.lsstcorp.org/browse/LVV-1570}{LVV-1570} - OSS-REQ-0239-V-01: In-band Ripple u-band

\item \href{https://jira.lsstcorp.org/browse/LVV-1572}{LVV-1572} - OSS-REQ-0239-V-03: In-band Ripple r-band

\item \href{https://jira.lsstcorp.org/browse/LVV-1571}{LVV-1571} - OSS-REQ-0239-V-02: In-band Ripple g-band

\end{itemize}

\subsubsection{Test Items}


\subsubsection{Predecessors}

\subsubsection{Environment Needs}

\paragraph{Software}

\paragraph{Hardware}

\subsubsection{Input Specification}

\subsubsection{Output Specification}

\subsubsection{Test Procedure}
    \begin{longtable}[]{p{1.3cm}p{2cm}p{13cm}}
    %\toprule
    Step & \multicolumn{2}{@{}l}{Description, Input Data and Expected Result} \\ \toprule
    \endhead

            \multirow{3}{*}{ 1 } & Description &
            \begin{minipage}[t]{13cm}{\footnotesize
            For each point (in clear aperture) sampled, measure the peak-to-valley
variation from mean throughput response. ~This deviation shall not
exceed \textbf{maxFiltRipple}.

            \vspace{\dp0}
            } \end{minipage} \\ \cline{2-3}
            & Test Data &
            \begin{minipage}[t]{13cm}{\footnotesize
                No data.
                \vspace{\dp0}
            } \end{minipage} \\ \cline{2-3}
            & Expected Result &
        \\ \midrule

            \multirow{3}{*}{ 2 } & Description &
            \begin{minipage}[t]{13cm}{\footnotesize
            Repeat for each filter.

            \vspace{\dp0}
            } \end{minipage} \\ \cline{2-3}
            & Test Data &
            \begin{minipage}[t]{13cm}{\footnotesize
                No data.
                \vspace{\dp0}
            } \end{minipage} \\ \cline{2-3}
            & Expected Result &
        \\ \midrule
    \end{longtable}

\subsection{\href{https://jira.lsstcorp.org/secure/Tests.jspa\#/testCase/LVV-T453}{LVV-T453}
    - Analyze Filter Response Envelope}\label{lvv-t453}

\begin{longtable}[]{llllll}
\toprule
Version & Status & Priority & Verification Type & Owner
\\\midrule
1 & Draft & Normal &
Test & Brian Stalder
\\\bottomrule
\end{longtable}

\subsubsection{Verification Elements}
\begin{itemize}
\item \href{https://jira.lsstcorp.org/browse/LVV-1582}{LVV-1582} - OSS-REQ-0240-V-01: u-band Response Envelope

\item \href{https://jira.lsstcorp.org/browse/LVV-1579}{LVV-1579} - OSS-REQ-0366-V-01: u-band not-to-exceed envelope

\item \href{https://jira.lsstcorp.org/browse/LVV-1561}{LVV-1561} - OSS-REQ-0241-V-01: g-band Response Envelope

\item \href{https://jira.lsstcorp.org/browse/LVV-1558}{LVV-1558} - OSS-REQ-0367-V-01: g-band not-to-exceed envelope

\item \href{https://jira.lsstcorp.org/browse/LVV-1576}{LVV-1576} - OSS-REQ-0242-V-01: r-band Response Envelope

\item \href{https://jira.lsstcorp.org/browse/LVV-1573}{LVV-1573} - OSS-REQ-0368-V-01: r-band not-to-exceed envelope

\item \href{https://jira.lsstcorp.org/browse/LVV-1567}{LVV-1567} - OSS-REQ-0243-V-01: i-band Response Envelope

\item \href{https://jira.lsstcorp.org/browse/LVV-1564}{LVV-1564} - OSS-REQ-0369-V-01: i-band not-to-exceed envelope

\item \href{https://jira.lsstcorp.org/browse/LVV-1594}{LVV-1594} - OSS-REQ-0244-V-01: z-band Response Envelope

\item \href{https://jira.lsstcorp.org/browse/LVV-1591}{LVV-1591} - OSS-REQ-0370-V-01: z-band not-to-exceed envelope

\item \href{https://jira.lsstcorp.org/browse/LVV-1588}{LVV-1588} - OSS-REQ-0245-V-01: y-band Response Envelope

\item \href{https://jira.lsstcorp.org/browse/LVV-1585}{LVV-1585} - OSS-REQ-0371-V-01: y-band not-to-exceed envelope

\end{itemize}

\subsubsection{Test Items}


\subsubsection{Predecessors}

\subsubsection{Environment Needs}

\paragraph{Software}

\paragraph{Hardware}

\subsubsection{Input Specification}

\subsubsection{Output Specification}

\subsubsection{Test Procedure}
    \begin{longtable}[]{p{1.3cm}p{2cm}p{13cm}}
    %\toprule
    Step & \multicolumn{2}{@{}l}{Description, Input Data and Expected Result} \\ \toprule
    \endhead

            \multirow{3}{*}{ 1 } & Description &
            \begin{minipage}[t]{13cm}{\footnotesize
            For each position sampled, verify that the response is within the
envelope and not the not to exceed envelope.

            \vspace{\dp0}
            } \end{minipage} \\ \cline{2-3}
            & Test Data &
            \begin{minipage}[t]{13cm}{\footnotesize
                No data.
                \vspace{\dp0}
            } \end{minipage} \\ \cline{2-3}
            & Expected Result &
        \\ \midrule
    \end{longtable}

\subsection{\href{https://jira.lsstcorp.org/secure/Tests.jspa\#/testCase/LVV-T461}{LVV-T461}
    - Filter Out of Band Constraints}\label{lvv-t461}

\begin{longtable}[]{llllll}
\toprule
Version & Status & Priority & Verification Type & Owner
\\\midrule
1 & Draft & Normal &
Analysis & Robert Morgan
\\\bottomrule
\end{longtable}

\subsubsection{Verification Elements}
    None.

\subsubsection{Test Items}
Verify that the out-of-band filter transmission (leakage) is low enough
to pass OSS-237



\subsubsection{Predecessors}

\subsubsection{Environment Needs}

\paragraph{Software}

\paragraph{Hardware}

\subsubsection{Input Specification}
Transmission curves for each filter and amp have been obtained using the
collimated beam projector.


\subsubsection{Output Specification}

\subsubsection{Test Procedure}
    \begin{longtable}[]{p{1.3cm}p{2cm}p{13cm}}
    %\toprule
    Step & \multicolumn{2}{@{}l}{Description, Input Data and Expected Result} \\ \toprule
    \endhead

            \multirow{3}{*}{ 1 } & Description &
            \begin{minipage}[t]{13cm}{\footnotesize
            Data Collection

            \vspace{\dp0}
            } \end{minipage} \\ \cline{2-3}
            & Test Data &
            \begin{minipage}[t]{13cm}{\footnotesize
                Using the collimated beam projector, each amplifier on the focal plane
is to be targeted with wavelengths 300 nm to 1200 nm. After ISR is
applied, the incident light and the transmitted light are to be compared
such that filter transmission as a function of wavelength is known. This
process is to be repeated for each filter.

                \vspace{\dp0}
            } \end{minipage} \\ \cline{2-3}
            & Expected Result &
        \\ \midrule

            \multirow{3}{*}{ 2 } & Description &
            \begin{minipage}[t]{13cm}{\footnotesize
            For each filter and amplifier, read in the transmission curve.

            \vspace{\dp0}
            } \end{minipage} \\ \cline{2-3}
            & Test Data &
            \begin{minipage}[t]{13cm}{\footnotesize
                Transmission curve for each filter and amplifier.

                \vspace{\dp0}
            } \end{minipage} \\ \cline{2-3}
            & Expected Result &
        \\ \midrule

            \multirow{3}{*}{ 3 } & Description &
            \begin{minipage}[t]{13cm}{\footnotesize
            Calculate the central wavelength and FWHM of the transmission curve.
Select the region of the curve between 300 nm and 1200 nm, but outside
one FWHM from the central wavelength.

            \vspace{\dp0}
            } \end{minipage} \\ \cline{2-3}
            & Test Data &
            \begin{minipage}[t]{13cm}{\footnotesize
                Transmission curve for each filter and amplifier.

                \vspace{\dp0}
            } \end{minipage} \\ \cline{2-3}
            & Expected Result &
        \\ \midrule

            \multirow{3}{*}{ 4 } & Description &
            \begin{minipage}[t]{13cm}{\footnotesize
            Bin the transmission curve values in 10 nm segments.

            \vspace{\dp0}
            } \end{minipage} \\ \cline{2-3}
            & Test Data &
            \begin{minipage}[t]{13cm}{\footnotesize
                Out-of-band transmission curve for each filter and amplifier.

                \vspace{\dp0}
            } \end{minipage} \\ \cline{2-3}
            & Expected Result &
        \\ \midrule

            \multirow{3}{*}{ 5 } & Description &
            \begin{minipage}[t]{13cm}{\footnotesize
            Iterate through the segments and obtain the following quantities:

\begin{enumerate}
\tightlist
\item
  The average transmission in each segment
\item
  The maximum transmission in each segment
\item
  The peak transmission of all segments
\item
  The number of segments with maximum transmission larger than 0.01
  percent
\item
  The cumulative integrated transmission

  \begin{enumerate}
  \tightlist
  \item
    Above wavelengths of 1050 nm, multiply the filter response by the
    silicon response when performing the integration.
  \end{enumerate}
\end{enumerate}

            \vspace{\dp0}
            } \end{minipage} \\ \cline{2-3}
            & Test Data &
            \begin{minipage}[t]{13cm}{\footnotesize
                Binned out-of-band transmission curve for each filter and amplifier.

                \vspace{\dp0}
            } \end{minipage} \\ \cline{2-3}
            & Expected Result &
        \\ \midrule

            \multirow{3}{*}{ 6 } & Description &
            \begin{minipage}[t]{13cm}{\footnotesize
            Calculate the percent of segments with maximum transmission larger than
0.01 percent

            \vspace{\dp0}
            } \end{minipage} \\ \cline{2-3}
            & Test Data &
            \begin{minipage}[t]{13cm}{\footnotesize
                Number of segments with maximum transmission larger than 0.01 percent
obtained from the iteration.

                \vspace{\dp0}
            } \end{minipage} \\ \cline{2-3}
            & Expected Result &
        \\ \midrule

            \multirow{3}{*}{ 7 } & Description &
            \begin{minipage}[t]{13cm}{\footnotesize
            Iterate through the segments a second time to get the cumulative
integrated transmission after the wavelength where the transmission
falls below 0.1 percent of the peak transmission of all the segments
found from the previous iteration.

            \vspace{\dp0}
            } \end{minipage} \\ \cline{2-3}
            & Test Data &
            \begin{minipage}[t]{13cm}{\footnotesize
                Binned out-of-band transmission curve for each filter and amplifier.

                \vspace{\dp0}
            } \end{minipage} \\ \cline{2-3}
            & Expected Result &
        \\ \midrule

            \multirow{3}{*}{ 8 } & Description &
            \begin{minipage}[t]{13cm}{\footnotesize
            Calculate the ratio of the cumulative integrated transmission after the
wavelength where the transmission falls below 0.1 percent of the peak
transmission of all the segments to the total integrated transmission of
all segments.

            \vspace{\dp0}
            } \end{minipage} \\ \cline{2-3}
            & Test Data &
            \begin{minipage}[t]{13cm}{\footnotesize
                Integrated transmission after 0.1 percent of peak obtained from second
iteration and total integrated transmission obtained from first
iteration.

                \vspace{\dp0}
            } \end{minipage} \\ \cline{2-3}
            & Expected Result &
        \\ \midrule

            \multirow{3}{*}{ 9 } & Description &
            \begin{minipage}[t]{13cm}{\footnotesize
            Check that all of the average transmissions are below 0.01 percent.

            \vspace{\dp0}
            } \end{minipage} \\ \cline{2-3}
            & Test Data &
            \begin{minipage}[t]{13cm}{\footnotesize
                Average transmission in each 10 nm segment found during first iteration.

                \vspace{\dp0}
            } \end{minipage} \\ \cline{2-3}
            & Expected Result &
        \\ \midrule

            \multirow{3}{*}{ 10 } & Description &
            \begin{minipage}[t]{13cm}{\footnotesize
            Check that all of the maximum transmissions are below 0.1 percent.

            \vspace{\dp0}
            } \end{minipage} \\ \cline{2-3}
            & Test Data &
            \begin{minipage}[t]{13cm}{\footnotesize
                Maximum transmission in each 10 nm segment found during first iteration.

                \vspace{\dp0}
            } \end{minipage} \\ \cline{2-3}
            & Expected Result &
        \\ \midrule

            \multirow{3}{*}{ 11 } & Description &
            \begin{minipage}[t]{13cm}{\footnotesize
            Check that the percentage of segments with maximum transmission larger
than 0.01 percent is below 5.0 percent.

            \vspace{\dp0}
            } \end{minipage} \\ \cline{2-3}
            & Test Data &
            \begin{minipage}[t]{13cm}{\footnotesize
                Percent of segments with maximum transmission larger than 0.01 percent
obtained after first iteration.

                \vspace{\dp0}
            } \end{minipage} \\ \cline{2-3}
            & Expected Result &
        \\ \midrule

            \multirow{3}{*}{ 12 } & Description &
            \begin{minipage}[t]{13cm}{\footnotesize
            Check that the ratio of the cumulative integrated transmission after the
wavelength where the transmission falls below 0.1 percent of the peak
transmission of all the segments to the total integrated transmission of
all segments is below 0.03 percent.

            \vspace{\dp0}
            } \end{minipage} \\ \cline{2-3}
            & Test Data &
            \begin{minipage}[t]{13cm}{\footnotesize
                The ratio of the cumulative integrated transmission after the wavelength
where the transmission falls below 0.1 percent of the peak transmission
of all the segments to the total integrated transmission of all segments
obtained after the second iteration.

                \vspace{\dp0}
            } \end{minipage} \\ \cline{2-3}
            & Expected Result &
        \\ \midrule
    \end{longtable}

\subsection{\href{https://jira.lsstcorp.org/secure/Tests.jspa\#/testCase/LVV-T532}{LVV-T532}
    - MOPS completeness threshold}\label{lvv-t532}

\begin{longtable}[]{llllll}
\toprule
Version & Status & Priority & Verification Type & Owner
\\\midrule
1 & Draft & Normal &
Test & Scott Daniel
\\\bottomrule
\end{longtable}

\subsubsection{Verification Elements}
\begin{itemize}
\item \href{https://jira.lsstcorp.org/browse/LVV-1261}{LVV-1261} - OSS-REQ-0354-V-01: Difference Source Spuriousness Threshold - MOPS

\end{itemize}

\subsubsection{Test Items}


\subsubsection{Predecessors}

\subsubsection{Environment Needs}

\paragraph{Software}

\paragraph{Hardware}

\subsubsection{Input Specification}

\subsubsection{Output Specification}

\subsubsection{Test Procedure}
    \begin{longtable}[]{p{1.3cm}p{2cm}p{13cm}}
    %\toprule
    Step & \multicolumn{2}{@{}l}{Description, Input Data and Expected Result} \\ \toprule
    \endhead

            \multirow{3}{*}{ 1 } & Description &
            \begin{minipage}[t]{13cm}{\footnotesize
            Generate catalog of simulated variable/transient sources

            \vspace{\dp0}
            } \end{minipage} \\ \cline{2-3}
            & Test Data &
            \begin{minipage}[t]{13cm}{\footnotesize
                No data.
                \vspace{\dp0}
            } \end{minipage} \\ \cline{2-3}
            & Expected Result &
        \\ \midrule

            \multirow{3}{*}{ 2 } & Description &
            \begin{minipage}[t]{13cm}{\footnotesize
            Inject simulated variables/transients into actual images

            \vspace{\dp0}
            } \end{minipage} \\ \cline{2-3}
            & Test Data &
            \begin{minipage}[t]{13cm}{\footnotesize
                Catalog of simulated objects from step 1

                \vspace{\dp0}
            } \end{minipage} \\ \cline{2-3}
            & Expected Result &
                \begin{minipage}[t]{13cm}{\footnotesize
                Set of images with simulated variables/transients

                \vspace{\dp0}
                } \end{minipage}
        \\ \midrule

            \multirow{3}{*}{ 3 } & Description &
            \begin{minipage}[t]{13cm}{\footnotesize
            Run difference imaging on images with injected variables/transients

            \vspace{\dp0}
            } \end{minipage} \\ \cline{2-3}
            & Test Data &
            \begin{minipage}[t]{13cm}{\footnotesize
                Images from step 2

                \vspace{\dp0}
            } \end{minipage} \\ \cline{2-3}
            & Expected Result &
                \begin{minipage}[t]{13cm}{\footnotesize
                Catalog of detected DIASources

                \vspace{\dp0}
                } \end{minipage}
        \\ \midrule

            \multirow{3}{*}{ 4 } & Description &
            \begin{minipage}[t]{13cm}{\footnotesize
            Rate DIASources according to spuriousness metric

            \vspace{\dp0}
            } \end{minipage} \\ \cline{2-3}
            & Test Data &
            \begin{minipage}[t]{13cm}{\footnotesize
                DIASources from step 3

                \vspace{\dp0}
            } \end{minipage} \\ \cline{2-3}
            & Expected Result &
                \begin{minipage}[t]{13cm}{\footnotesize
                Catalog of DIASources with assigned spuriousness values

                \vspace{\dp0}
                } \end{minipage}
        \\ \midrule

            \multirow{3}{*}{ 5 } & Description &
            \begin{minipage}[t]{13cm}{\footnotesize
            Find value of spuriousness threshold that preserves injected sources at
completeness mopsCompletenessMin

            \vspace{\dp0}
            } \end{minipage} \\ \cline{2-3}
            & Test Data &
            \begin{minipage}[t]{13cm}{\footnotesize
                DIASources and spuriousness metric

                \vspace{\dp0}
            } \end{minipage} \\ \cline{2-3}
            & Expected Result &
                \begin{minipage}[t]{13cm}{\footnotesize
                Threshold in spuriousness metric

                \vspace{\dp0}
                } \end{minipage}
        \\ \midrule

            \multirow{3}{*}{ 6 } & Description &
            \begin{minipage}[t]{13cm}{\footnotesize
            Compare to analysis of purity as a function of spuriousness metric

            \vspace{\dp0}
            } \end{minipage} \\ \cline{2-3}
            & Test Data &
            \begin{minipage}[t]{13cm}{\footnotesize
                No data.
                \vspace{\dp0}
            } \end{minipage} \\ \cline{2-3}
            & Expected Result &
        \\ \midrule
    \end{longtable}

\subsection{\href{https://jira.lsstcorp.org/secure/Tests.jspa\#/testCase/LVV-T533}{LVV-T533}
    - MOPS purity threshold}\label{lvv-t533}

\begin{longtable}[]{llllll}
\toprule
Version & Status & Priority & Verification Type & Owner
\\\midrule
1 & Draft & Normal &
Test & Scott Daniel
\\\bottomrule
\end{longtable}

\subsubsection{Verification Elements}
\begin{itemize}
\item \href{https://jira.lsstcorp.org/browse/LVV-1262}{LVV-1262} - OSS-REQ-0354-V-02: Difference Source Spuriousness Threshold - MOPS2

\end{itemize}

\subsubsection{Test Items}


\subsubsection{Predecessors}

\subsubsection{Environment Needs}

\paragraph{Software}

\paragraph{Hardware}

\subsubsection{Input Specification}

\subsubsection{Output Specification}

\subsubsection{Test Procedure}
    \begin{longtable}[]{p{1.3cm}p{2cm}p{13cm}}
    %\toprule
    Step & \multicolumn{2}{@{}l}{Description, Input Data and Expected Result} \\ \toprule
    \endhead

            \multirow{3}{*}{ 1 } & Description &
            \begin{minipage}[t]{13cm}{\footnotesize
            Identify all truly variable sources in a mini-survey area. ~This will
either be done by waiting for the mini-survey to complete and running a
full historical light curve analysis on the region, or through human
inspection of difference images (or some combination of both).

            \vspace{\dp0}
            } \end{minipage} \\ \cline{2-3}
            & Test Data &
            \begin{minipage}[t]{13cm}{\footnotesize
                Completed min-survey images and coadds.

                \vspace{\dp0}
            } \end{minipage} \\ \cline{2-3}
            & Expected Result &
                \begin{minipage}[t]{13cm}{\footnotesize
                Catalog of true variables in the region.

                \vspace{\dp0}
                } \end{minipage}
        \\ \midrule

            \multirow{3}{*}{ 2 } & Description &
            \begin{minipage}[t]{13cm}{\footnotesize
            Go back to the individual images in the mini-survey and perform
difference image analysis.

            \vspace{\dp0}
            } \end{minipage} \\ \cline{2-3}
            & Test Data &
            \begin{minipage}[t]{13cm}{\footnotesize
                Images and multiple epochs in the mini-survey.

                \vspace{\dp0}
            } \end{minipage} \\ \cline{2-3}
            & Expected Result &
                \begin{minipage}[t]{13cm}{\footnotesize
                Catalogs of DIASources, some of them bogus.

                \vspace{\dp0}
                } \end{minipage}
        \\ \midrule

            \multirow{3}{*}{ 3 } & Description &
            \begin{minipage}[t]{13cm}{\footnotesize
            Use catalog from step 1 to identify which of the DIASources in step 2
are real and which are artifacts.

            \vspace{\dp0}
            } \end{minipage} \\ \cline{2-3}
            & Test Data &
            \begin{minipage}[t]{13cm}{\footnotesize
                DIASources from step 2 and catalog of true sources from step 1.

                \vspace{\dp0}
            } \end{minipage} \\ \cline{2-3}
            & Expected Result &
                \begin{minipage}[t]{13cm}{\footnotesize
                Catalog of DIASources labeled as either 'real' or 'bogus'.

                \vspace{\dp0}
                } \end{minipage}
        \\ \midrule

            \multirow{3}{*}{ 4 } & Description &
            \begin{minipage}[t]{13cm}{\footnotesize
            Rate DIASource detections with spuriousness metric.

            \vspace{\dp0}
            } \end{minipage} \\ \cline{2-3}
            & Test Data &
            \begin{minipage}[t]{13cm}{\footnotesize
                DIASources from step 2.

                \vspace{\dp0}
            } \end{minipage} \\ \cline{2-3}
            & Expected Result &
                \begin{minipage}[t]{13cm}{\footnotesize
                Catalog of DIASources with spuriousness metric assigned.

                \vspace{\dp0}
                } \end{minipage}
        \\ \midrule

            \multirow{3}{*}{ 5 } & Description &
            \begin{minipage}[t]{13cm}{\footnotesize
            Find value of spuriousness metric which gives desired purity
mopsPurityMin

            \vspace{\dp0}
            } \end{minipage} \\ \cline{2-3}
            & Test Data &
            \begin{minipage}[t]{13cm}{\footnotesize
                Catalogs from steps 2 and 3.

                \vspace{\dp0}
            } \end{minipage} \\ \cline{2-3}
            & Expected Result &
                \begin{minipage}[t]{13cm}{\footnotesize
                Threshold in spuriousness metric.

                \vspace{\dp0}
                } \end{minipage}
        \\ \midrule

            \multirow{3}{*}{ 6 } & Description &
            \begin{minipage}[t]{13cm}{\footnotesize
            Compare to completeness threshold in spuriousness metric.

            \vspace{\dp0}
            } \end{minipage} \\ \cline{2-3}
            & Test Data &
            \begin{minipage}[t]{13cm}{\footnotesize
                No data.
                \vspace{\dp0}
            } \end{minipage} \\ \cline{2-3}
            & Expected Result &
        \\ \midrule
    \end{longtable}

\subsection{\href{https://jira.lsstcorp.org/secure/Tests.jspa\#/testCase/LVV-T543}{LVV-T543}
    - Astrometric error -- level 1 processing -- simulations}\label{lvv-t543}

\begin{longtable}[]{llllll}
\toprule
Version & Status & Priority & Verification Type & Owner
\\\midrule
1 & Defined & Normal &
Test & Scott Daniel
\\\bottomrule
\end{longtable}

\subsubsection{Verification Elements}
\begin{itemize}
\item \href{https://jira.lsstcorp.org/browse/LVV-1273}{LVV-1273} - OSS-REQ-0149-V-01: Level 1 Catalog Precision

\end{itemize}

\subsubsection{Test Items}


\subsubsection{Predecessors}

\subsubsection{Environment Needs}

\paragraph{Software}

\paragraph{Hardware}

\subsubsection{Input Specification}

\subsubsection{Output Specification}

\subsubsection{Test Procedure}
    \begin{longtable}[]{p{1.3cm}p{2cm}p{13cm}}
    %\toprule
    Step & \multicolumn{2}{@{}l}{Description, Input Data and Expected Result} \\ \toprule
    \endhead

            \multirow{3}{*}{ 1 } & Description &
            \begin{minipage}[t]{13cm}{\footnotesize
            Simulate visits at diverse observing conditions (airmass, seeing,
etc.)\\
~\\
Could also take data from DESC data challenges.

            \vspace{\dp0}
            } \end{minipage} \\ \cline{2-3}
            & Test Data &
            \begin{minipage}[t]{13cm}{\footnotesize
                No data.
                \vspace{\dp0}
            } \end{minipage} \\ \cline{2-3}
            & Expected Result &
                \begin{minipage}[t]{13cm}{\footnotesize
                Set of simulated images.\\
Truth information associated with astrophysical objects in those images.

                \vspace{\dp0}
                } \end{minipage}
        \\ \midrule

            \multirow{3}{*}{ 2 } & Description &
            \begin{minipage}[t]{13cm}{\footnotesize
            Produce template coadd from simulated images.

            \vspace{\dp0}
            } \end{minipage} \\ \cline{2-3}
            & Test Data &
            \begin{minipage}[t]{13cm}{\footnotesize
                Simulated images from step 1

                \vspace{\dp0}
            } \end{minipage} \\ \cline{2-3}
            & Expected Result &
                \begin{minipage}[t]{13cm}{\footnotesize
                {}Template coadd for Level 1 data processing

                \vspace{\dp0}
                } \end{minipage}
        \\ \midrule

            \multirow{3}{*}{ 3 } & Description &
            \begin{minipage}[t]{13cm}{\footnotesize
            Perform Level 1 data processing on simulated images.

            \vspace{\dp0}
            } \end{minipage} \\ \cline{2-3}
            & Test Data &
            \begin{minipage}[t]{13cm}{\footnotesize
                Simulated images from step 1\\
Template coadd from step 2

                \vspace{\dp0}
            } \end{minipage} \\ \cline{2-3}
            & Expected Result &
                \begin{minipage}[t]{13cm}{\footnotesize
                Set of DIASource measurements

                \vspace{\dp0}
                } \end{minipage}
        \\ \midrule

            \multirow{3}{*}{ 4 } & Description &
            \begin{minipage}[t]{13cm}{\footnotesize
            For each simulation, calculate astrometric error expected due to
observing conditions (airmass, seeing, etc.)

            \vspace{\dp0}
            } \end{minipage} \\ \cline{2-3}
            & Test Data &
            \begin{minipage}[t]{13cm}{\footnotesize
                Observing conditions of simulations in step 1

                \vspace{\dp0}
            } \end{minipage} \\ \cline{2-3}
            & Expected Result &
                \begin{minipage}[t]{13cm}{\footnotesize
                Model of expected astrometric errors in DIASource measurements just from
observing conditions.

                \vspace{\dp0}
                } \end{minipage}
        \\ \midrule

            \multirow{3}{*}{ 5 } & Description &
            \begin{minipage}[t]{13cm}{\footnotesize
            Compare DIASource measurements from step 3 with true positions of
astrophysical sources simulated in step 1.\\
~\\
Compare the astrometric errors with the model calculated in step 4.\\
~\\
Verify that the RMS of the residual (measured error minus error expected
just due to observing conditions) is within the specified limit.

            \vspace{\dp0}
            } \end{minipage} \\ \cline{2-3}
            & Test Data &
            \begin{minipage}[t]{13cm}{\footnotesize
                DIASource measurements from step 3.\\
Astrometric error model from step 4

                \vspace{\dp0}
            } \end{minipage} \\ \cline{2-3}
            & Expected Result &
        \\ \midrule
    \end{longtable}

\subsection{\href{https://jira.lsstcorp.org/secure/Tests.jspa\#/testCase/LVV-T544}{LVV-T544}
    - Astrometric error -- level 1 processing -- on-sky data}\label{lvv-t544}

\begin{longtable}[]{llllll}
\toprule
Version & Status & Priority & Verification Type & Owner
\\\midrule
1 & Defined & Normal &
Test & Scott Daniel
\\\bottomrule
\end{longtable}

\subsubsection{Verification Elements}
\begin{itemize}
\item \href{https://jira.lsstcorp.org/browse/LVV-1273}{LVV-1273} - OSS-REQ-0149-V-01: Level 1 Catalog Precision

\end{itemize}

\subsubsection{Test Items}


\subsubsection{Predecessors}

\subsubsection{Environment Needs}

\paragraph{Software}

\paragraph{Hardware}

\subsubsection{Input Specification}

\subsubsection{Output Specification}

\subsubsection{Test Procedure}
    \begin{longtable}[]{p{1.3cm}p{2cm}p{13cm}}
    %\toprule
    Step & \multicolumn{2}{@{}l}{Description, Input Data and Expected Result} \\ \toprule
    \endhead

            \multirow{3}{*}{ 1 } & Description &
            \begin{minipage}[t]{13cm}{\footnotesize
            Perform full-depth mini-survey on a patch of sky.

            \vspace{\dp0}
            } \end{minipage} \\ \cline{2-3}
            & Test Data &
            \begin{minipage}[t]{13cm}{\footnotesize
                No data.
                \vspace{\dp0}
            } \end{minipage} \\ \cline{2-3}
            & Expected Result &
                \begin{minipage}[t]{13cm}{\footnotesize
                Images going down to full LSST depth

                \vspace{\dp0}
                } \end{minipage}
        \\ \midrule

            \multirow{3}{*}{ 2 } & Description &
            \begin{minipage}[t]{13cm}{\footnotesize
            Perform Level 2 processing to get ground truth position of sources.

            \vspace{\dp0}
            } \end{minipage} \\ \cline{2-3}
            & Test Data &
            \begin{minipage}[t]{13cm}{\footnotesize
                Images from step 1

                \vspace{\dp0}
            } \end{minipage} \\ \cline{2-3}
            & Expected Result &
                \begin{minipage}[t]{13cm}{\footnotesize
                Catalog of sources to be used as truth for analysis

                \vspace{\dp0}
                } \end{minipage}
        \\ \midrule

            \multirow{3}{*}{ 3 } & Description &
            \begin{minipage}[t]{13cm}{\footnotesize
            Perform Level 1 analysis on images from step 1.

            \vspace{\dp0}
            } \end{minipage} \\ \cline{2-3}
            & Test Data &
            \begin{minipage}[t]{13cm}{\footnotesize
                Images from step 1\\
A template coadd constructed from those images

                \vspace{\dp0}
            } \end{minipage} \\ \cline{2-3}
            & Expected Result &
                \begin{minipage}[t]{13cm}{\footnotesize
                Catalog of DIASources

                \vspace{\dp0}
                } \end{minipage}
        \\ \midrule

            \multirow{3}{*}{ 4 } & Description &
            \begin{minipage}[t]{13cm}{\footnotesize
            Model astrometric errors as a function of observing conditions (airmass,
seeing, etc.) in images from step 1.

            \vspace{\dp0}
            } \end{minipage} \\ \cline{2-3}
            & Test Data &
            \begin{minipage}[t]{13cm}{\footnotesize
                Metadata from images in step 1

                \vspace{\dp0}
            } \end{minipage} \\ \cline{2-3}
            & Expected Result &
                \begin{minipage}[t]{13cm}{\footnotesize
                Model of astrometric errors due only to observing conditions

                \vspace{\dp0}
                } \end{minipage}
        \\ \midrule

            \multirow{3}{*}{ 5 } & Description &
            \begin{minipage}[t]{13cm}{\footnotesize
            Compare measured positions of DIASources to ground truth catalog from
step 2 to get distribution of astrometric errors.

            \vspace{\dp0}
            } \end{minipage} \\ \cline{2-3}
            & Test Data &
            \begin{minipage}[t]{13cm}{\footnotesize
                DIASources from step 3\\
Ground truth catalog from step 2\\
(relies on a significant portion of the DIASources appearing in the
ground truth catalog)

                \vspace{\dp0}
            } \end{minipage} \\ \cline{2-3}
            & Expected Result &
                \begin{minipage}[t]{13cm}{\footnotesize
                Distribution of measured astrometric errors

                \vspace{\dp0}
                } \end{minipage}
        \\ \midrule

            \multirow{3}{*}{ 6 } & Description &
            \begin{minipage}[t]{13cm}{\footnotesize
            Compute the RMS residual between the measured astrometric errors in step
5 and the model of errors due just to observing conditions in step 4.
~Verify that residual is within specified tolerance.

            \vspace{\dp0}
            } \end{minipage} \\ \cline{2-3}
            & Test Data &
            \begin{minipage}[t]{13cm}{\footnotesize
                Measured astrometric errors from step 5\\
Model of errors due just to observing conditions from step 4

                \vspace{\dp0}
            } \end{minipage} \\ \cline{2-3}
            & Expected Result &
        \\ \midrule
    \end{longtable}

\subsection{\href{https://jira.lsstcorp.org/secure/Tests.jspa\#/testCase/LVV-T545}{LVV-T545}
    - Astrometric error -- level 1 processing -- reference catalog}\label{lvv-t545}

\begin{longtable}[]{llllll}
\toprule
Version & Status & Priority & Verification Type & Owner
\\\midrule
1 & Defined & Normal &
Test & Scott Daniel
\\\bottomrule
\end{longtable}

\subsubsection{Verification Elements}
\begin{itemize}
\item \href{https://jira.lsstcorp.org/browse/LVV-1273}{LVV-1273} - OSS-REQ-0149-V-01: Level 1 Catalog Precision

\end{itemize}

\subsubsection{Test Items}


\subsubsection{Predecessors}

\subsubsection{Environment Needs}

\paragraph{Software}

\paragraph{Hardware}

\subsubsection{Input Specification}

\subsubsection{Output Specification}

\subsubsection{Test Procedure}
    \begin{longtable}[]{p{1.3cm}p{2cm}p{13cm}}
    %\toprule
    Step & \multicolumn{2}{@{}l}{Description, Input Data and Expected Result} \\ \toprule
    \endhead

            \multirow{3}{*}{ 1 } & Description &
            \begin{minipage}[t]{13cm}{\footnotesize
            Find catalog of sources with well-measured astrometry.

            \vspace{\dp0}
            } \end{minipage} \\ \cline{2-3}
            & Test Data &
            \begin{minipage}[t]{13cm}{\footnotesize
                No data.
                \vspace{\dp0}
            } \end{minipage} \\ \cline{2-3}
            & Expected Result &
                \begin{minipage}[t]{13cm}{\footnotesize
                Catalog of sources to be used as ground truth

                \vspace{\dp0}
                } \end{minipage}
        \\ \midrule

            \multirow{3}{*}{ 2 } & Description &
            \begin{minipage}[t]{13cm}{\footnotesize
            Image the area of sky overlapping ground truth catalog from step 1 under
diverse observing conditions (airmass, seeing, etc.).

            \vspace{\dp0}
            } \end{minipage} \\ \cline{2-3}
            & Test Data &
            \begin{minipage}[t]{13cm}{\footnotesize
                No data.
                \vspace{\dp0}
            } \end{minipage} \\ \cline{2-3}
            & Expected Result &
                \begin{minipage}[t]{13cm}{\footnotesize
                Images overlapping catalog from step 1

                \vspace{\dp0}
                } \end{minipage}
        \\ \midrule

            \multirow{3}{*}{ 3 } & Description &
            \begin{minipage}[t]{13cm}{\footnotesize
            Perform Level 1 processing on images taken in step 2.

            \vspace{\dp0}
            } \end{minipage} \\ \cline{2-3}
            & Test Data &
            \begin{minipage}[t]{13cm}{\footnotesize
                Images from step 2\\
Coadd template constructed therefrom

                \vspace{\dp0}
            } \end{minipage} \\ \cline{2-3}
            & Expected Result &
                \begin{minipage}[t]{13cm}{\footnotesize
                Catalog of DIASources

                \vspace{\dp0}
                } \end{minipage}
        \\ \midrule

            \multirow{3}{*}{ 4 } & Description &
            \begin{minipage}[t]{13cm}{\footnotesize
            Construct a model of astrometric errors due only to observing conditions
of images in step 2.

            \vspace{\dp0}
            } \end{minipage} \\ \cline{2-3}
            & Test Data &
            \begin{minipage}[t]{13cm}{\footnotesize
                Metadata from images taken in step 2

                \vspace{\dp0}
            } \end{minipage} \\ \cline{2-3}
            & Expected Result &
                \begin{minipage}[t]{13cm}{\footnotesize
                Model of astrometric errors expected from observing conditions

                \vspace{\dp0}
                } \end{minipage}
        \\ \midrule

            \multirow{3}{*}{ 5 } & Description &
            \begin{minipage}[t]{13cm}{\footnotesize
            Compare DIASources in step 3 to catalog from step 1 to find measured
astrometric errors.

            \vspace{\dp0}
            } \end{minipage} \\ \cline{2-3}
            & Test Data &
            \begin{minipage}[t]{13cm}{\footnotesize
                Truth catalog from step 1\\
DIASource catalog from step 3\\
(relies on a signficant portion of the DIASources appearing in the truth
catalog)

                \vspace{\dp0}
            } \end{minipage} \\ \cline{2-3}
            & Expected Result &
                \begin{minipage}[t]{13cm}{\footnotesize
                Catalog of measured astrometric errors

                \vspace{\dp0}
                } \end{minipage}
        \\ \midrule

            \multirow{3}{*}{ 6 } & Description &
            \begin{minipage}[t]{13cm}{\footnotesize
            Calculate RMS residual between measured astrometric errors (step 5) and
astrometric errors expected due only to observing conditions (step 4).
~Verify that residual is less than specified limit.

            \vspace{\dp0}
            } \end{minipage} \\ \cline{2-3}
            & Test Data &
            \begin{minipage}[t]{13cm}{\footnotesize
                Measured astrometric errors from step 5\\
Model of astrometric errors from step 4

                \vspace{\dp0}
            } \end{minipage} \\ \cline{2-3}
            & Expected Result &
        \\ \midrule
    \end{longtable}

\subsection{\href{https://jira.lsstcorp.org/secure/Tests.jspa\#/testCase/LVV-T546}{LVV-T546}
    - Photometric error -- level 1 processing -- simulations}\label{lvv-t546}

\begin{longtable}[]{llllll}
\toprule
Version & Status & Priority & Verification Type & Owner
\\\midrule
1 & Defined & Normal &
Test & Scott Daniel
\\\bottomrule
\end{longtable}

\subsubsection{Verification Elements}
\begin{itemize}
\item \href{https://jira.lsstcorp.org/browse/LVV-1274}{LVV-1274} - OSS-REQ-0149-V-02: Level 1 Catalog Precision2

\end{itemize}

\subsubsection{Test Items}


\subsubsection{Predecessors}

\subsubsection{Environment Needs}

\paragraph{Software}

\paragraph{Hardware}

\subsubsection{Input Specification}

\subsubsection{Output Specification}

\subsubsection{Test Procedure}
    \begin{longtable}[]{p{1.3cm}p{2cm}p{13cm}}
    %\toprule
    Step & \multicolumn{2}{@{}l}{Description, Input Data and Expected Result} \\ \toprule
    \endhead

            \multirow{3}{*}{ 1 } & Description &
            \begin{minipage}[t]{13cm}{\footnotesize
            Generate simulated images at diverse observing conditions (airmass,
seeing, etc.), or just take images from a pre-existing set of
simulations (e.g. DESC DC2)

            \vspace{\dp0}
            } \end{minipage} \\ \cline{2-3}
            & Test Data &
            \begin{minipage}[t]{13cm}{\footnotesize
                No data.
                \vspace{\dp0}
            } \end{minipage} \\ \cline{2-3}
            & Expected Result &
                \begin{minipage}[t]{13cm}{\footnotesize
                Simulated images and an associated catalog of truth values

                \vspace{\dp0}
                } \end{minipage}
        \\ \midrule

            \multirow{3}{*}{ 2 } & Description &
            \begin{minipage}[t]{13cm}{\footnotesize
            Perform Level 1 processing on the images from step 1.

            \vspace{\dp0}
            } \end{minipage} \\ \cline{2-3}
            & Test Data &
            \begin{minipage}[t]{13cm}{\footnotesize
                Simulated images from step 1

                \vspace{\dp0}
            } \end{minipage} \\ \cline{2-3}
            & Expected Result &
                \begin{minipage}[t]{13cm}{\footnotesize
                Catalog of DIASources

                \vspace{\dp0}
                } \end{minipage}
        \\ \midrule

            \multirow{3}{*}{ 3 } & Description &
            \begin{minipage}[t]{13cm}{\footnotesize
            Construct model of expected photometric errors due to observing
conditions in images simulated in step 1.

            \vspace{\dp0}
            } \end{minipage} \\ \cline{2-3}
            & Test Data &
            \begin{minipage}[t]{13cm}{\footnotesize
                Metadata from images in step 1

                \vspace{\dp0}
            } \end{minipage} \\ \cline{2-3}
            & Expected Result &
                \begin{minipage}[t]{13cm}{\footnotesize
                Model of photometric errors expected due to observing conditions

                \vspace{\dp0}
                } \end{minipage}
        \\ \midrule

            \multirow{3}{*}{ 4 } & Description &
            \begin{minipage}[t]{13cm}{\footnotesize
            Compare DIASources from step 2 to known fluxes in truth catalog from
step 1.

            \vspace{\dp0}
            } \end{minipage} \\ \cline{2-3}
            & Test Data &
            \begin{minipage}[t]{13cm}{\footnotesize
                DIASources from step 2\\
Truth catalog from step 1

                \vspace{\dp0}
            } \end{minipage} \\ \cline{2-3}
            & Expected Result &
                \begin{minipage}[t]{13cm}{\footnotesize
                Catalog of measured photometric errors

                \vspace{\dp0}
                } \end{minipage}
        \\ \midrule

            \multirow{3}{*}{ 5 } & Description &
            \begin{minipage}[t]{13cm}{\footnotesize
            Compute RMS residual between measured photometric errors and model of
errors expected due to just observing conditions. ~Verify that residual
is within specified tolerance.

            \vspace{\dp0}
            } \end{minipage} \\ \cline{2-3}
            & Test Data &
            \begin{minipage}[t]{13cm}{\footnotesize
                Catalog of measured photometric errors from step 4\\
Model of expected photometric errors from step 3

                \vspace{\dp0}
            } \end{minipage} \\ \cline{2-3}
            & Expected Result &
        \\ \midrule
    \end{longtable}

\subsection{\href{https://jira.lsstcorp.org/secure/Tests.jspa\#/testCase/LVV-T547}{LVV-T547}
    - Photometric errors -- level 1 processing -- on-sky data}\label{lvv-t547}

\begin{longtable}[]{llllll}
\toprule
Version & Status & Priority & Verification Type & Owner
\\\midrule
1 & Defined & Normal &
Test & Scott Daniel
\\\bottomrule
\end{longtable}

\subsubsection{Verification Elements}
\begin{itemize}
\item \href{https://jira.lsstcorp.org/browse/LVV-1274}{LVV-1274} - OSS-REQ-0149-V-02: Level 1 Catalog Precision2

\end{itemize}

\subsubsection{Test Items}


\subsubsection{Predecessors}

\subsubsection{Environment Needs}

\paragraph{Software}

\paragraph{Hardware}

\subsubsection{Input Specification}

\subsubsection{Output Specification}

\subsubsection{Test Procedure}
    \begin{longtable}[]{p{1.3cm}p{2cm}p{13cm}}
    %\toprule
    Step & \multicolumn{2}{@{}l}{Description, Input Data and Expected Result} \\ \toprule
    \endhead

            \multirow{3}{*}{ 1 } & Description &
            \begin{minipage}[t]{13cm}{\footnotesize
            Image a patch of sky at various observing conditions (airmass, seeing
,etc.).

            \vspace{\dp0}
            } \end{minipage} \\ \cline{2-3}
            & Test Data &
            \begin{minipage}[t]{13cm}{\footnotesize
                No data.
                \vspace{\dp0}
            } \end{minipage} \\ \cline{2-3}
            & Expected Result &
                \begin{minipage}[t]{13cm}{\footnotesize
                Images of sky at various observing conditions

                \vspace{\dp0}
                } \end{minipage}
        \\ \midrule

            \multirow{3}{*}{ 2 } & Description &
            \begin{minipage}[t]{13cm}{\footnotesize
            Perform Level 2 processing on images from step 1. ~Save catalog of
fluxes as well as standard deviation of flux measurements from
individual images.

            \vspace{\dp0}
            } \end{minipage} \\ \cline{2-3}
            & Test Data &
            \begin{minipage}[t]{13cm}{\footnotesize
                Images from step 1

                \vspace{\dp0}
            } \end{minipage} \\ \cline{2-3}
            & Expected Result &
                \begin{minipage}[t]{13cm}{\footnotesize
                Catalog of all sources in images from step 1\\
Characterization of width of flux measurements for each source

                \vspace{\dp0}
                } \end{minipage}
        \\ \midrule

            \multirow{3}{*}{ 3 } & Description &
            \begin{minipage}[t]{13cm}{\footnotesize
            Perform Level 1 processing on images from step 1. ~Keep difference
images for "forced DIA photometry" in later steps.

            \vspace{\dp0}
            } \end{minipage} \\ \cline{2-3}
            & Test Data &
            \begin{minipage}[t]{13cm}{\footnotesize
                Images from step 1

                \vspace{\dp0}
            } \end{minipage} \\ \cline{2-3}
            & Expected Result &
                \begin{minipage}[t]{13cm}{\footnotesize
                Catalog of variable sources in images from step 1\\
Difference images corresponding to images in step 1

                \vspace{\dp0}
                } \end{minipage}
        \\ \midrule

            \multirow{3}{*}{ 4 } & Description &
            \begin{minipage}[t]{13cm}{\footnotesize
            Identify sources in step 2 that did not appear as DIASources. ~These
will be taken as totally static objects.

            \vspace{\dp0}
            } \end{minipage} \\ \cline{2-3}
            & Test Data &
            \begin{minipage}[t]{13cm}{\footnotesize
                Catalog of sources from step 2\\
Catalog of DIASources from step 3

                \vspace{\dp0}
            } \end{minipage} \\ \cline{2-3}
            & Expected Result &
                \begin{minipage}[t]{13cm}{\footnotesize
                Catalog of static sources

                \vspace{\dp0}
                } \end{minipage}
        \\ \midrule

            \multirow{3}{*}{ 5 } & Description &
            \begin{minipage}[t]{13cm}{\footnotesize
            Perform forced photometry on difference images from step 3 at locations
of static objects identified in step 4.~

            \vspace{\dp0}
            } \end{minipage} \\ \cline{2-3}
            & Test Data &
            \begin{minipage}[t]{13cm}{\footnotesize
                Catalog of static sources from step 4.\\
Difference images from step 3.

                \vspace{\dp0}
            } \end{minipage} \\ \cline{2-3}
            & Expected Result &
                \begin{minipage}[t]{13cm}{\footnotesize
                Catalog of difference image photometry for static sources

                \vspace{\dp0}
                } \end{minipage}
        \\ \midrule

            \multirow{3}{*}{ 6 } & Description &
            \begin{minipage}[t]{13cm}{\footnotesize
            Construct model of photometric uncertainty based only observing
conditions of images in step 1.

            \vspace{\dp0}
            } \end{minipage} \\ \cline{2-3}
            & Test Data &
            \begin{minipage}[t]{13cm}{\footnotesize
                Metadata from images in step 1

                \vspace{\dp0}
            } \end{minipage} \\ \cline{2-3}
            & Expected Result &
                \begin{minipage}[t]{13cm}{\footnotesize
                Model of photometric uncertainty expected solely due to observing
conditions

                \vspace{\dp0}
                } \end{minipage}
        \\ \midrule

            \multirow{3}{*}{ 7 } & Description &
            \begin{minipage}[t]{13cm}{\footnotesize
            Compare distribution of force difference image photometry measurements
from step 5 with intrinsic width of flux measurements in step 2 and
model of uncertainty due to observing conditions in step 6. ~Verify that
RMS residual is within specified tolerance.

            \vspace{\dp0}
            } \end{minipage} \\ \cline{2-3}
            & Test Data &
            \begin{minipage}[t]{13cm}{\footnotesize
                Forced difference image photometry from step 5\\
Uncertainty model from step 6\\
Intrinsic width of flux measurements from step 2

                \vspace{\dp0}
            } \end{minipage} \\ \cline{2-3}
            & Expected Result &
        \\ \midrule
    \end{longtable}

\subsection{\href{https://jira.lsstcorp.org/secure/Tests.jspa\#/testCase/LVV-T548}{LVV-T548}
    - Photometric errors -- level 1 processing -- reference catalog}\label{lvv-t548}

\begin{longtable}[]{llllll}
\toprule
Version & Status & Priority & Verification Type & Owner
\\\midrule
1 & Defined & Normal &
Test & Scott Daniel
\\\bottomrule
\end{longtable}

\subsubsection{Verification Elements}
\begin{itemize}
\item \href{https://jira.lsstcorp.org/browse/LVV-1274}{LVV-1274} - OSS-REQ-0149-V-02: Level 1 Catalog Precision2

\end{itemize}

\subsubsection{Test Items}


\subsubsection{Predecessors}

\subsubsection{Environment Needs}

\paragraph{Software}

\paragraph{Hardware}

\subsubsection{Input Specification}

\subsubsection{Output Specification}

\subsubsection{Test Procedure}
    \begin{longtable}[]{p{1.3cm}p{2cm}p{13cm}}
    %\toprule
    Step & \multicolumn{2}{@{}l}{Description, Input Data and Expected Result} \\ \toprule
    \endhead

            \multirow{3}{*}{ 1 } & Description &
            \begin{minipage}[t]{13cm}{\footnotesize
            Identify catalog of static sources with well measured fluxes. ~We will
also need a sense for the historical RMS variation of the sources' flux.

            \vspace{\dp0}
            } \end{minipage} \\ \cline{2-3}
            & Test Data &
            \begin{minipage}[t]{13cm}{\footnotesize
                No data.
                \vspace{\dp0}
            } \end{minipage} \\ \cline{2-3}
            & Expected Result &
                \begin{minipage}[t]{13cm}{\footnotesize
                Catalog of static sources with flux values and uncertainties

                \vspace{\dp0}
                } \end{minipage}
        \\ \midrule

            \multirow{3}{*}{ 2 } & Description &
            \begin{minipage}[t]{13cm}{\footnotesize
            Image region of sky overlapping the catalog in step 1 at varying
observing conditions (airmass, seeing, etc.)

            \vspace{\dp0}
            } \end{minipage} \\ \cline{2-3}
            & Test Data &
            \begin{minipage}[t]{13cm}{\footnotesize
                No data.
                \vspace{\dp0}
            } \end{minipage} \\ \cline{2-3}
            & Expected Result &
                \begin{minipage}[t]{13cm}{\footnotesize
                Images overlapping catalog from step 1

                \vspace{\dp0}
                } \end{minipage}
        \\ \midrule

            \multirow{3}{*}{ 3 } & Description &
            \begin{minipage}[t]{13cm}{\footnotesize
            Perform level 1 processing on images from step 2. ~Keep difference
images.

            \vspace{\dp0}
            } \end{minipage} \\ \cline{2-3}
            & Test Data &
            \begin{minipage}[t]{13cm}{\footnotesize
                Images from step 2

                \vspace{\dp0}
            } \end{minipage} \\ \cline{2-3}
            & Expected Result &
                \begin{minipage}[t]{13cm}{\footnotesize
                Catalog of DIASources\\
Difference images

                \vspace{\dp0}
                } \end{minipage}
        \\ \midrule

            \multirow{3}{*}{ 4 } & Description &
            \begin{minipage}[t]{13cm}{\footnotesize
            Perform forced photometry on difference images from step 3 at location
of sources identified in the catalog from step 1.

            \vspace{\dp0}
            } \end{minipage} \\ \cline{2-3}
            & Test Data &
            \begin{minipage}[t]{13cm}{\footnotesize
                Difference images from step 3.

                \vspace{\dp0}
            } \end{minipage} \\ \cline{2-3}
            & Expected Result &
                \begin{minipage}[t]{13cm}{\footnotesize
                Catalog of forced difference image photometry measurements.

                \vspace{\dp0}
                } \end{minipage}
        \\ \midrule

            \multirow{3}{*}{ 5 } & Description &
            \begin{minipage}[t]{13cm}{\footnotesize
            Construct model of photometric uncertainty based only on observing
conditions of images in step 1.

            \vspace{\dp0}
            } \end{minipage} \\ \cline{2-3}
            & Test Data &
            \begin{minipage}[t]{13cm}{\footnotesize
                Metadata from images in step 1

                \vspace{\dp0}
            } \end{minipage} \\ \cline{2-3}
            & Expected Result &
                \begin{minipage}[t]{13cm}{\footnotesize
                Model of photometric uncertainty expected from observing conditions

                \vspace{\dp0}
                } \end{minipage}
        \\ \midrule

            \multirow{3}{*}{ 6 } & Description &
            \begin{minipage}[t]{13cm}{\footnotesize
            Compare force difference image photometry to intrinsic width of source
photometry measurements in step 1 and model of uncertainty from step 5.
~Compare RMS residual to specified tolerance.

            \vspace{\dp0}
            } \end{minipage} \\ \cline{2-3}
            & Test Data &
            \begin{minipage}[t]{13cm}{\footnotesize
                Photometric uncertainty model from step 5\\
Intrinsic widths of photometric ~measurements from step 1\\
Forced difference image photometry from step 4

                \vspace{\dp0}
            } \end{minipage} \\ \cline{2-3}
            & Expected Result &
        \\ \midrule
    \end{longtable}

\subsection{\href{https://jira.lsstcorp.org/secure/Tests.jspa\#/testCase/LVV-T549}{LVV-T549}
    - Zeropoint consistency}\label{lvv-t549}

\begin{longtable}[]{llllll}
\toprule
Version & Status & Priority & Verification Type & Owner
\\\midrule
1 & Defined & Normal &
Test & Scott Daniel
\\\bottomrule
\end{longtable}

\subsubsection{Verification Elements}
\begin{itemize}
\item \href{https://jira.lsstcorp.org/browse/LVV-1291}{LVV-1291} - OSS-REQ-0152-V-01: Level 1 Photometric Zero Point Error

\end{itemize}

\subsubsection{Test Items}


\subsubsection{Predecessors}

\subsubsection{Environment Needs}

\paragraph{Software}

\paragraph{Hardware}

\subsubsection{Input Specification}

\subsubsection{Output Specification}

\subsubsection{Test Procedure}
    \begin{longtable}[]{p{1.3cm}p{2cm}p{13cm}}
    %\toprule
    Step & \multicolumn{2}{@{}l}{Description, Input Data and Expected Result} \\ \toprule
    \endhead

            \multirow{3}{*}{ 1 } & Description &
            \begin{minipage}[t]{13cm}{\footnotesize
            Image a patch of sky to a specified depth (1yr? 3yr? full LSST?).

            \vspace{\dp0}
            } \end{minipage} \\ \cline{2-3}
            & Test Data &
            \begin{minipage}[t]{13cm}{\footnotesize
                No data.
                \vspace{\dp0}
            } \end{minipage} \\ \cline{2-3}
            & Expected Result &
                \begin{minipage}[t]{13cm}{\footnotesize
                Images of the same patch of sky

                \vspace{\dp0}
                } \end{minipage}
        \\ \midrule

            \multirow{3}{*}{ 2 } & Description &
            \begin{minipage}[t]{13cm}{\footnotesize
            Perform Level 2 processing on images from step 1. ~Store final zeropoint
determination.

            \vspace{\dp0}
            } \end{minipage} \\ \cline{2-3}
            & Test Data &
            \begin{minipage}[t]{13cm}{\footnotesize
                Images from step 1

                \vspace{\dp0}
            } \end{minipage} \\ \cline{2-3}
            & Expected Result &
                \begin{minipage}[t]{13cm}{\footnotesize
                Determination of photometric zeropoint from final photometric
calibration algorithm

                \vspace{\dp0}
                } \end{minipage}
        \\ \midrule

            \multirow{3}{*}{ 3 } & Description &
            \begin{minipage}[t]{13cm}{\footnotesize
            Calibrate images from step 1 using Level 1 pipeline

            \vspace{\dp0}
            } \end{minipage} \\ \cline{2-3}
            & Test Data &
            \begin{minipage}[t]{13cm}{\footnotesize
                Images from step 1

                \vspace{\dp0}
            } \end{minipage} \\ \cline{2-3}
            & Expected Result &
                \begin{minipage}[t]{13cm}{\footnotesize
                Calibrated images based on raw images from step 1

                \vspace{\dp0}
                } \end{minipage}
        \\ \midrule

            \multirow{3}{*}{ 4 } & Description &
            \begin{minipage}[t]{13cm}{\footnotesize
            For each calibrated image produced in step 3, verify that the
photometric zeropoint agrees with the final zeropoint determine din step
2 to the specified tolerance.

            \vspace{\dp0}
            } \end{minipage} \\ \cline{2-3}
            & Test Data &
            \begin{minipage}[t]{13cm}{\footnotesize
                Calibrated images from step 3

                \vspace{\dp0}
            } \end{minipage} \\ \cline{2-3}
            & Expected Result &
        \\ \midrule
    \end{longtable}

\subsection{\href{https://jira.lsstcorp.org/secure/Tests.jspa\#/testCase/LVV-T550}{LVV-T550}
    - MOPS -- orbit association completeness}\label{lvv-t550}

\begin{longtable}[]{llllll}
\toprule
Version & Status & Priority & Verification Type & Owner
\\\midrule
1 & Defined & Normal &
Test & Scott Daniel
\\\bottomrule
\end{longtable}

\subsubsection{Verification Elements}
\begin{itemize}
\item \href{https://jira.lsstcorp.org/browse/LVV-3766}{LVV-3766} - OSS-REQ-0159-V-01: Level 1 Moving Object Quality

\end{itemize}

\subsubsection{Test Items}


\subsubsection{Predecessors}

\subsubsection{Environment Needs}

\paragraph{Software}

\paragraph{Hardware}

\subsubsection{Input Specification}

\subsubsection{Output Specification}

\subsubsection{Test Procedure}
    \begin{longtable}[]{p{1.3cm}p{2cm}p{13cm}}
    %\toprule
    Step & \multicolumn{2}{@{}l}{Description, Input Data and Expected Result} \\ \toprule
    \endhead

            \multirow{3}{*}{ 1 } & Description &
            \begin{minipage}[t]{13cm}{\footnotesize
            Take a series of image near the ecliptic.

            \vspace{\dp0}
            } \end{minipage} \\ \cline{2-3}
            & Test Data &
            \begin{minipage}[t]{13cm}{\footnotesize
                No data.
                \vspace{\dp0}
            } \end{minipage} \\ \cline{2-3}
            & Expected Result &
                \begin{minipage}[t]{13cm}{\footnotesize
                Set of images into which we can inject simulated solar system objects

                \vspace{\dp0}
                } \end{minipage}
        \\ \midrule

            \multirow{3}{*}{ 2 } & Description &
            \begin{minipage}[t]{13cm}{\footnotesize
            Generate catalog of simulated solar system objects at times and
locations overlapping the observations in step 1.

            \vspace{\dp0}
            } \end{minipage} \\ \cline{2-3}
            & Test Data &
            \begin{minipage}[t]{13cm}{\footnotesize
                Metadata from images in step 1\\
~\\

                \vspace{\dp0}
            } \end{minipage} \\ \cline{2-3}
            & Expected Result &
                \begin{minipage}[t]{13cm}{\footnotesize
                Catalog of simulated solar system objects

                \vspace{\dp0}
                } \end{minipage}
        \\ \midrule

            \multirow{3}{*}{ 3 } & Description &
            \begin{minipage}[t]{13cm}{\footnotesize
            Inject simulated solar system objects from step 2 into images from step
1.

            \vspace{\dp0}
            } \end{minipage} \\ \cline{2-3}
            & Test Data &
            \begin{minipage}[t]{13cm}{\footnotesize
                Images from step 1\\
Simulated objects from step 2

                \vspace{\dp0}
            } \end{minipage} \\ \cline{2-3}
            & Expected Result &
                \begin{minipage}[t]{13cm}{\footnotesize
                Images containing simulated solar system objects

                \vspace{\dp0}
                } \end{minipage}
        \\ \midrule

            \multirow{3}{*}{ 4 } & Description &
            \begin{minipage}[t]{13cm}{\footnotesize
            Perform Level 1 processing on the images with the injected simulations.
~Keep track of which simulated objects were observed the specified
number of times within the specified interval.

            \vspace{\dp0}
            } \end{minipage} \\ \cline{2-3}
            & Test Data &
            \begin{minipage}[t]{13cm}{\footnotesize
                Images with injected objects from step 3

                \vspace{\dp0}
            } \end{minipage} \\ \cline{2-3}
            & Expected Result &
                \begin{minipage}[t]{13cm}{\footnotesize
                Catalog of detected DIAsources

                \vspace{\dp0}
                } \end{minipage}
        \\ \midrule

            \multirow{3}{*}{ 5 } & Description &
            \begin{minipage}[t]{13cm}{\footnotesize
            Perform MOPS orbit-linking on DIASources from step 4.

            \vspace{\dp0}
            } \end{minipage} \\ \cline{2-3}
            & Test Data &
            \begin{minipage}[t]{13cm}{\footnotesize
                Catalog of DIASources from step 4

                \vspace{\dp0}
            } \end{minipage} \\ \cline{2-3}
            & Expected Result &
                \begin{minipage}[t]{13cm}{\footnotesize
                Set of candidate orbits

                \vspace{\dp0}
                } \end{minipage}
        \\ \midrule

            \multirow{3}{*}{ 6 } & Description &
            \begin{minipage}[t]{13cm}{\footnotesize
            Verify that the specified fraction of simulated objects which were
observed at the specified cadence were correctly linked into orbits.

            \vspace{\dp0}
            } \end{minipage} \\ \cline{2-3}
            & Test Data &
            \begin{minipage}[t]{13cm}{\footnotesize
                MOPS-generated orbits from step 5\\
Catalog of simulated objects that were observed at the desired cadence
from step 4

                \vspace{\dp0}
            } \end{minipage} \\ \cline{2-3}
            & Expected Result &
        \\ \midrule
    \end{longtable}

\subsection{\href{https://jira.lsstcorp.org/secure/Tests.jspa\#/testCase/LVV-T551}{LVV-T551}
    - WCS accuracy -- simulations}\label{lvv-t551}

\begin{longtable}[]{llllll}
\toprule
Version & Status & Priority & Verification Type & Owner
\\\midrule
1 & Defined & Normal &
Test & Scott Daniel
\\\bottomrule
\end{longtable}

\subsubsection{Verification Elements}
\begin{itemize}
\item \href{https://jira.lsstcorp.org/browse/LVV-1339}{LVV-1339} - OSS-REQ-0153-V-01: World Coordinate System Accuracy

\end{itemize}

\subsubsection{Test Items}


\subsubsection{Predecessors}

\subsubsection{Environment Needs}

\paragraph{Software}

\paragraph{Hardware}

\subsubsection{Input Specification}

\subsubsection{Output Specification}

\subsubsection{Test Procedure}
    \begin{longtable}[]{p{1.3cm}p{2cm}p{13cm}}
    %\toprule
    Step & \multicolumn{2}{@{}l}{Description, Input Data and Expected Result} \\ \toprule
    \endhead

            \multirow{3}{*}{ 1 } & Description &
            \begin{minipage}[t]{13cm}{\footnotesize
            Select image simulations for one patch from existing Data Challenge
(i.e. DESC DC2). ~Images must have centroid files describing the
centroid pixels for each object.

            \vspace{\dp0}
            } \end{minipage} \\ \cline{2-3}
            & Test Data &
            \begin{minipage}[t]{13cm}{\footnotesize
                No data.
                \vspace{\dp0}
            } \end{minipage} \\ \cline{2-3}
            & Expected Result &
                \begin{minipage}[t]{13cm}{\footnotesize
                Simulated images and centroid files

                \vspace{\dp0}
                } \end{minipage}
        \\ \midrule

            \multirow{3}{*}{ 2 } & Description &
            \begin{minipage}[t]{13cm}{\footnotesize
            Create copies of the images. ~Use centroid files to modify the WCS of
each individual to be as accurate as possible. ~These will be treated as
images with ground truth WCSes.

            \vspace{\dp0}
            } \end{minipage} \\ \cline{2-3}
            & Test Data &
            \begin{minipage}[t]{13cm}{\footnotesize
                Images from step 1

                \vspace{\dp0}
            } \end{minipage} \\ \cline{2-3}
            & Expected Result &
                \begin{minipage}[t]{13cm}{\footnotesize
                Simulated images with exact WCSes

                \vspace{\dp0}
                } \end{minipage}
        \\ \midrule

            \multirow{3}{*}{ 3 } & Description &
            \begin{minipage}[t]{13cm}{\footnotesize
            Create coadds at different time steps (1 year, 2 year, 3 year, etc.)
from the images in step 1.

            \vspace{\dp0}
            } \end{minipage} \\ \cline{2-3}
            & Test Data &
            \begin{minipage}[t]{13cm}{\footnotesize
                Images from step 1

                \vspace{\dp0}
            } \end{minipage} \\ \cline{2-3}
            & Expected Result &
                \begin{minipage}[t]{13cm}{\footnotesize
                Catalog of coadd sources

                \vspace{\dp0}
                } \end{minipage}
        \\ \midrule

            \multirow{3}{*}{ 4 } & Description &
            \begin{minipage}[t]{13cm}{\footnotesize
            Create coadds at the same time steps as in step 4 from the images with
ground truth WCSes in step 2.

            \vspace{\dp0}
            } \end{minipage} \\ \cline{2-3}
            & Test Data &
            \begin{minipage}[t]{13cm}{\footnotesize
                Images from step 2

                \vspace{\dp0}
            } \end{minipage} \\ \cline{2-3}
            & Expected Result &
                \begin{minipage}[t]{13cm}{\footnotesize
                Catalog of coadd sources from images with exact WCSes

                \vspace{\dp0}
                } \end{minipage}
        \\ \midrule

            \multirow{3}{*}{ 5 } & Description &
            \begin{minipage}[t]{13cm}{\footnotesize
            Compare PSF FWHM measurements between the catalogs in step 4 and the
catalogs in step 3. ~Any difference should be due to inaccuracies in the
WCSes used for the coadds in step 3. ~Verify that this difference is
within specified tolerance.

            \vspace{\dp0}
            } \end{minipage} \\ \cline{2-3}
            & Test Data &
            \begin{minipage}[t]{13cm}{\footnotesize
                Coadd catalogs from steps 3 and 4

                \vspace{\dp0}
            } \end{minipage} \\ \cline{2-3}
            & Expected Result &
        \\ \midrule
    \end{longtable}

\subsection{\href{https://jira.lsstcorp.org/secure/Tests.jspa\#/testCase/LVV-T554}{LVV-T554}
    - Single exposure dynamic range}\label{lvv-t554}

\begin{longtable}[]{llllll}
\toprule
Version & Status & Priority & Verification Type & Owner
\\\midrule
1 & Defined & Normal &
Test & Scott Daniel
\\\bottomrule
\end{longtable}

\subsubsection{Verification Elements}
\begin{itemize}
\item \href{https://jira.lsstcorp.org/browse/LVV-1645}{LVV-1645} - OSS-REQ-0268-V-01: Dynamic Range

\end{itemize}

\subsubsection{Test Items}


\subsubsection{Predecessors}

\subsubsection{Environment Needs}

\paragraph{Software}

\paragraph{Hardware}

\subsubsection{Input Specification}

\subsubsection{Output Specification}

\subsubsection{Test Procedure}
    \begin{longtable}[]{p{1.3cm}p{2cm}p{13cm}}
    %\toprule
    Step & \multicolumn{2}{@{}l}{Description, Input Data and Expected Result} \\ \toprule
    \endhead

            \multirow{3}{*}{ 1 } & Description &
            \begin{minipage}[t]{13cm}{\footnotesize
            Find images observed at:\\
airmass = 1.0\\
r-band skybrightness = 21 magnitude/arcsec\^{}2\\
r-band seeing = 0.7 arcsec

            \vspace{\dp0}
            } \end{minipage} \\ \cline{2-3}
            & Test Data &
            \begin{minipage}[t]{13cm}{\footnotesize
                Exposures from a mini survey

                \vspace{\dp0}
            } \end{minipage} \\ \cline{2-3}
            & Expected Result &
                \begin{minipage}[t]{13cm}{\footnotesize
                Set of images to test

                \vspace{\dp0}
                } \end{minipage}
        \\ \midrule

            \multirow{3}{*}{ 2 } & Description &
            \begin{minipage}[t]{13cm}{\footnotesize
            Run single-exposure processing on the images from step 1.

            \vspace{\dp0}
            } \end{minipage} \\ \cline{2-3}
            & Test Data &
            \begin{minipage}[t]{13cm}{\footnotesize
                Set of images identified at fiducial conditions.

                \vspace{\dp0}
            } \end{minipage} \\ \cline{2-3}
            & Expected Result &
                \begin{minipage}[t]{13cm}{\footnotesize
                5-sigma limiting magnitude and list of detected sources for each image.

                \vspace{\dp0}
                } \end{minipage}
        \\ \midrule

            \multirow{3}{*}{ 3 } & Description &
            \begin{minipage}[t]{13cm}{\footnotesize
            Check that sources within specified dynamic range are not saturated.

            \vspace{\dp0}
            } \end{minipage} \\ \cline{2-3}
            & Test Data &
            \begin{minipage}[t]{13cm}{\footnotesize
                List of detected sources from step 2

                \vspace{\dp0}
            } \end{minipage} \\ \cline{2-3}
            & Expected Result &
        \\ \midrule
    \end{longtable}

\subsection{\href{https://jira.lsstcorp.org/secure/Tests.jspa\#/testCase/LVV-T587}{LVV-T587}
    - PSF size in pixels}\label{lvv-t587}

\begin{longtable}[]{llllll}
\toprule
Version & Status & Priority & Verification Type & Owner
\\\midrule
1 & Defined & Normal &
Test & Scott Daniel
\\\bottomrule
\end{longtable}

\subsubsection{Verification Elements}
\begin{itemize}
\item \href{https://jira.lsstcorp.org/browse/LVV-253}{LVV-253} - LSR-REQ-0007-V-01: Delivered Image Quality1

\end{itemize}

\subsubsection{Test Items}


\subsubsection{Predecessors}

\subsubsection{Environment Needs}

\paragraph{Software}

\paragraph{Hardware}

\subsubsection{Input Specification}

\subsubsection{Output Specification}

\subsubsection{Test Procedure}
    \begin{longtable}[]{p{1.3cm}p{2cm}p{13cm}}
    %\toprule
    Step & \multicolumn{2}{@{}l}{Description, Input Data and Expected Result} \\ \toprule
    \endhead

            \multirow{3}{*}{ 1 } & Description &
            \begin{minipage}[t]{13cm}{\footnotesize
            Use DIMM data to select a set of pointings observed at 0.6 arcsecond
seeing.

            \vspace{\dp0}
            } \end{minipage} \\ \cline{2-3}
            & Test Data &
            \begin{minipage}[t]{13cm}{\footnotesize
                No data.
                \vspace{\dp0}
            } \end{minipage} \\ \cline{2-3}
            & Expected Result &
        \\ \midrule

            \multirow{3}{*}{ 2 } & Description &
            \begin{minipage}[t]{13cm}{\footnotesize
            Run single image processing on pointings from step 1

            \vspace{\dp0}
            } \end{minipage} \\ \cline{2-3}
            & Test Data &
            \begin{minipage}[t]{13cm}{\footnotesize
                pointings from step 1

                \vspace{\dp0}
            } \end{minipage} \\ \cline{2-3}
            & Expected Result &
                \begin{minipage}[t]{13cm}{\footnotesize
                Catalog of detected sources

                \vspace{\dp0}
                } \end{minipage}
        \\ \midrule

            \multirow{3}{*}{ 3 } & Description &
            \begin{minipage}[t]{13cm}{\footnotesize
            Verify that the minimum FWHM of the PSFs of unresolved point sources is
3 pixels or greater

            \vspace{\dp0}
            } \end{minipage} \\ \cline{2-3}
            & Test Data &
            \begin{minipage}[t]{13cm}{\footnotesize
                source catalog from step 2

                \vspace{\dp0}
            } \end{minipage} \\ \cline{2-3}
            & Expected Result &
        \\ \midrule
    \end{longtable}

\subsection{\href{https://jira.lsstcorp.org/secure/Tests.jspa\#/testCase/LVV-T588}{LVV-T588}
    - Median image quality at 0.44 arcsecond seeing}\label{lvv-t588}

\begin{longtable}[]{llllll}
\toprule
Version & Status & Priority & Verification Type & Owner
\\\midrule
1 & Defined & Normal &
Test & Scott Daniel
\\\bottomrule
\end{longtable}

\subsubsection{Verification Elements}
\begin{itemize}
\item \href{https://jira.lsstcorp.org/browse/LVV-254}{LVV-254} - LSR-REQ-0007-V-02: Delivered Image Quality2

\end{itemize}

\subsubsection{Test Items}




\subsubsection{Predecessors}

\subsubsection{Environment Needs}

\paragraph{Software}

\paragraph{Hardware}

\subsubsection{Input Specification}

\subsubsection{Output Specification}

\subsubsection{Test Procedure}
    \begin{longtable}[]{p{1.3cm}p{2cm}p{13cm}}
    %\toprule
    Step & \multicolumn{2}{@{}l}{Description, Input Data and Expected Result} \\ \toprule
    \endhead

            \multirow{3}{*}{ 1 } & Description &
            \begin{minipage}[t]{13cm}{\footnotesize
            Use DIMM to select pointings taken at 0.44 arcsecond seeing in the r and
i bands

            \vspace{\dp0}
            } \end{minipage} \\ \cline{2-3}
            & Test Data &
            \begin{minipage}[t]{13cm}{\footnotesize
                No data.
                \vspace{\dp0}
            } \end{minipage} \\ \cline{2-3}
            & Expected Result &
        \\ \midrule

            \multirow{3}{*}{ 2 } & Description &
            \begin{minipage}[t]{13cm}{\footnotesize
            Run single image processing on exposures selected in step 1

            \vspace{\dp0}
            } \end{minipage} \\ \cline{2-3}
            & Test Data &
            \begin{minipage}[t]{13cm}{\footnotesize
                Exposures from step 1

                \vspace{\dp0}
            } \end{minipage} \\ \cline{2-3}
            & Expected Result &
                \begin{minipage}[t]{13cm}{\footnotesize
                Source catalog

                \vspace{\dp0}
                } \end{minipage}
        \\ \midrule

            \multirow{3}{*}{ 3 } & Description &
            \begin{minipage}[t]{13cm}{\footnotesize
            Verify median of distribution of residual PSF FWHM is approximately 0.59
arcseconds

            \vspace{\dp0}
            } \end{minipage} \\ \cline{2-3}
            & Test Data &
            \begin{minipage}[t]{13cm}{\footnotesize
                Source catalog from step 2

                \vspace{\dp0}
            } \end{minipage} \\ \cline{2-3}
            & Expected Result &
        \\ \midrule
    \end{longtable}

\subsection{\href{https://jira.lsstcorp.org/secure/Tests.jspa\#/testCase/LVV-T589}{LVV-T589}
    - Median image quality at 0.6 arcsecond seeing}\label{lvv-t589}

\begin{longtable}[]{llllll}
\toprule
Version & Status & Priority & Verification Type & Owner
\\\midrule
1 & Defined & Normal &
Test & Scott Daniel
\\\bottomrule
\end{longtable}

\subsubsection{Verification Elements}
\begin{itemize}
\item \href{https://jira.lsstcorp.org/browse/LVV-255}{LVV-255} - LSR-REQ-0007-V-03: Delivered Image Quality3

\end{itemize}

\subsubsection{Test Items}


\subsubsection{Predecessors}

\subsubsection{Environment Needs}

\paragraph{Software}

\paragraph{Hardware}

\subsubsection{Input Specification}

\subsubsection{Output Specification}

\subsubsection{Test Procedure}
    \begin{longtable}[]{p{1.3cm}p{2cm}p{13cm}}
    %\toprule
    Step & \multicolumn{2}{@{}l}{Description, Input Data and Expected Result} \\ \toprule
    \endhead

            \multirow{3}{*}{ 1 } & Description &
            \begin{minipage}[t]{13cm}{\footnotesize
            Use DIMM to select pointings taken at 0.6 arcsecond seeing in the r and
i bands

            \vspace{\dp0}
            } \end{minipage} \\ \cline{2-3}
            & Test Data &
            \begin{minipage}[t]{13cm}{\footnotesize
                No data.
                \vspace{\dp0}
            } \end{minipage} \\ \cline{2-3}
            & Expected Result &
        \\ \midrule

            \multirow{3}{*}{ 2 } & Description &
            \begin{minipage}[t]{13cm}{\footnotesize
            Run single image processing on the exposures from step 1.

            \vspace{\dp0}
            } \end{minipage} \\ \cline{2-3}
            & Test Data &
            \begin{minipage}[t]{13cm}{\footnotesize
                Exposures from step 1

                \vspace{\dp0}
            } \end{minipage} \\ \cline{2-3}
            & Expected Result &
                \begin{minipage}[t]{13cm}{\footnotesize
                Source catalog

                \vspace{\dp0}
                } \end{minipage}
        \\ \midrule

            \multirow{3}{*}{ 3 } & Description &
            \begin{minipage}[t]{13cm}{\footnotesize
            Verify that the median PSF FWHM of unresolved point sources is 0.72
arcseconds.

            \vspace{\dp0}
            } \end{minipage} \\ \cline{2-3}
            & Test Data &
            \begin{minipage}[t]{13cm}{\footnotesize
                Source catalog from step 2

                \vspace{\dp0}
            } \end{minipage} \\ \cline{2-3}
            & Expected Result &
        \\ \midrule
    \end{longtable}

\subsection{\href{https://jira.lsstcorp.org/secure/Tests.jspa\#/testCase/LVV-T590}{LVV-T590}
    - Median image quality at 0.8 arcsecond seeing}\label{lvv-t590}

\begin{longtable}[]{llllll}
\toprule
Version & Status & Priority & Verification Type & Owner
\\\midrule
1 & Defined & Normal &
Test & Scott Daniel
\\\bottomrule
\end{longtable}

\subsubsection{Verification Elements}
\begin{itemize}
\item \href{https://jira.lsstcorp.org/browse/LVV-256}{LVV-256} - LSR-REQ-0007-V-04: Delivered Image Quality4

\end{itemize}

\subsubsection{Test Items}


\subsubsection{Predecessors}

\subsubsection{Environment Needs}

\paragraph{Software}

\paragraph{Hardware}

\subsubsection{Input Specification}

\subsubsection{Output Specification}

\subsubsection{Test Procedure}
    \begin{longtable}[]{p{1.3cm}p{2cm}p{13cm}}
    %\toprule
    Step & \multicolumn{2}{@{}l}{Description, Input Data and Expected Result} \\ \toprule
    \endhead

            \multirow{3}{*}{ 1 } & Description &
            \begin{minipage}[t]{13cm}{\footnotesize
            Use DIMM to select exposures taken at 0.8 arcsecond seeing in the r and
i bands

            \vspace{\dp0}
            } \end{minipage} \\ \cline{2-3}
            & Test Data &
            \begin{minipage}[t]{13cm}{\footnotesize
                No data.
                \vspace{\dp0}
            } \end{minipage} \\ \cline{2-3}
            & Expected Result &
        \\ \midrule

            \multirow{3}{*}{ 2 } & Description &
            \begin{minipage}[t]{13cm}{\footnotesize
            Run single image processing on the exposures from step 1

            \vspace{\dp0}
            } \end{minipage} \\ \cline{2-3}
            & Test Data &
            \begin{minipage}[t]{13cm}{\footnotesize
                Exposures from step 1

                \vspace{\dp0}
            } \end{minipage} \\ \cline{2-3}
            & Expected Result &
                \begin{minipage}[t]{13cm}{\footnotesize
                Source catalog

                \vspace{\dp0}
                } \end{minipage}
        \\ \midrule

            \multirow{3}{*}{ 3 } & Description &
            \begin{minipage}[t]{13cm}{\footnotesize
            Verify that median of PSF FWHM is approximately 0.89 arcsecond

            \vspace{\dp0}
            } \end{minipage} \\ \cline{2-3}
            & Test Data &
            \begin{minipage}[t]{13cm}{\footnotesize
                No data.
                \vspace{\dp0}
            } \end{minipage} \\ \cline{2-3}
            & Expected Result &
        \\ \midrule
    \end{longtable}

\subsection{\href{https://jira.lsstcorp.org/secure/Tests.jspa\#/testCase/LVV-T591}{LVV-T591}
    - Flux-enclosing radius}\label{lvv-t591}

\begin{longtable}[]{llllll}
\toprule
Version & Status & Priority & Verification Type & Owner
\\\midrule
1 & Defined & Normal &
Test & Scott Daniel
\\\bottomrule
\end{longtable}

\subsubsection{Verification Elements}
\begin{itemize}
\item \href{https://jira.lsstcorp.org/browse/LVV-257}{LVV-257} - LSR-REQ-0007-V-05: Delivered Image Quality5

\end{itemize}

\subsubsection{Test Items}


\subsubsection{Predecessors}

\subsubsection{Environment Needs}

\paragraph{Software}

\paragraph{Hardware}

\subsubsection{Input Specification}

\subsubsection{Output Specification}

\subsubsection{Test Procedure}
    \begin{longtable}[]{p{1.3cm}p{2cm}p{13cm}}
    %\toprule
    Step & \multicolumn{2}{@{}l}{Description, Input Data and Expected Result} \\ \toprule
    \endhead

            \multirow{3}{*}{ 1 } & Description &
            \begin{minipage}[t]{13cm}{\footnotesize
            Select pointings in sparse fields (so that we don't have to worry about
more than one object getting encircled by the test apertures below)

            \vspace{\dp0}
            } \end{minipage} \\ \cline{2-3}
            & Test Data &
            \begin{minipage}[t]{13cm}{\footnotesize
                No data.
                \vspace{\dp0}
            } \end{minipage} \\ \cline{2-3}
            & Expected Result &
        \\ \midrule

            \multirow{3}{*}{ 2 } & Description &
            \begin{minipage}[t]{13cm}{\footnotesize
            Run single image processing on the exposures from step 1 to identify all
of the sources.

            \vspace{\dp0}
            } \end{minipage} \\ \cline{2-3}
            & Test Data &
            \begin{minipage}[t]{13cm}{\footnotesize
                Exposures from step 1

                \vspace{\dp0}
            } \end{minipage} \\ \cline{2-3}
            & Expected Result &
                \begin{minipage}[t]{13cm}{\footnotesize
                Source catalog

                \vspace{\dp0}
                } \end{minipage}
        \\ \midrule

            \multirow{3}{*}{ 3 } & Description &
            \begin{minipage}[t]{13cm}{\footnotesize
            Run forced photometry on all detected unresolved point sources,
measuring fluxes inside of a 2 arcescond (or some other unseemly large)
aperture.

            \vspace{\dp0}
            } \end{minipage} \\ \cline{2-3}
            & Test Data &
            \begin{minipage}[t]{13cm}{\footnotesize
                Source catalog from step 2

                \vspace{\dp0}
            } \end{minipage} \\ \cline{2-3}
            & Expected Result &
                \begin{minipage}[t]{13cm}{\footnotesize
                Catalog of source fluxes in a large aperture

                \vspace{\dp0}
                } \end{minipage}
        \\ \midrule

            \multirow{3}{*}{ 4 } & Description &
            \begin{minipage}[t]{13cm}{\footnotesize
            Re-run forced photometry on sources from step 3, reducing the aperture
to 1.81 arcsecond, 1.31 arcsecond, and 0.8 arcsecond. ~Verify that three
measurements contain at least 95\%, 90\%, and 80\% of the flux for all
sources.

            \vspace{\dp0}
            } \end{minipage} \\ \cline{2-3}
            & Test Data &
            \begin{minipage}[t]{13cm}{\footnotesize
                Flux catalogs from step 3\\
Sources from step 2

                \vspace{\dp0}
            } \end{minipage} \\ \cline{2-3}
            & Expected Result &
        \\ \midrule
    \end{longtable}

\subsection{\href{https://jira.lsstcorp.org/secure/Tests.jspa\#/testCase/LVV-T592}{LVV-T592}
    - Image quality - maximum system contribution}\label{lvv-t592}

\begin{longtable}[]{llllll}
\toprule
Version & Status & Priority & Verification Type & Owner
\\\midrule
1 & Defined & Normal &
Test & Scott Daniel
\\\bottomrule
\end{longtable}

\subsubsection{Verification Elements}
\begin{itemize}
\item \href{https://jira.lsstcorp.org/browse/LVV-9804}{LVV-9804} - LSR-REQ-0007-V-06: Delivered Image Quality\_6

\end{itemize}

\subsubsection{Test Items}


\subsubsection{Predecessors}

\subsubsection{Environment Needs}

\paragraph{Software}

\paragraph{Hardware}

\subsubsection{Input Specification}

\subsubsection{Output Specification}

\subsubsection{Test Procedure}
    \begin{longtable}[]{p{1.3cm}p{2cm}p{13cm}}
    %\toprule
    Step & \multicolumn{2}{@{}l}{Description, Input Data and Expected Result} \\ \toprule
    \endhead

            \multirow{3}{*}{ 1 } & Description &
            \begin{minipage}[t]{13cm}{\footnotesize
            Use DIMM to select observations taken at 0.6 arcsecond seeing.

            \vspace{\dp0}
            } \end{minipage} \\ \cline{2-3}
            & Test Data &
            \begin{minipage}[t]{13cm}{\footnotesize
                No data.
                \vspace{\dp0}
            } \end{minipage} \\ \cline{2-3}
            & Expected Result &
                \begin{minipage}[t]{13cm}{\footnotesize
                Set of images

                \vspace{\dp0}
                } \end{minipage}
        \\ \midrule

            \multirow{3}{*}{ 2 } & Description &
            \begin{minipage}[t]{13cm}{\footnotesize
            Calculate theoretical PSF size for images in step 1 given observing
conditions.

            \vspace{\dp0}
            } \end{minipage} \\ \cline{2-3}
            & Test Data &
            \begin{minipage}[t]{13cm}{\footnotesize
                Metadata from images in step 1

                \vspace{\dp0}
            } \end{minipage} \\ \cline{2-3}
            & Expected Result &
                \begin{minipage}[t]{13cm}{\footnotesize
                Theoretical model of PSF sizes

                \vspace{\dp0}
                } \end{minipage}
        \\ \midrule

            \multirow{3}{*}{ 3 } & Description &
            \begin{minipage}[t]{13cm}{\footnotesize
            Perform single image processing on images from step 1.

            \vspace{\dp0}
            } \end{minipage} \\ \cline{2-3}
            & Test Data &
            \begin{minipage}[t]{13cm}{\footnotesize
                Images from step 1

                \vspace{\dp0}
            } \end{minipage} \\ \cline{2-3}
            & Expected Result &
                \begin{minipage}[t]{13cm}{\footnotesize
                Catalog of detected sources

                \vspace{\dp0}
                } \end{minipage}
        \\ \midrule

            \multirow{3}{*}{ 4 } & Description &
            \begin{minipage}[t]{13cm}{\footnotesize
            Subtract (in quadrature) theoretical model from step 2 from PSF sizes of
sources in catalogs from step 3

            \vspace{\dp0}
            } \end{minipage} \\ \cline{2-3}
            & Test Data &
            \begin{minipage}[t]{13cm}{\footnotesize
                Catalog of sources from step 3\\
Theoretical PSF model from step 2

                \vspace{\dp0}
            } \end{minipage} \\ \cline{2-3}
            & Expected Result &
                \begin{minipage}[t]{13cm}{\footnotesize
                Residual PSF sizes

                \vspace{\dp0}
                } \end{minipage}
        \\ \midrule

            \multirow{3}{*}{ 5 } & Description &
            \begin{minipage}[t]{13cm}{\footnotesize
            Verify that residual PSF size, which will have been contributed by the
LSST system, is no more than 15\% of total PSF size.

            \vspace{\dp0}
            } \end{minipage} \\ \cline{2-3}
            & Test Data &
            \begin{minipage}[t]{13cm}{\footnotesize
                No data.
                \vspace{\dp0}
            } \end{minipage} \\ \cline{2-3}
            & Expected Result &
        \\ \midrule
    \end{longtable}

\subsection{\href{https://jira.lsstcorp.org/secure/Tests.jspa\#/testCase/LVV-T593}{LVV-T593}
    - Image quality at zenith}\label{lvv-t593}

\begin{longtable}[]{llllll}
\toprule
Version & Status & Priority & Verification Type & Owner
\\\midrule
1 & Defined & Normal &
Test & Scott Daniel
\\\bottomrule
\end{longtable}

\subsubsection{Verification Elements}
\begin{itemize}
\item \href{https://jira.lsstcorp.org/browse/LVV-9805}{LVV-9805} - LSR-REQ-0007-V-07: Delivered Image Quality\_7

\end{itemize}

\subsubsection{Test Items}


\subsubsection{Predecessors}

\subsubsection{Environment Needs}

\paragraph{Software}

\paragraph{Hardware}

\subsubsection{Input Specification}

\subsubsection{Output Specification}

\subsubsection{Test Procedure}
    \begin{longtable}[]{p{1.3cm}p{2cm}p{13cm}}
    %\toprule
    Step & \multicolumn{2}{@{}l}{Description, Input Data and Expected Result} \\ \toprule
    \endhead

            \multirow{3}{*}{ 1 } & Description &
            \begin{minipage}[t]{13cm}{\footnotesize
            Take a series of images at zenith in all six filters.

            \vspace{\dp0}
            } \end{minipage} \\ \cline{2-3}
            & Test Data &
            \begin{minipage}[t]{13cm}{\footnotesize
                No data.
                \vspace{\dp0}
            } \end{minipage} \\ \cline{2-3}
            & Expected Result &
                \begin{minipage}[t]{13cm}{\footnotesize
                Set of images

                \vspace{\dp0}
                } \end{minipage}
        \\ \midrule

            \multirow{3}{*}{ 2 } & Description &
            \begin{minipage}[t]{13cm}{\footnotesize
            Calculate theoretical PSF size for the images in step 1 based on
observing conditions.

            \vspace{\dp0}
            } \end{minipage} \\ \cline{2-3}
            & Test Data &
            \begin{minipage}[t]{13cm}{\footnotesize
                Metadata from images in step 1

                \vspace{\dp0}
            } \end{minipage} \\ \cline{2-3}
            & Expected Result &
                \begin{minipage}[t]{13cm}{\footnotesize
                Theoretical model of PSF size

                \vspace{\dp0}
                } \end{minipage}
        \\ \midrule

            \multirow{3}{*}{ 3 } & Description &
            \begin{minipage}[t]{13cm}{\footnotesize
            Perform single image processing on images from step 1

            \vspace{\dp0}
            } \end{minipage} \\ \cline{2-3}
            & Test Data &
            \begin{minipage}[t]{13cm}{\footnotesize
                Images from step 1

                \vspace{\dp0}
            } \end{minipage} \\ \cline{2-3}
            & Expected Result &
                \begin{minipage}[t]{13cm}{\footnotesize
                Catalog of detected sources

                \vspace{\dp0}
                } \end{minipage}
        \\ \midrule

            \multirow{3}{*}{ 4 } & Description &
            \begin{minipage}[t]{13cm}{\footnotesize
            Subtract (in quadrature) theoretical PSF model from step 2 from measured
PSF sizes in catalog from step 3

            \vspace{\dp0}
            } \end{minipage} \\ \cline{2-3}
            & Test Data &
            \begin{minipage}[t]{13cm}{\footnotesize
                Detected source catalog from step 3\\
PSF size model from step 2

                \vspace{\dp0}
            } \end{minipage} \\ \cline{2-3}
            & Expected Result &
                \begin{minipage}[t]{13cm}{\footnotesize
                Residual PSF size

                \vspace{\dp0}
                } \end{minipage}
        \\ \midrule

            \multirow{3}{*}{ 5 } & Description &
            \begin{minipage}[t]{13cm}{\footnotesize
            Verify that residual PSF size does not exceed 0.4 arcseconds

            \vspace{\dp0}
            } \end{minipage} \\ \cline{2-3}
            & Test Data &
            \begin{minipage}[t]{13cm}{\footnotesize
                No data.
                \vspace{\dp0}
            } \end{minipage} \\ \cline{2-3}
            & Expected Result &
        \\ \midrule
    \end{longtable}

\subsection{\href{https://jira.lsstcorp.org/secure/Tests.jspa\#/testCase/LVV-T594}{LVV-T594}
    - Image quality - degradation from zenith}\label{lvv-t594}

\begin{longtable}[]{llllll}
\toprule
Version & Status & Priority & Verification Type & Owner
\\\midrule
1 & Defined & Normal &
Test & Scott Daniel
\\\bottomrule
\end{longtable}

\subsubsection{Verification Elements}
\begin{itemize}
\item \href{https://jira.lsstcorp.org/browse/LVV-268}{LVV-268} - LSR-REQ-0087-V-01: Off Zenith Degradation1

\end{itemize}

\subsubsection{Test Items}


\subsubsection{Predecessors}

\subsubsection{Environment Needs}

\paragraph{Software}

\paragraph{Hardware}

\subsubsection{Input Specification}

\subsubsection{Output Specification}

\subsubsection{Test Procedure}
    \begin{longtable}[]{p{1.3cm}p{2cm}p{13cm}}
    %\toprule
    Step & \multicolumn{2}{@{}l}{Description, Input Data and Expected Result} \\ \toprule
    \endhead

            \multirow{3}{*}{ 1 } & Description &
            \begin{minipage}[t]{13cm}{\footnotesize
            Take images at zenith, airmass=1.4, and airmass=2.0. ~Be sure to get
complete sets of images in all six filters at the three specified
airmasses, changing the airmass rapidly enough that observing conditions
do not change.

            \vspace{\dp0}
            } \end{minipage} \\ \cline{2-3}
            & Test Data &
            \begin{minipage}[t]{13cm}{\footnotesize
                No data.
                \vspace{\dp0}
            } \end{minipage} \\ \cline{2-3}
            & Expected Result &
                \begin{minipage}[t]{13cm}{\footnotesize
                In all six filters, collections of images at airmass=1, 1.4, 2 under
identical (or very similar) observing conditions.

                \vspace{\dp0}
                } \end{minipage}
        \\ \midrule

            \multirow{3}{*}{ 2 } & Description &
            \begin{minipage}[t]{13cm}{\footnotesize
            Use the DIMM to measure observing conditions under which the images in
step 1 were taken.

            \vspace{\dp0}
            } \end{minipage} \\ \cline{2-3}
            & Test Data &
            \begin{minipage}[t]{13cm}{\footnotesize
                No data.
                \vspace{\dp0}
            } \end{minipage} \\ \cline{2-3}
            & Expected Result &
                \begin{minipage}[t]{13cm}{\footnotesize
                Observing conditions for images in step 1

                \vspace{\dp0}
                } \end{minipage}
        \\ \midrule

            \multirow{3}{*}{ 3 } & Description &
            \begin{minipage}[t]{13cm}{\footnotesize
            Calculate the theoretical PSF size for each of the exposures in step 1
given the observing conditions measured in step 2.

            \vspace{\dp0}
            } \end{minipage} \\ \cline{2-3}
            & Test Data &
            \begin{minipage}[t]{13cm}{\footnotesize
                Metadata of images from step 1.\\
DIMM measurements from step 2.

                \vspace{\dp0}
            } \end{minipage} \\ \cline{2-3}
            & Expected Result &
                \begin{minipage}[t]{13cm}{\footnotesize
                Theoretical model of PSF sizes for the images in step 1.

                \vspace{\dp0}
                } \end{minipage}
        \\ \midrule

            \multirow{3}{*}{ 4 } & Description &
            \begin{minipage}[t]{13cm}{\footnotesize
            Perform single image processing on the images in step 1.

            \vspace{\dp0}
            } \end{minipage} \\ \cline{2-3}
            & Test Data &
            \begin{minipage}[t]{13cm}{\footnotesize
                Images from step 1

                \vspace{\dp0}
            } \end{minipage} \\ \cline{2-3}
            & Expected Result &
                \begin{minipage}[t]{13cm}{\footnotesize
                Catalog of detected sources in step 1.

                \vspace{\dp0}
                } \end{minipage}
        \\ \midrule

            \multirow{3}{*}{ 5 } & Description &
            \begin{minipage}[t]{13cm}{\footnotesize
            Subtract (in quadrature) the theoretical PSF sizes from the PSF sizes
measured in step 4.

            \vspace{\dp0}
            } \end{minipage} \\ \cline{2-3}
            & Test Data &
            \begin{minipage}[t]{13cm}{\footnotesize
                Theoretical PSF model from step 4\\
Catalog of measured sources in step 3

                \vspace{\dp0}
            } \end{minipage} \\ \cline{2-3}
            & Expected Result &
                \begin{minipage}[t]{13cm}{\footnotesize
                Residual PSF size

                \vspace{\dp0}
                } \end{minipage}
        \\ \midrule

            \multirow{3}{*}{ 6 } & Description &
            \begin{minipage}[t]{13cm}{\footnotesize
            Verify that the residual PSF size at each airmass does not exceed the
following limits:\\
~\\
0.4 arcseconds at airmass=1\\
0.49 arcseconds at airmass=1.4\\
0.6 arcseconds at airmass=2\\
~\\
so that the system contribution to PSF width degrades no more rapidly
than airmass\^{}0.6

            \vspace{\dp0}
            } \end{minipage} \\ \cline{2-3}
            & Test Data &
            \begin{minipage}[t]{13cm}{\footnotesize
                No data.
                \vspace{\dp0}
            } \end{minipage} \\ \cline{2-3}
            & Expected Result &
        \\ \midrule
    \end{longtable}

\subsection{\href{https://jira.lsstcorp.org/secure/Tests.jspa\#/testCase/LVV-T595}{LVV-T595}
    - PSF ellipticity}\label{lvv-t595}

\begin{longtable}[]{llllll}
\toprule
Version & Status & Priority & Verification Type & Owner
\\\midrule
1 & Defined & Normal &
Test & Scott Daniel
\\\bottomrule
\end{longtable}

\subsubsection{Verification Elements}
\begin{itemize}
\item \href{https://jira.lsstcorp.org/browse/LVV-248}{LVV-248} - LSR-REQ-0092-V-01: Delivered Image Ellipticity1

\end{itemize}

\subsubsection{Test Items}


\subsubsection{Predecessors}

\subsubsection{Environment Needs}

\paragraph{Software}

\paragraph{Hardware}

\subsubsection{Input Specification}

\subsubsection{Output Specification}

\subsubsection{Test Procedure}
    \begin{longtable}[]{p{1.3cm}p{2cm}p{13cm}}
    %\toprule
    Step & \multicolumn{2}{@{}l}{Description, Input Data and Expected Result} \\ \toprule
    \endhead

            \multirow{3}{*}{ 1 } & Description &
            \begin{minipage}[t]{13cm}{\footnotesize
            Take a collection of images in all six filters at a diverse range of
airmasses and atmospheric seeing condtions in uncrowded fields.

            \vspace{\dp0}
            } \end{minipage} \\ \cline{2-3}
            & Test Data &
            \begin{minipage}[t]{13cm}{\footnotesize
                No data.
                \vspace{\dp0}
            } \end{minipage} \\ \cline{2-3}
            & Expected Result &
                \begin{minipage}[t]{13cm}{\footnotesize
                Collection of images

                \vspace{\dp0}
                } \end{minipage}
        \\ \midrule

            \multirow{3}{*}{ 2 } & Description &
            \begin{minipage}[t]{13cm}{\footnotesize
            Perform single image processing on the images from step 1.

            \vspace{\dp0}
            } \end{minipage} \\ \cline{2-3}
            & Test Data &
            \begin{minipage}[t]{13cm}{\footnotesize
                Images from step 1.

                \vspace{\dp0}
            } \end{minipage} \\ \cline{2-3}
            & Expected Result &
                \begin{minipage}[t]{13cm}{\footnotesize
                Catalog of detected sources.

                \vspace{\dp0}
                } \end{minipage}
        \\ \midrule

            \multirow{3}{*}{ 3 } & Description &
            \begin{minipage}[t]{13cm}{\footnotesize
            For each full-focal plane exposure, select all of the measured,
unresolved point sources brighter than some threshold (17th magnitude?).
~Calculate the ellipticity of the PSF measured at each of these sources.

            \vspace{\dp0}
            } \end{minipage} \\ \cline{2-3}
            & Test Data &
            \begin{minipage}[t]{13cm}{\footnotesize
                Catalog of measured sources from step 2.

                \vspace{\dp0}
            } \end{minipage} \\ \cline{2-3}
            & Expected Result &
                \begin{minipage}[t]{13cm}{\footnotesize
                Distribution of PSF ellipticities in the images.

                \vspace{\dp0}
                } \end{minipage}
        \\ \midrule

            \multirow{3}{*}{ 4 } & Description &
            \begin{minipage}[t]{13cm}{\footnotesize
            For each full-focal plane exposure, verify that the median PSF
ellipticity of the sources from step 3 is less than or equal to 0.04.\\
~\\
Verify that no more than 5\% of the sources from step 3 have PSF
ellipticity greater than 0.07.

            \vspace{\dp0}
            } \end{minipage} \\ \cline{2-3}
            & Test Data &
            \begin{minipage}[t]{13cm}{\footnotesize
                PSF ellipticity distributions from step 3.

                \vspace{\dp0}
            } \end{minipage} \\ \cline{2-3}
            & Expected Result &
        \\ \midrule
    \end{longtable}

\subsection{\href{https://jira.lsstcorp.org/secure/Tests.jspa\#/testCase/LVV-T596}{LVV-T596}
    - Depth: r-band}\label{lvv-t596}

\begin{longtable}[]{llllll}
\toprule
Version & Status & Priority & Verification Type & Owner
\\\midrule
1 & Defined & Normal &
Test & Scott Daniel
\\\bottomrule
\end{longtable}

\subsubsection{Verification Elements}
\begin{itemize}
\item \href{https://jira.lsstcorp.org/browse/LVV-3751}{LVV-3751} - LSR-REQ-0089-V-01: r-band Reference Depth1

\end{itemize}

\subsubsection{Test Items}


\subsubsection{Predecessors}

\subsubsection{Environment Needs}

\paragraph{Software}

\paragraph{Hardware}

\subsubsection{Input Specification}

\subsubsection{Output Specification}

\subsubsection{Test Procedure}
    \begin{longtable}[]{p{1.3cm}p{2cm}p{13cm}}
    %\toprule
    Step & \multicolumn{2}{@{}l}{Description, Input Data and Expected Result} \\ \toprule
    \endhead

            \multirow{3}{*}{ 1 } & Description &
            \begin{minipage}[t]{13cm}{\footnotesize
            Upon completion of mini-survey, select all exposures taken at or near\\
~\\
band = r\\
airmass = 1\\
seeing = 0.7 arcsecond\\
sky brightness = 21 magnitudes per square arcsecond

            \vspace{\dp0}
            } \end{minipage} \\ \cline{2-3}
            & Test Data &
            \begin{minipage}[t]{13cm}{\footnotesize
                No data.
                \vspace{\dp0}
            } \end{minipage} \\ \cline{2-3}
            & Expected Result &
                \begin{minipage}[t]{13cm}{\footnotesize
                Set of exposures taken at reference conditions

                \vspace{\dp0}
                } \end{minipage}
        \\ \midrule

            \multirow{3}{*}{ 2 } & Description &
            \begin{minipage}[t]{13cm}{\footnotesize
            Run single visit processing on images from step 1

            \vspace{\dp0}
            } \end{minipage} \\ \cline{2-3}
            & Test Data &
            \begin{minipage}[t]{13cm}{\footnotesize
                images from step 1

                \vspace{\dp0}
            } \end{minipage} \\ \cline{2-3}
            & Expected Result &
                \begin{minipage}[t]{13cm}{\footnotesize
                catalog of measured sources

                \vspace{\dp0}
                } \end{minipage}
        \\ \midrule

            \multirow{3}{*}{ 3 } & Description &
            \begin{minipage}[t]{13cm}{\footnotesize
            For each visit, find the 5-sigma limiting magnitude by examining the
distribution of sources detected at SNR=5

            \vspace{\dp0}
            } \end{minipage} \\ \cline{2-3}
            & Test Data &
            \begin{minipage}[t]{13cm}{\footnotesize
                catalog of measured sources from step 2

                \vspace{\dp0}
            } \end{minipage} \\ \cline{2-3}
            & Expected Result &
                \begin{minipage}[t]{13cm}{\footnotesize
                distribution of 5-sigma limiting magnitudes

                \vspace{\dp0}
                } \end{minipage}
        \\ \midrule

            \multirow{3}{*}{ 4 } & Description &
            \begin{minipage}[t]{13cm}{\footnotesize
            Verify that the median of the distribution of 5-sigma limiting
magnitudes is no brighter than 24.7 AB magnitudes

            \vspace{\dp0}
            } \end{minipage} \\ \cline{2-3}
            & Test Data &
            \begin{minipage}[t]{13cm}{\footnotesize
                Distribution of 5-sigma limiting magnitudes from step 3

                \vspace{\dp0}
            } \end{minipage} \\ \cline{2-3}
            & Expected Result &
        \\ \midrule

            \multirow{3}{*}{ 5 } & Description &
            \begin{minipage}[t]{13cm}{\footnotesize
            Verify that no more than 10\% of the images have a 5-sigma limiting
magnitudes brighter than 24.4 AB magnitudes

            \vspace{\dp0}
            } \end{minipage} \\ \cline{2-3}
            & Test Data &
            \begin{minipage}[t]{13cm}{\footnotesize
                Distribution of 5-sigma limiting magnitudes from step 3

                \vspace{\dp0}
            } \end{minipage} \\ \cline{2-3}
            & Expected Result &
        \\ \midrule
    \end{longtable}

\subsection{\href{https://jira.lsstcorp.org/secure/Tests.jspa\#/testCase/LVV-T597}{LVV-T597}
    - Depth variation over field of view}\label{lvv-t597}

\begin{longtable}[]{llllll}
\toprule
Version & Status & Priority & Verification Type & Owner
\\\midrule
1 & Defined & Normal &
Test & Scott Daniel
\\\bottomrule
\end{longtable}

\subsubsection{Verification Elements}
\begin{itemize}
\item \href{https://jira.lsstcorp.org/browse/LVV-258}{LVV-258} - LSR-REQ-0109-V-01: Depth Variation Over FOV1

\end{itemize}

\subsubsection{Test Items}


\subsubsection{Predecessors}

\subsubsection{Environment Needs}

\paragraph{Software}

\paragraph{Hardware}

\subsubsection{Input Specification}

\subsubsection{Output Specification}

\subsubsection{Test Procedure}
    \begin{longtable}[]{p{1.3cm}p{2cm}p{13cm}}
    %\toprule
    Step & \multicolumn{2}{@{}l}{Description, Input Data and Expected Result} \\ \toprule
    \endhead

            \multirow{3}{*}{ 1 } & Description &
            \begin{minipage}[t]{13cm}{\footnotesize
            After conclusion of a mini-survey, select all pointings that meet the
depth requirement specified in LSR-REQ-0090 (LVV-263)

            \vspace{\dp0}
            } \end{minipage} \\ \cline{2-3}
            & Test Data &
            \begin{minipage}[t]{13cm}{\footnotesize
                No data.
                \vspace{\dp0}
            } \end{minipage} \\ \cline{2-3}
            & Expected Result &
                \begin{minipage}[t]{13cm}{\footnotesize
                Set of images that meet the fiducial depth requirement.

                \vspace{\dp0}
                } \end{minipage}
        \\ \midrule

            \multirow{3}{*}{ 2 } & Description &
            \begin{minipage}[t]{13cm}{\footnotesize
            Perform single image processing on the images from step 1 (if not done
already)

            \vspace{\dp0}
            } \end{minipage} \\ \cline{2-3}
            & Test Data &
            \begin{minipage}[t]{13cm}{\footnotesize
                Images from step 1 that meet the fiducial depth requirement

                \vspace{\dp0}
            } \end{minipage} \\ \cline{2-3}
            & Expected Result &
                \begin{minipage}[t]{13cm}{\footnotesize
                Catalogs of measured sources from the images in step 1

                \vspace{\dp0}
                } \end{minipage}
        \\ \midrule

            \multirow{3}{*}{ 3 } & Description &
            \begin{minipage}[t]{13cm}{\footnotesize
            Subdivide each image into small regions (\textasciitilde{}1 CCD should
be enough, since 1/189 = 5*10\^{}-3). ~In each region, examine the SNR
distribution of measured sources to determine the 5-sigma limiting
magnitude for that region of the focal plane.

            \vspace{\dp0}
            } \end{minipage} \\ \cline{2-3}
            & Test Data &
            \begin{minipage}[t]{13cm}{\footnotesize
                Measured sources from step 3

                \vspace{\dp0}
            } \end{minipage} \\ \cline{2-3}
            & Expected Result &
                \begin{minipage}[t]{13cm}{\footnotesize
                Distribution of depths for subregions of the focal plane

                \vspace{\dp0}
                } \end{minipage}
        \\ \midrule

            \multirow{3}{*}{ 4 } & Description &
            \begin{minipage}[t]{13cm}{\footnotesize
            Verify that, for each exposure, no more than 15\% of the focal plane
area has a 5-sigma limiting magnitude 0.2 AB magnitude brighter than the
median 5-sigma limiting magnitude for the entire exposure.

            \vspace{\dp0}
            } \end{minipage} \\ \cline{2-3}
            & Test Data &
            \begin{minipage}[t]{13cm}{\footnotesize
                No data.
                \vspace{\dp0}
            } \end{minipage} \\ \cline{2-3}
            & Expected Result &
        \\ \midrule
    \end{longtable}

\subsection{\href{https://jira.lsstcorp.org/secure/Tests.jspa\#/testCase/LVV-T938}{LVV-T938}
    - Level 2 reproducibility (same computer hardware)}\label{lvv-t938}

\begin{longtable}[]{llllll}
\toprule
Version & Status & Priority & Verification Type & Owner
\\\midrule
1 & Defined & Normal &
Test & Scott Daniel
\\\bottomrule
\end{longtable}

\subsubsection{Verification Elements}
\begin{itemize}
\item \href{https://jira.lsstcorp.org/browse/LVV-1237}{LVV-1237} - OSS-REQ-0123-V-01: Reproducibility

\end{itemize}

\subsubsection{Test Items}


\subsubsection{Predecessors}

\subsubsection{Environment Needs}

\paragraph{Software}

\paragraph{Hardware}

\subsubsection{Input Specification}

\subsubsection{Output Specification}

\subsubsection{Test Procedure}
    \begin{longtable}[]{p{1.3cm}p{2cm}p{13cm}}
    %\toprule
    Step & \multicolumn{2}{@{}l}{Description, Input Data and Expected Result} \\ \toprule
    \endhead

            \multirow{3}{*}{ 1 } & Description &
            \begin{minipage}[t]{13cm}{\footnotesize
            Take precursor mini-survey data

            \vspace{\dp0}
            } \end{minipage} \\ \cline{2-3}
            & Test Data &
            \begin{minipage}[t]{13cm}{\footnotesize
                No data.
                \vspace{\dp0}
            } \end{minipage} \\ \cline{2-3}
            & Expected Result &
                \begin{minipage}[t]{13cm}{\footnotesize
                Images and calibration products

                \vspace{\dp0}
                } \end{minipage}
        \\ \midrule

            \multirow{3}{*}{ 2 } & Description &
            \begin{minipage}[t]{13cm}{\footnotesize
            Run Level 2 processing on precursor data from step 1

            \vspace{\dp0}
            } \end{minipage} \\ \cline{2-3}
            & Test Data &
            \begin{minipage}[t]{13cm}{\footnotesize
                Images and calibration products from step 1

                \vspace{\dp0}
            } \end{minipage} \\ \cline{2-3}
            & Expected Result &
                \begin{minipage}[t]{13cm}{\footnotesize
                Coadded images\\
Catalogs detected on coadded images

                \vspace{\dp0}
                } \end{minipage}
        \\ \midrule

            \multirow{3}{*}{ 3 } & Description &
            \begin{minipage}[t]{13cm}{\footnotesize
            Re-run Level 2 processing on data from step 1, using the same system as
in step 2

            \vspace{\dp0}
            } \end{minipage} \\ \cline{2-3}
            & Test Data &
            \begin{minipage}[t]{13cm}{\footnotesize
                Images and calibration products from step 1

                \vspace{\dp0}
            } \end{minipage} \\ \cline{2-3}
            & Expected Result &
                \begin{minipage}[t]{13cm}{\footnotesize
                Coadded images\\
Catalogs detected on coadded images

                \vspace{\dp0}
                } \end{minipage}
        \\ \midrule

            \multirow{3}{*}{ 4 } & Description &
            \begin{minipage}[t]{13cm}{\footnotesize
            Verify that catalogs from step 2 and step 3 identify all of the same
sources and give identical measurements (since the two analyses were run
on the same system).

            \vspace{\dp0}
            } \end{minipage} \\ \cline{2-3}
            & Test Data &
            \begin{minipage}[t]{13cm}{\footnotesize
                Catalogs from steps 2 and 3

                \vspace{\dp0}
            } \end{minipage} \\ \cline{2-3}
            & Expected Result &
                \begin{minipage}[t]{13cm}{\footnotesize
                Catalogs are identical

                \vspace{\dp0}
                } \end{minipage}
        \\ \midrule
    \end{longtable}

\subsection{\href{https://jira.lsstcorp.org/secure/Tests.jspa\#/testCase/LVV-T939}{LVV-T939}
    - Level 1 reproducibility (same computer hardware)}\label{lvv-t939}

\begin{longtable}[]{llllll}
\toprule
Version & Status & Priority & Verification Type & Owner
\\\midrule
1 & Defined & Normal &
Test & Scott Daniel
\\\bottomrule
\end{longtable}

\subsubsection{Verification Elements}
\begin{itemize}
\item \href{https://jira.lsstcorp.org/browse/LVV-1237}{LVV-1237} - OSS-REQ-0123-V-01: Reproducibility

\end{itemize}

\subsubsection{Test Items}


\subsubsection{Predecessors}

\subsubsection{Environment Needs}

\paragraph{Software}

\paragraph{Hardware}

\subsubsection{Input Specification}
LVV-T938 has been run


\subsubsection{Output Specification}

\subsubsection{Test Procedure}
    \begin{longtable}[]{p{1.3cm}p{2cm}p{13cm}}
    %\toprule
    Step & \multicolumn{2}{@{}l}{Description, Input Data and Expected Result} \\ \toprule
    \endhead

            \multirow{3}{*}{ 1 } & Description &
            \begin{minipage}[t]{13cm}{\footnotesize
            Run Level 1 analysis on images taken in LVV-T938 using coadds from the
first run as templates.

            \vspace{\dp0}
            } \end{minipage} \\ \cline{2-3}
            & Test Data &
            \begin{minipage}[t]{13cm}{\footnotesize
                Precursor survey and coadded images from LVV-T938

                \vspace{\dp0}
            } \end{minipage} \\ \cline{2-3}
            & Expected Result &
                \begin{minipage}[t]{13cm}{\footnotesize
                Set of DIASources

                \vspace{\dp0}
                } \end{minipage}
        \\ \midrule

            \multirow{3}{*}{ 2 } & Description &
            \begin{minipage}[t]{13cm}{\footnotesize
            Run Level 1 analysis on images taken in LVV-T938 using coadds from the
first run as templates. ~Run on the same system as step 1.

            \vspace{\dp0}
            } \end{minipage} \\ \cline{2-3}
            & Test Data &
            \begin{minipage}[t]{13cm}{\footnotesize
                Precursor survey and coadded images from LVV-T938

                \vspace{\dp0}
            } \end{minipage} \\ \cline{2-3}
            & Expected Result &
                \begin{minipage}[t]{13cm}{\footnotesize
                Set of DIASources

                \vspace{\dp0}
                } \end{minipage}
        \\ \midrule

            \multirow{3}{*}{ 3 } & Description &
            \begin{minipage}[t]{13cm}{\footnotesize
            Verify that the sets of DIASources produced in steps 1 and 2 are
identical, since they were run on the same system.

            \vspace{\dp0}
            } \end{minipage} \\ \cline{2-3}
            & Test Data &
            \begin{minipage}[t]{13cm}{\footnotesize
                DIASources detected in steps 1 and 2

                \vspace{\dp0}
            } \end{minipage} \\ \cline{2-3}
            & Expected Result &
                \begin{minipage}[t]{13cm}{\footnotesize
                Sets of DIASources should be identical

                \vspace{\dp0}
                } \end{minipage}
        \\ \midrule
    \end{longtable}

\subsection{\href{https://jira.lsstcorp.org/secure/Tests.jspa\#/testCase/LVV-T940}{LVV-T940}
    - Level 2 reproducibility (different computer hardware)}\label{lvv-t940}

\begin{longtable}[]{llllll}
\toprule
Version & Status & Priority & Verification Type & Owner
\\\midrule
1 & Defined & Normal &
Test & Scott Daniel
\\\bottomrule
\end{longtable}

\subsubsection{Verification Elements}
\begin{itemize}
\item \href{https://jira.lsstcorp.org/browse/LVV-1237}{LVV-1237} - OSS-REQ-0123-V-01: Reproducibility

\end{itemize}

\subsubsection{Test Items}


\subsubsection{Predecessors}

\subsubsection{Environment Needs}

\paragraph{Software}

\paragraph{Hardware}

\subsubsection{Input Specification}

\subsubsection{Output Specification}

\subsubsection{Test Procedure}
    \begin{longtable}[]{p{1.3cm}p{2cm}p{13cm}}
    %\toprule
    Step & \multicolumn{2}{@{}l}{Description, Input Data and Expected Result} \\ \toprule
    \endhead

            \multirow{3}{*}{ 1 } & Description &
            \begin{minipage}[t]{13cm}{\footnotesize
            Take precursor data and calibration products from mini survey

            \vspace{\dp0}
            } \end{minipage} \\ \cline{2-3}
            & Test Data &
            \begin{minipage}[t]{13cm}{\footnotesize
                No data.
                \vspace{\dp0}
            } \end{minipage} \\ \cline{2-3}
            & Expected Result &
                \begin{minipage}[t]{13cm}{\footnotesize
                Calibration products\\
Images

                \vspace{\dp0}
                } \end{minipage}
        \\ \midrule

            \multirow{3}{*}{ 2 } & Description &
            \begin{minipage}[t]{13cm}{\footnotesize
            Run Level 2 analysis on data from step 1

            \vspace{\dp0}
            } \end{minipage} \\ \cline{2-3}
            & Test Data &
            \begin{minipage}[t]{13cm}{\footnotesize
                Calibration products and images from step 1

                \vspace{\dp0}
            } \end{minipage} \\ \cline{2-3}
            & Expected Result &
                \begin{minipage}[t]{13cm}{\footnotesize
                Coadded images\\
Catalogs of detected sources

                \vspace{\dp0}
                } \end{minipage}
        \\ \midrule

            \multirow{3}{*}{ 3 } & Description &
            \begin{minipage}[t]{13cm}{\footnotesize
            Re-run Level 2 analysis on data from step 1 using a different hardware
system

            \vspace{\dp0}
            } \end{minipage} \\ \cline{2-3}
            & Test Data &
            \begin{minipage}[t]{13cm}{\footnotesize
                Calibration products and images from step 1

                \vspace{\dp0}
            } \end{minipage} \\ \cline{2-3}
            & Expected Result &
                \begin{minipage}[t]{13cm}{\footnotesize
                Coadded images\\
Catalogs of detected sources

                \vspace{\dp0}
                } \end{minipage}
        \\ \midrule

            \multirow{3}{*}{ 4 } & Description &
            \begin{minipage}[t]{13cm}{\footnotesize
            Verify that the catalogs from steps 2 and 3 agree to within some small
tolerance

            \vspace{\dp0}
            } \end{minipage} \\ \cline{2-3}
            & Test Data &
            \begin{minipage}[t]{13cm}{\footnotesize
                Catalogs of detected sources from steps 2 and 3

                \vspace{\dp0}
            } \end{minipage} \\ \cline{2-3}
            & Expected Result &
                \begin{minipage}[t]{13cm}{\footnotesize
                Catalogs will agree to within some tolerance

                \vspace{\dp0}
                } \end{minipage}
        \\ \midrule
    \end{longtable}

\subsection{\href{https://jira.lsstcorp.org/secure/Tests.jspa\#/testCase/LVV-T941}{LVV-T941}
    - Level 1 reproducibility (different computer hardware)}\label{lvv-t941}

\begin{longtable}[]{llllll}
\toprule
Version & Status & Priority & Verification Type & Owner
\\\midrule
1 & Defined & Normal &
Test & Scott Daniel
\\\bottomrule
\end{longtable}

\subsubsection{Verification Elements}
\begin{itemize}
\item \href{https://jira.lsstcorp.org/browse/LVV-1237}{LVV-1237} - OSS-REQ-0123-V-01: Reproducibility

\end{itemize}

\subsubsection{Test Items}


\subsubsection{Predecessors}

\subsubsection{Environment Needs}

\paragraph{Software}

\paragraph{Hardware}

\subsubsection{Input Specification}
LVV-T940 has been run


\subsubsection{Output Specification}

\subsubsection{Test Procedure}
    \begin{longtable}[]{p{1.3cm}p{2cm}p{13cm}}
    %\toprule
    Step & \multicolumn{2}{@{}l}{Description, Input Data and Expected Result} \\ \toprule
    \endhead

            \multirow{3}{*}{ 1 } & Description &
            \begin{minipage}[t]{13cm}{\footnotesize
            Run Level 1 analysis on data from step 1 of LVV-T940, using coadded
images from first Level 2 run as templates

            \vspace{\dp0}
            } \end{minipage} \\ \cline{2-3}
            & Test Data &
            \begin{minipage}[t]{13cm}{\footnotesize
                Precursor data and coadded images from step 2 LVV-T940

                \vspace{\dp0}
            } \end{minipage} \\ \cline{2-3}
            & Expected Result &
                \begin{minipage}[t]{13cm}{\footnotesize
                Catalog of DIASources

                \vspace{\dp0}
                } \end{minipage}
        \\ \midrule

            \multirow{3}{*}{ 2 } & Description &
            \begin{minipage}[t]{13cm}{\footnotesize
            Re-run Level 1 analysis on data from step 1 of LVV-T940 using coadded
images from second Level 2 run and using a different computer system
than step 1 of this requirement

            \vspace{\dp0}
            } \end{minipage} \\ \cline{2-3}
            & Test Data &
            \begin{minipage}[t]{13cm}{\footnotesize
                Precursor data and coadded images from step 3 of LVV-T940

                \vspace{\dp0}
            } \end{minipage} \\ \cline{2-3}
            & Expected Result &
                \begin{minipage}[t]{13cm}{\footnotesize
                Catalog of DIASources

                \vspace{\dp0}
                } \end{minipage}
        \\ \midrule

            \multirow{3}{*}{ 3 } & Description &
            \begin{minipage}[t]{13cm}{\footnotesize
            Verify that DIASource catalogs from steps 1 and 2 agree to within some
small tolerance

            \vspace{\dp0}
            } \end{minipage} \\ \cline{2-3}
            & Test Data &
            \begin{minipage}[t]{13cm}{\footnotesize
                DIASource catalogs from steps 1 and 2

                \vspace{\dp0}
            } \end{minipage} \\ \cline{2-3}
            & Expected Result &
                \begin{minipage}[t]{13cm}{\footnotesize
                DIASource catalogs should agree to within some small tolerance

                \vspace{\dp0}
                } \end{minipage}
        \\ \midrule
    \end{longtable}

\subsection{\href{https://jira.lsstcorp.org/secure/Tests.jspa\#/testCase/LVV-T942}{LVV-T942}
    - Provenance on Level 2 catalogs}\label{lvv-t942}

\begin{longtable}[]{llllll}
\toprule
Version & Status & Priority & Verification Type & Owner
\\\midrule
1 & Defined & Normal &
Test & Scott Daniel
\\\bottomrule
\end{longtable}

\subsubsection{Verification Elements}
\begin{itemize}
\item \href{https://jira.lsstcorp.org/browse/LVV-1234}{LVV-1234} - OSS-REQ-0122-V-01: Provenance

\end{itemize}

\subsubsection{Test Items}
Verify that provenance information is stored in level 2 catalog products



\subsubsection{Predecessors}

\subsubsection{Environment Needs}

\paragraph{Software}

\paragraph{Hardware}

\subsubsection{Input Specification}

\subsubsection{Output Specification}

\subsubsection{Test Procedure}
    \begin{longtable}[]{p{1.3cm}p{2cm}p{13cm}}
    %\toprule
    Step & \multicolumn{2}{@{}l}{Description, Input Data and Expected Result} \\ \toprule
    \endhead

                \multirow{3}{*}{\parbox{1.3cm}{ 1-1
                {\scriptsize from \hyperref[lvv-t64]
                {LVV-T64} } } }

                & {\small Description} &
                \begin{minipage}[t]{13cm}{\scriptsize
                
                \vspace{\dp0}
                } \end{minipage} \\ \cdashline{2-3}
                & {\small Test Data} &
                \begin{minipage}[t]{13cm}{\scriptsize
                } \end{minipage} \\ \cdashline{2-3}
                & {\small Expected Result} &
                \\ \hdashline


                \multirow{3}{*}{\parbox{1.3cm}{ 1-2
                {\scriptsize from \hyperref[lvv-t64]
                {LVV-T64} } } }

                & {\small Description} &
                \begin{minipage}[t]{13cm}{\scriptsize
                Query and verify provenance of input images, and software versions that
went into producing stack.

                \vspace{\dp0}
                } \end{minipage} \\ \cdashline{2-3}
                & {\small Test Data} &
                \begin{minipage}[t]{13cm}{\scriptsize
                } \end{minipage} \\ \cdashline{2-3}
                & {\small Expected Result} &
                \\ \hdashline


                \multirow{3}{*}{\parbox{1.3cm}{ 1-3
                {\scriptsize from \hyperref[lvv-t64]
                {LVV-T64} } } }

                & {\small Description} &
                \begin{minipage}[t]{13cm}{\scriptsize
                Test re-generating 10 different coadds tract+patches based on the
provenance image given

                \vspace{\dp0}
                } \end{minipage} \\ \cdashline{2-3}
                & {\small Test Data} &
                \begin{minipage}[t]{13cm}{\scriptsize
                } \end{minipage} \\ \cdashline{2-3}
                & {\small Expected Result} &
                \\ \hdashline


        \\ \midrule

            \multirow{3}{*}{ 2 } & Description &
            \begin{minipage}[t]{13cm}{\footnotesize
            Run source detection on coadded images from step 1

            \vspace{\dp0}
            } \end{minipage} \\ \cline{2-3}
            & Test Data &
            \begin{minipage}[t]{13cm}{\footnotesize
                Coadded images from step 1

                \vspace{\dp0}
            } \end{minipage} \\ \cline{2-3}
            & Expected Result &
                \begin{minipage}[t]{13cm}{\footnotesize
                Catalog of sources detected on coadds

                \vspace{\dp0}
                } \end{minipage}
        \\ \midrule

            \multirow{3}{*}{ 3 } & Description &
            \begin{minipage}[t]{13cm}{\footnotesize
            Verify that correct provenance information is stored in catalogs from
step 2

            \vspace{\dp0}
            } \end{minipage} \\ \cline{2-3}
            & Test Data &
            \begin{minipage}[t]{13cm}{\footnotesize
                No data.
                \vspace{\dp0}
            } \end{minipage} \\ \cline{2-3}
            & Expected Result &
        \\ \midrule
    \end{longtable}

\subsection{\href{https://jira.lsstcorp.org/secure/Tests.jspa\#/testCase/LVV-T943}{LVV-T943}
    - Provenance in Level 1 catalogs}\label{lvv-t943}

\begin{longtable}[]{llllll}
\toprule
Version & Status & Priority & Verification Type & Owner
\\\midrule
1 & Defined & Normal &
Test & Scott Daniel
\\\bottomrule
\end{longtable}

\subsubsection{Verification Elements}
\begin{itemize}
\item \href{https://jira.lsstcorp.org/browse/LVV-1234}{LVV-1234} - OSS-REQ-0122-V-01: Provenance

\end{itemize}

\subsubsection{Test Items}
Verify that provenance information is correctly stored in Level 1
catalog products



\subsubsection{Predecessors}

\subsubsection{Environment Needs}

\paragraph{Software}

\paragraph{Hardware}

\subsubsection{Input Specification}

\subsubsection{Output Specification}

\subsubsection{Test Procedure}
    \begin{longtable}[]{p{1.3cm}p{2cm}p{13cm}}
    %\toprule
    Step & \multicolumn{2}{@{}l}{Description, Input Data and Expected Result} \\ \toprule
    \endhead

                \multirow{3}{*}{\parbox{1.3cm}{ 1-1
                {\scriptsize from \hyperref[lvv-t33]
                {LVV-T33} } } }

                & {\small Description} &
                \begin{minipage}[t]{13cm}{\scriptsize
                Verify that time of exposure start/end, site metadata, telescope
metadata, and camera metadata are stored in DMS system.\\
~\\

                \vspace{\dp0}
                } \end{minipage} \\ \cdashline{2-3}
                & {\small Test Data} &
                \begin{minipage}[t]{13cm}{\scriptsize
                } \end{minipage} \\ \cdashline{2-3}
                & {\small Expected Result} &
                \\ \hdashline


                \multirow{3}{*}{\parbox{1.3cm}{ 1-2
                {\scriptsize from \hyperref[lvv-t33]
                {LVV-T33} } } }

                & {\small Description} &
                \begin{minipage}[t]{13cm}{\scriptsize
                
                \vspace{\dp0}
                } \end{minipage} \\ \cdashline{2-3}
                & {\small Test Data} &
                \begin{minipage}[t]{13cm}{\scriptsize
                } \end{minipage} \\ \cdashline{2-3}
                & {\small Expected Result} &
                \\ \hdashline


                \multirow{3}{*}{\parbox{1.3cm}{ 1-3
                {\scriptsize from \hyperref[lvv-t33]
                {LVV-T33} } } }

                & {\small Description} &
                \begin{minipage}[t]{13cm}{\scriptsize
                
                \vspace{\dp0}
                } \end{minipage} \\ \cdashline{2-3}
                & {\small Test Data} &
                \begin{minipage}[t]{13cm}{\scriptsize
                } \end{minipage} \\ \cdashline{2-3}
                & {\small Expected Result} &
                \\ \hdashline


        \\ \midrule

            \multirow{3}{*}{ 2 } & Description &
            \begin{minipage}[t]{13cm}{\footnotesize
            Detect sources on difference images from step 1

            \vspace{\dp0}
            } \end{minipage} \\ \cline{2-3}
            & Test Data &
            \begin{minipage}[t]{13cm}{\footnotesize
                Difference images from step 1

                \vspace{\dp0}
            } \end{minipage} \\ \cline{2-3}
            & Expected Result &
                \begin{minipage}[t]{13cm}{\footnotesize
                Catalog of DIASources

                \vspace{\dp0}
                } \end{minipage}
        \\ \midrule

            \multirow{3}{*}{ 3 } & Description &
            \begin{minipage}[t]{13cm}{\footnotesize
            Verify that provenance information is correctly stored in DIASource
catalogs from step 2

            \vspace{\dp0}
            } \end{minipage} \\ \cline{2-3}
            & Test Data &
            \begin{minipage}[t]{13cm}{\footnotesize
                No data.
                \vspace{\dp0}
            } \end{minipage} \\ \cline{2-3}
            & Expected Result &
        \\ \midrule
    \end{longtable}

\subsection{\href{https://jira.lsstcorp.org/secure/Tests.jspa\#/testCase/LVV-T950}{LVV-T950}
    - DIASource misassociation rate}\label{lvv-t950}

\begin{longtable}[]{llllll}
\toprule
Version & Status & Priority & Verification Type & Owner
\\\midrule
1 & Defined & Normal &
Test & Scott Daniel
\\\bottomrule
\end{longtable}

\subsubsection{Verification Elements}
\begin{itemize}
\item \href{https://jira.lsstcorp.org/browse/LVV-1285}{LVV-1285} - OSS-REQ-0160-V-01: Level 1 Difference Source - Difference Object
Association Quality

\end{itemize}

\subsubsection{Test Items}
Verify that DIASources are not misassociated with DIAObjects at too
large a rate



\subsubsection{Predecessors}

\subsubsection{Environment Needs}

\paragraph{Software}

\paragraph{Hardware}

\subsubsection{Input Specification}

\subsubsection{Output Specification}

\subsubsection{Test Procedure}
    \begin{longtable}[]{p{1.3cm}p{2cm}p{13cm}}
    %\toprule
    Step & \multicolumn{2}{@{}l}{Description, Input Data and Expected Result} \\ \toprule
    \endhead

            \multirow{3}{*}{ 1 } & Description &
            \begin{minipage}[t]{13cm}{\footnotesize
            Collect images from a minisurvey

            \vspace{\dp0}
            } \end{minipage} \\ \cline{2-3}
            & Test Data &
            \begin{minipage}[t]{13cm}{\footnotesize
                No data.
                \vspace{\dp0}
            } \end{minipage} \\ \cline{2-3}
            & Expected Result &
                \begin{minipage}[t]{13cm}{\footnotesize
                Images and calibration products from a minisurvey

                \vspace{\dp0}
                } \end{minipage}
        \\ \midrule

            \multirow{3}{*}{ 2 } & Description &
            \begin{minipage}[t]{13cm}{\footnotesize
            Use synpipe (or some other tool) to inject variable sources with
different degrees of variability and different signal-to-noise ratios
into the images from step 1

            \vspace{\dp0}
            } \end{minipage} \\ \cline{2-3}
            & Test Data &
            \begin{minipage}[t]{13cm}{\footnotesize
                Images from step 1

                \vspace{\dp0}
            } \end{minipage} \\ \cline{2-3}
            & Expected Result &
                \begin{minipage}[t]{13cm}{\footnotesize
                Images with an artificial population of variable sources

                \vspace{\dp0}
                } \end{minipage}
        \\ \midrule

            \multirow{3}{*}{ 3 } & Description &
            \begin{minipage}[t]{13cm}{\footnotesize
            Perform level 1 processing on the images with artificial variable
sources in them

            \vspace{\dp0}
            } \end{minipage} \\ \cline{2-3}
            & Test Data &
            \begin{minipage}[t]{13cm}{\footnotesize
                Images with artificial variable sources from step 2

                \vspace{\dp0}
            } \end{minipage} \\ \cline{2-3}
            & Expected Result &
                \begin{minipage}[t]{13cm}{\footnotesize
                Catalog of DIASources and DIAObjects

                \vspace{\dp0}
                } \end{minipage}
        \\ \midrule

            \multirow{3}{*}{ 4 } & Description &
            \begin{minipage}[t]{13cm}{\footnotesize
            Consider only the artificial variables injected in step 2 (for which we
know the ground truth). ~Verify that they are not misassociated with
DIAObjects at a larger than acceptable rate.

            \vspace{\dp0}
            } \end{minipage} \\ \cline{2-3}
            & Test Data &
            \begin{minipage}[t]{13cm}{\footnotesize
                DIASources and DIAObjects from step 3

                \vspace{\dp0}
            } \end{minipage} \\ \cline{2-3}
            & Expected Result &
        \\ \midrule
    \end{longtable}

\appendix
% generated from JIRA project LVV
% using template at /var/jenkins_home/.local/lib/python3.7/site-packages/docsteady/templates/dm-spec-appendix.latex.jinja2.
% Collecting ATM data from folder: "/Project Systems Engineering/Commissioning Science Verification"
% using dosteady version 1.2rc1
% Please do not edit -- update information in Jira instead

\section{Traceability}
\scriptsize{
\begin{longtable}[]{rp{6cm}p{2cm}p{2cm}p{1.5cm}p{1.5cm}p{1.5cm}}
\toprule
& Verification Elements & VE Status & VE Assignee &Test Cases & TC Status & TC Owner \tabularnewline
\midrule
\endhead

1 &
    \href{https://jira.lsstcorp.org/browse/LVV-1544}{LVV-1544 - OSS-REQ-0228-V-02: Image Size in Pixels
}
& Not Covered & Bo Xin
        & 
        \hyperref[lvv-t293]{LVV-T293} 
        & Draft & Keith Bechtol
        \\
            &&&
        & 
        \hyperref[lvv-t295]{LVV-T295} 
        & Draft & Keith Bechtol
        \\
            &&&
        & 
        \hyperref[lvv-t297]{LVV-T297} 
        & Draft & Keith Bechtol
        \\
            &&&
        & 
        \hyperref[lvv-t298]{LVV-T298} 
        & Draft & Keith Bechtol
        \\
            &&&
        & 
        \hyperref[lvv-t299]{LVV-T299} 
        & Draft & Keith Bechtol
        \\
            \hdashline
2 &
    \href{https://jira.lsstcorp.org/browse/LVV-1545}{LVV-1545 - OSS-REQ-0228-V-03: Image Quality vs Field
}
& Not Covered & Bo Xin
        & 
        \hyperref[lvv-t294]{LVV-T294} 
        & Draft & Keith Bechtol
        \\
            &&&
        & 
        \hyperref[lvv-t296]{LVV-T296} 
        & Draft & Keith Bechtol
        \\
            \hdashline
3 &
    \href{https://jira.lsstcorp.org/browse/LVV-7213}{LVV-7213 - OSS-REQ-0228-V-04: Image Quality in Encircled Energy
}
& Not Covered & Bo Xin
        & 
        \hyperref[lvv-t294]{LVV-T294} 
        & Draft & Keith Bechtol
        \\
            &&&
        & 
        \hyperref[lvv-t296]{LVV-T296} 
        & Draft & Keith Bechtol
        \\
            \hdashline
4 &
    \href{https://jira.lsstcorp.org/browse/LVV-7214}{LVV-7214 - OSS-REQ-0228-V-05: System Contribution to Image Quality
}
& Not Covered & Bo Xin
        & 
        \hyperref[lvv-t294]{LVV-T294} 
        & Draft & Keith Bechtol
        \\
            &&&
        & 
        \hyperref[lvv-t296]{LVV-T296} 
        & Draft & Keith Bechtol
        \\
            \hdashline
5 &
    \href{https://jira.lsstcorp.org/browse/LVV-238}{LVV-238 - LSR-REQ-0094-V-01: Astrometric Performance1
}
& Not Covered & Chuck Claver
        & 
        \hyperref[lvv-t297]{LVV-T297} 
        & Draft & Keith Bechtol
        \\
            \hdashline
6 &
    \href{https://jira.lsstcorp.org/browse/LVV-1543}{LVV-1543 - OSS-REQ-0228-V-01: Image Quality Off-Zenith Degredation
}
& Not Covered & Bo Xin
        & 
        \hyperref[lvv-t360]{LVV-T360} 
        & Draft & Keith Bechtol
        \\
            \hdashline
7 &
    \href{https://jira.lsstcorp.org/browse/LVV-1546}{LVV-1546 - OSS-REQ-0234-V-01: 10-year Ellipticity Residuals
}
& Not Covered & Tony Tyson
        & 
        \hyperref[lvv-t361]{LVV-T361} 
        & Draft & Keith Bechtol
        \\
            \hdashline
8 &
    \href{https://jira.lsstcorp.org/browse/LVV-1366}{LVV-1366 - OSS-REQ-0390-V-01: Ellipticity Correlations
}
& Not Covered & Chuck Claver
        & 
        \hyperref[lvv-t361]{LVV-T361} 
        & Draft & Keith Bechtol
        \\
            \hdashline
9 &
    \href{https://jira.lsstcorp.org/browse/LVV-273}{LVV-273 - LSR-REQ-0093-V-01: Photometric Performance1
}
& Not Covered & Imram Hasan
        & 
        \hyperref[lvv-t389]{LVV-T389} 
        & Draft & Imram Hasan
        \\
            \hdashline
10 &
    \href{https://jira.lsstcorp.org/browse/LVV-1372}{LVV-1372 - OSS-REQ-0387-V-01: Photometric Performance\_1
}
& Not Covered & Imram Hasan
        & 
        \hyperref[lvv-t389]{LVV-T389} 
        & Draft & Imram Hasan
        \\
            \hdashline
11 &
    \href{https://jira.lsstcorp.org/browse/LVV-1555}{LVV-1555 - OSS-REQ-0238-V-01: Filter Response Uniformity u-band blue edge
}
& Not Covered & Brian Stalder
        & 
        \hyperref[lvv-t434]{LVV-T434} 
        & Draft & Brian Stalder
        \\
            &&&
        & 
        \hyperref[lvv-t445]{LVV-T445} 
        & Draft & Brian Stalder
        \\
            &&&
        & 
        \hyperref[lvv-t446]{LVV-T446} 
        & Draft & Brian Stalder
        \\
            &&&
        & 
        \hyperref[lvv-t447]{LVV-T447} 
        & Draft & Brian Stalder
        \\
            &&&
        & 
        \hyperref[lvv-t448]{LVV-T448} 
        & Draft & Brian Stalder
        \\
            \hdashline
12 &
    \href{https://jira.lsstcorp.org/browse/LVV-1570}{LVV-1570 - OSS-REQ-0239-V-01: In-band Ripple u-band
}
& Not Covered & Brian Stalder
        & 
        \hyperref[lvv-t434]{LVV-T434} 
        & Draft & Brian Stalder
        \\
            &&&
        & 
        \hyperref[lvv-t445]{LVV-T445} 
        & Draft & Brian Stalder
        \\
            &&&
        & 
        \hyperref[lvv-t446]{LVV-T446} 
        & Draft & Brian Stalder
        \\
            &&&
        & 
        \hyperref[lvv-t447]{LVV-T447} 
        & Draft & Brian Stalder
        \\
            &&&
        & 
        \hyperref[lvv-t452]{LVV-T452} 
        & Draft & Brian Stalder
        \\
            \hdashline
13 &
    \href{https://jira.lsstcorp.org/browse/LVV-1582}{LVV-1582 - OSS-REQ-0240-V-01: u-band Response Envelope
}
& Not Covered & Brian Stalder
        & 
        \hyperref[lvv-t434]{LVV-T434} 
        & Draft & Brian Stalder
        \\
            &&&
        & 
        \hyperref[lvv-t445]{LVV-T445} 
        & Draft & Brian Stalder
        \\
            &&&
        & 
        \hyperref[lvv-t446]{LVV-T446} 
        & Draft & Brian Stalder
        \\
            &&&
        & 
        \hyperref[lvv-t447]{LVV-T447} 
        & Draft & Brian Stalder
        \\
            &&&
        & 
        \hyperref[lvv-t453]{LVV-T453} 
        & Draft & Brian Stalder
        \\
            \hdashline
14 &
    \href{https://jira.lsstcorp.org/browse/LVV-1579}{LVV-1579 - OSS-REQ-0366-V-01: u-band not-to-exceed envelope
}
& Not Covered & Brian Stalder
        & 
        \hyperref[lvv-t434]{LVV-T434} 
        & Draft & Brian Stalder
        \\
            &&&
        & 
        \hyperref[lvv-t445]{LVV-T445} 
        & Draft & Brian Stalder
        \\
            &&&
        & 
        \hyperref[lvv-t446]{LVV-T446} 
        & Draft & Brian Stalder
        \\
            &&&
        & 
        \hyperref[lvv-t447]{LVV-T447} 
        & Draft & Brian Stalder
        \\
            &&&
        & 
        \hyperref[lvv-t453]{LVV-T453} 
        & Draft & Brian Stalder
        \\
            \hdashline
15 &
    \href{https://jira.lsstcorp.org/browse/LVV-1561}{LVV-1561 - OSS-REQ-0241-V-01: g-band Response Envelope
}
& Not Covered & Brian Stalder
        & 
        \hyperref[lvv-t434]{LVV-T434} 
        & Draft & Brian Stalder
        \\
            &&&
        & 
        \hyperref[lvv-t445]{LVV-T445} 
        & Draft & Brian Stalder
        \\
            &&&
        & 
        \hyperref[lvv-t446]{LVV-T446} 
        & Draft & Brian Stalder
        \\
            &&&
        & 
        \hyperref[lvv-t447]{LVV-T447} 
        & Draft & Brian Stalder
        \\
            &&&
        & 
        \hyperref[lvv-t453]{LVV-T453} 
        & Draft & Brian Stalder
        \\
            \hdashline
16 &
    \href{https://jira.lsstcorp.org/browse/LVV-1558}{LVV-1558 - OSS-REQ-0367-V-01: g-band not-to-exceed envelope
}
& Not Covered & Brian Stalder
        & 
        \hyperref[lvv-t434]{LVV-T434} 
        & Draft & Brian Stalder
        \\
            &&&
        & 
        \hyperref[lvv-t445]{LVV-T445} 
        & Draft & Brian Stalder
        \\
            &&&
        & 
        \hyperref[lvv-t446]{LVV-T446} 
        & Draft & Brian Stalder
        \\
            &&&
        & 
        \hyperref[lvv-t447]{LVV-T447} 
        & Draft & Brian Stalder
        \\
            &&&
        & 
        \hyperref[lvv-t453]{LVV-T453} 
        & Draft & Brian Stalder
        \\
            \hdashline
17 &
    \href{https://jira.lsstcorp.org/browse/LVV-1576}{LVV-1576 - OSS-REQ-0242-V-01: r-band Response Envelope
}
& Not Covered & Brian Stalder
        & 
        \hyperref[lvv-t434]{LVV-T434} 
        & Draft & Brian Stalder
        \\
            &&&
        & 
        \hyperref[lvv-t445]{LVV-T445} 
        & Draft & Brian Stalder
        \\
            &&&
        & 
        \hyperref[lvv-t446]{LVV-T446} 
        & Draft & Brian Stalder
        \\
            &&&
        & 
        \hyperref[lvv-t447]{LVV-T447} 
        & Draft & Brian Stalder
        \\
            &&&
        & 
        \hyperref[lvv-t453]{LVV-T453} 
        & Draft & Brian Stalder
        \\
            \hdashline
18 &
    \href{https://jira.lsstcorp.org/browse/LVV-1573}{LVV-1573 - OSS-REQ-0368-V-01: r-band not-to-exceed envelope
}
& Not Covered & Brian Stalder
        & 
        \hyperref[lvv-t434]{LVV-T434} 
        & Draft & Brian Stalder
        \\
            &&&
        & 
        \hyperref[lvv-t445]{LVV-T445} 
        & Draft & Brian Stalder
        \\
            &&&
        & 
        \hyperref[lvv-t446]{LVV-T446} 
        & Draft & Brian Stalder
        \\
            &&&
        & 
        \hyperref[lvv-t447]{LVV-T447} 
        & Draft & Brian Stalder
        \\
            &&&
        & 
        \hyperref[lvv-t453]{LVV-T453} 
        & Draft & Brian Stalder
        \\
            \hdashline
19 &
    \href{https://jira.lsstcorp.org/browse/LVV-1567}{LVV-1567 - OSS-REQ-0243-V-01: i-band Response Envelope
}
& Not Covered & Brian Stalder
        & 
        \hyperref[lvv-t434]{LVV-T434} 
        & Draft & Brian Stalder
        \\
            &&&
        & 
        \hyperref[lvv-t445]{LVV-T445} 
        & Draft & Brian Stalder
        \\
            &&&
        & 
        \hyperref[lvv-t446]{LVV-T446} 
        & Draft & Brian Stalder
        \\
            &&&
        & 
        \hyperref[lvv-t447]{LVV-T447} 
        & Draft & Brian Stalder
        \\
            &&&
        & 
        \hyperref[lvv-t453]{LVV-T453} 
        & Draft & Brian Stalder
        \\
            \hdashline
20 &
    \href{https://jira.lsstcorp.org/browse/LVV-1564}{LVV-1564 - OSS-REQ-0369-V-01: i-band not-to-exceed envelope
}
& Not Covered & Brian Stalder
        & 
        \hyperref[lvv-t434]{LVV-T434} 
        & Draft & Brian Stalder
        \\
            &&&
        & 
        \hyperref[lvv-t445]{LVV-T445} 
        & Draft & Brian Stalder
        \\
            &&&
        & 
        \hyperref[lvv-t446]{LVV-T446} 
        & Draft & Brian Stalder
        \\
            &&&
        & 
        \hyperref[lvv-t447]{LVV-T447} 
        & Draft & Brian Stalder
        \\
            &&&
        & 
        \hyperref[lvv-t453]{LVV-T453} 
        & Draft & Brian Stalder
        \\
            \hdashline
21 &
    \href{https://jira.lsstcorp.org/browse/LVV-1594}{LVV-1594 - OSS-REQ-0244-V-01: z-band Response Envelope
}
& Not Covered & Brian Stalder
        & 
        \hyperref[lvv-t434]{LVV-T434} 
        & Draft & Brian Stalder
        \\
            &&&
        & 
        \hyperref[lvv-t445]{LVV-T445} 
        & Draft & Brian Stalder
        \\
            &&&
        & 
        \hyperref[lvv-t446]{LVV-T446} 
        & Draft & Brian Stalder
        \\
            &&&
        & 
        \hyperref[lvv-t447]{LVV-T447} 
        & Draft & Brian Stalder
        \\
            &&&
        & 
        \hyperref[lvv-t453]{LVV-T453} 
        & Draft & Brian Stalder
        \\
            \hdashline
22 &
    \href{https://jira.lsstcorp.org/browse/LVV-1591}{LVV-1591 - OSS-REQ-0370-V-01: z-band not-to-exceed envelope
}
& Not Covered & Brian Stalder
        & 
        \hyperref[lvv-t434]{LVV-T434} 
        & Draft & Brian Stalder
        \\
            &&&
        & 
        \hyperref[lvv-t445]{LVV-T445} 
        & Draft & Brian Stalder
        \\
            &&&
        & 
        \hyperref[lvv-t446]{LVV-T446} 
        & Draft & Brian Stalder
        \\
            &&&
        & 
        \hyperref[lvv-t447]{LVV-T447} 
        & Draft & Brian Stalder
        \\
            &&&
        & 
        \hyperref[lvv-t453]{LVV-T453} 
        & Draft & Brian Stalder
        \\
            \hdashline
23 &
    \href{https://jira.lsstcorp.org/browse/LVV-1588}{LVV-1588 - OSS-REQ-0245-V-01: y-band Response Envelope
}
& Not Covered & Brian Stalder
        & 
        \hyperref[lvv-t434]{LVV-T434} 
        & Draft & Brian Stalder
        \\
            &&&
        & 
        \hyperref[lvv-t445]{LVV-T445} 
        & Draft & Brian Stalder
        \\
            &&&
        & 
        \hyperref[lvv-t446]{LVV-T446} 
        & Draft & Brian Stalder
        \\
            &&&
        & 
        \hyperref[lvv-t447]{LVV-T447} 
        & Draft & Brian Stalder
        \\
            &&&
        & 
        \hyperref[lvv-t453]{LVV-T453} 
        & Draft & Brian Stalder
        \\
            \hdashline
24 &
    \href{https://jira.lsstcorp.org/browse/LVV-1585}{LVV-1585 - OSS-REQ-0371-V-01: y-band not-to-exceed envelope
}
& Not Covered & Brian Stalder
        & 
        \hyperref[lvv-t434]{LVV-T434} 
        & Draft & Brian Stalder
        \\
            &&&
        & 
        \hyperref[lvv-t445]{LVV-T445} 
        & Draft & Brian Stalder
        \\
            &&&
        & 
        \hyperref[lvv-t446]{LVV-T446} 
        & Draft & Brian Stalder
        \\
            &&&
        & 
        \hyperref[lvv-t447]{LVV-T447} 
        & Draft & Brian Stalder
        \\
            &&&
        & 
        \hyperref[lvv-t453]{LVV-T453} 
        & Draft & Brian Stalder
        \\
            \hdashline
25 &
    \href{https://jira.lsstcorp.org/browse/LVV-1556}{LVV-1556 - OSS-REQ-0238-V-02: Filter Response Uniformity u-band red edge
}
& Not Covered & Brian Stalder
        & 
        \hyperref[lvv-t434]{LVV-T434} 
        & Draft & Brian Stalder
        \\
            &&&
        & 
        \hyperref[lvv-t445]{LVV-T445} 
        & Draft & Brian Stalder
        \\
            &&&
        & 
        \hyperref[lvv-t446]{LVV-T446} 
        & Draft & Brian Stalder
        \\
            &&&
        & 
        \hyperref[lvv-t447]{LVV-T447} 
        & Draft & Brian Stalder
        \\
            &&&
        & 
        \hyperref[lvv-t448]{LVV-T448} 
        & Draft & Brian Stalder
        \\
            \hdashline
26 &
    \href{https://jira.lsstcorp.org/browse/LVV-1557}{LVV-1557 - OSS-REQ-0238-V-03: Filter Response Uniformity g-band blue edge
}
& Not Covered & Brian Stalder
        & 
        \hyperref[lvv-t434]{LVV-T434} 
        & Draft & Brian Stalder
        \\
            &&&
        & 
        \hyperref[lvv-t445]{LVV-T445} 
        & Draft & Brian Stalder
        \\
            &&&
        & 
        \hyperref[lvv-t446]{LVV-T446} 
        & Draft & Brian Stalder
        \\
            &&&
        & 
        \hyperref[lvv-t447]{LVV-T447} 
        & Draft & Brian Stalder
        \\
            &&&
        & 
        \hyperref[lvv-t448]{LVV-T448} 
        & Draft & Brian Stalder
        \\
            \hdashline
27 &
    \href{https://jira.lsstcorp.org/browse/LVV-8678}{LVV-8678 - OSS-REQ-0238-V-04: Filter Response Uniformity g-band red edge
}
& Not Covered & Brian Stalder
        & 
        \hyperref[lvv-t434]{LVV-T434} 
        & Draft & Brian Stalder
        \\
            &&&
        & 
        \hyperref[lvv-t445]{LVV-T445} 
        & Draft & Brian Stalder
        \\
            &&&
        & 
        \hyperref[lvv-t446]{LVV-T446} 
        & Draft & Brian Stalder
        \\
            &&&
        & 
        \hyperref[lvv-t447]{LVV-T447} 
        & Draft & Brian Stalder
        \\
            &&&
        & 
        \hyperref[lvv-t448]{LVV-T448} 
        & Draft & Brian Stalder
        \\
            \hdashline
28 &
    \href{https://jira.lsstcorp.org/browse/LVV-8681}{LVV-8681 - OSS-REQ-0238-V-05: Filter Response Uniformity r-band blue edge
}
& Not Covered & Brian Stalder
        & 
        \hyperref[lvv-t434]{LVV-T434} 
        & Draft & Brian Stalder
        \\
            &&&
        & 
        \hyperref[lvv-t445]{LVV-T445} 
        & Draft & Brian Stalder
        \\
            &&&
        & 
        \hyperref[lvv-t446]{LVV-T446} 
        & Draft & Brian Stalder
        \\
            &&&
        & 
        \hyperref[lvv-t447]{LVV-T447} 
        & Draft & Brian Stalder
        \\
            &&&
        & 
        \hyperref[lvv-t448]{LVV-T448} 
        & Draft & Brian Stalder
        \\
            \hdashline
29 &
    \href{https://jira.lsstcorp.org/browse/LVV-8683}{LVV-8683 - OSS-REQ-0238-V-06: Filter Response Uniformity r-band red edge
}
& Not Covered & Brian Stalder
        & 
        \hyperref[lvv-t434]{LVV-T434} 
        & Draft & Brian Stalder
        \\
            &&&
        & 
        \hyperref[lvv-t445]{LVV-T445} 
        & Draft & Brian Stalder
        \\
            &&&
        & 
        \hyperref[lvv-t446]{LVV-T446} 
        & Draft & Brian Stalder
        \\
            &&&
        & 
        \hyperref[lvv-t447]{LVV-T447} 
        & Draft & Brian Stalder
        \\
            &&&
        & 
        \hyperref[lvv-t448]{LVV-T448} 
        & Draft & Brian Stalder
        \\
            \hdashline
30 &
    \href{https://jira.lsstcorp.org/browse/LVV-8686}{LVV-8686 - OSS-REQ-0238-V-07: Filter Response Uniformity i-band blue edge
}
& Not Covered & Brian Stalder
        & 
        \hyperref[lvv-t434]{LVV-T434} 
        & Draft & Brian Stalder
        \\
            &&&
        & 
        \hyperref[lvv-t445]{LVV-T445} 
        & Draft & Brian Stalder
        \\
            &&&
        & 
        \hyperref[lvv-t446]{LVV-T446} 
        & Draft & Brian Stalder
        \\
            &&&
        & 
        \hyperref[lvv-t447]{LVV-T447} 
        & Draft & Brian Stalder
        \\
            &&&
        & 
        \hyperref[lvv-t448]{LVV-T448} 
        & Draft & Brian Stalder
        \\
            \hdashline
31 &
    \href{https://jira.lsstcorp.org/browse/LVV-8689}{LVV-8689 - OSS-REQ-0238-V-08: Filter Response Uniformity i-band red edge
}
& Not Covered & Brian Stalder
        & 
        \hyperref[lvv-t434]{LVV-T434} 
        & Draft & Brian Stalder
        \\
            &&&
        & 
        \hyperref[lvv-t445]{LVV-T445} 
        & Draft & Brian Stalder
        \\
            &&&
        & 
        \hyperref[lvv-t446]{LVV-T446} 
        & Draft & Brian Stalder
        \\
            &&&
        & 
        \hyperref[lvv-t447]{LVV-T447} 
        & Draft & Brian Stalder
        \\
            &&&
        & 
        \hyperref[lvv-t448]{LVV-T448} 
        & Draft & Brian Stalder
        \\
            \hdashline
32 &
    \href{https://jira.lsstcorp.org/browse/LVV-8691}{LVV-8691 - OSS-REQ-0238-V-09: Filter Response Uniformity z-band blue edge
}
& Not Covered & Brian Stalder
        & 
        \hyperref[lvv-t434]{LVV-T434} 
        & Draft & Brian Stalder
        \\
            &&&
        & 
        \hyperref[lvv-t445]{LVV-T445} 
        & Draft & Brian Stalder
        \\
            &&&
        & 
        \hyperref[lvv-t446]{LVV-T446} 
        & Draft & Brian Stalder
        \\
            &&&
        & 
        \hyperref[lvv-t447]{LVV-T447} 
        & Draft & Brian Stalder
        \\
            &&&
        & 
        \hyperref[lvv-t448]{LVV-T448} 
        & Draft & Brian Stalder
        \\
            \hdashline
33 &
    \href{https://jira.lsstcorp.org/browse/LVV-8694}{LVV-8694 - OSS-REQ-0238-V-10: Filter Response Uniformity z-band red edge
}
& Not Covered & Brian Stalder
        & 
        \hyperref[lvv-t434]{LVV-T434} 
        & Draft & Brian Stalder
        \\
            &&&
        & 
        \hyperref[lvv-t445]{LVV-T445} 
        & Draft & Brian Stalder
        \\
            &&&
        & 
        \hyperref[lvv-t446]{LVV-T446} 
        & Draft & Brian Stalder
        \\
            &&&
        & 
        \hyperref[lvv-t447]{LVV-T447} 
        & Draft & Brian Stalder
        \\
            &&&
        & 
        \hyperref[lvv-t448]{LVV-T448} 
        & Draft & Brian Stalder
        \\
            \hdashline
34 &
    \href{https://jira.lsstcorp.org/browse/LVV-8696}{LVV-8696 - OSS-REQ-0238-V-11: Filter Response Uniformity y-band blue edge
}
& Not Covered & Brian Stalder
        & 
        \hyperref[lvv-t434]{LVV-T434} 
        & Draft & Brian Stalder
        \\
            &&&
        & 
        \hyperref[lvv-t445]{LVV-T445} 
        & Draft & Brian Stalder
        \\
            &&&
        & 
        \hyperref[lvv-t446]{LVV-T446} 
        & Draft & Brian Stalder
        \\
            &&&
        & 
        \hyperref[lvv-t447]{LVV-T447} 
        & Draft & Brian Stalder
        \\
            &&&
        & 
        \hyperref[lvv-t448]{LVV-T448} 
        & Draft & Brian Stalder
        \\
            \hdashline
35 &
    \href{https://jira.lsstcorp.org/browse/LVV-8698}{LVV-8698 - OSS-REQ-0238-V-12: Filter Response Uniformity y-band red edge
}
& Not Covered & Brian Stalder
        & 
        \hyperref[lvv-t434]{LVV-T434} 
        & Draft & Brian Stalder
        \\
            &&&
        & 
        \hyperref[lvv-t445]{LVV-T445} 
        & Draft & Brian Stalder
        \\
            &&&
        & 
        \hyperref[lvv-t446]{LVV-T446} 
        & Draft & Brian Stalder
        \\
            &&&
        & 
        \hyperref[lvv-t447]{LVV-T447} 
        & Draft & Brian Stalder
        \\
            &&&
        & 
        \hyperref[lvv-t448]{LVV-T448} 
        & Draft & Brian Stalder
        \\
            \hdashline
36 &
    \href{https://jira.lsstcorp.org/browse/LVV-1572}{LVV-1572 - OSS-REQ-0239-V-03: In-band Ripple r-band
}
& Not Covered & Brian Stalder
        & 
        \hyperref[lvv-t434]{LVV-T434} 
        & Draft & Brian Stalder
        \\
            &&&
        & 
        \hyperref[lvv-t445]{LVV-T445} 
        & Draft & Brian Stalder
        \\
            &&&
        & 
        \hyperref[lvv-t446]{LVV-T446} 
        & Draft & Brian Stalder
        \\
            &&&
        & 
        \hyperref[lvv-t447]{LVV-T447} 
        & Draft & Brian Stalder
        \\
            &&&
        & 
        \hyperref[lvv-t452]{LVV-T452} 
        & Draft & Brian Stalder
        \\
            \hdashline
37 &
    \href{https://jira.lsstcorp.org/browse/LVV-1571}{LVV-1571 - OSS-REQ-0239-V-02: In-band Ripple g-band
}
& Not Covered & Brian Stalder
        & 
        \hyperref[lvv-t434]{LVV-T434} 
        & Draft & Brian Stalder
        \\
            &&&
        & 
        \hyperref[lvv-t445]{LVV-T445} 
        & Draft & Brian Stalder
        \\
            &&&
        & 
        \hyperref[lvv-t446]{LVV-T446} 
        & Draft & Brian Stalder
        \\
            &&&
        & 
        \hyperref[lvv-t447]{LVV-T447} 
        & Draft & Brian Stalder
        \\
            &&&
        & 
        \hyperref[lvv-t452]{LVV-T452} 
        & Draft & Brian Stalder
        \\
            \hdashline
38 &
    \href{https://jira.lsstcorp.org/browse/LVV-1261}{LVV-1261 - OSS-REQ-0354-V-01: Difference Source Spuriousness Threshold - MOPS
}
& Not Covered & Scott Daniel
        & 
        \hyperref[lvv-t532]{LVV-T532} 
        & Draft & Scott Daniel
        \\
            \hdashline
39 &
    \href{https://jira.lsstcorp.org/browse/LVV-1262}{LVV-1262 - OSS-REQ-0354-V-02: Difference Source Spuriousness Threshold - MOPS2
}
& Not Covered & Scott Daniel
        & 
        \hyperref[lvv-t533]{LVV-T533} 
        & Draft & Scott Daniel
        \\
            \hdashline
40 &
    \href{https://jira.lsstcorp.org/browse/LVV-1273}{LVV-1273 - OSS-REQ-0149-V-01: Level 1 Catalog Precision
}
& Not Covered & Scott Daniel
        & 
        \hyperref[lvv-t543]{LVV-T543} 
        & Defined & Scott Daniel
        \\
            &&&
        & 
        \hyperref[lvv-t544]{LVV-T544} 
        & Defined & Scott Daniel
        \\
            &&&
        & 
        \hyperref[lvv-t545]{LVV-T545} 
        & Defined & Scott Daniel
        \\
            \hdashline
41 &
    \href{https://jira.lsstcorp.org/browse/LVV-1274}{LVV-1274 - OSS-REQ-0149-V-02: Level 1 Catalog Precision2
}
& Not Covered & Scott Daniel
        & 
        \hyperref[lvv-t546]{LVV-T546} 
        & Defined & Scott Daniel
        \\
            &&&
        & 
        \hyperref[lvv-t547]{LVV-T547} 
        & Defined & Scott Daniel
        \\
            &&&
        & 
        \hyperref[lvv-t548]{LVV-T548} 
        & Defined & Scott Daniel
        \\
            \hdashline
42 &
    \href{https://jira.lsstcorp.org/browse/LVV-1291}{LVV-1291 - OSS-REQ-0152-V-01: Level 1 Photometric Zero Point Error
}
& Not Covered & Scott Daniel
        & 
        \hyperref[lvv-t549]{LVV-T549} 
        & Defined & Scott Daniel
        \\
            \hdashline
43 &
    \href{https://jira.lsstcorp.org/browse/LVV-3766}{LVV-3766 - OSS-REQ-0159-V-01: Level 1 Moving Object Quality
}
& Not Covered & Scott Daniel
        & 
        \hyperref[lvv-t550]{LVV-T550} 
        & Defined & Scott Daniel
        \\
            \hdashline
44 &
    \href{https://jira.lsstcorp.org/browse/LVV-1339}{LVV-1339 - OSS-REQ-0153-V-01: World Coordinate System Accuracy
}
& Not Covered & Scott Daniel
        & 
        \hyperref[lvv-t551]{LVV-T551} 
        & Defined & Scott Daniel
        \\
            \hdashline
45 &
    \href{https://jira.lsstcorp.org/browse/LVV-1645}{LVV-1645 - OSS-REQ-0268-V-01: Dynamic Range
}
& Not Covered & Scott Daniel
        & 
        \hyperref[lvv-t554]{LVV-T554} 
        & Defined & Scott Daniel
        \\
            \hdashline
46 &
    \href{https://jira.lsstcorp.org/browse/LVV-253}{LVV-253 - LSR-REQ-0007-V-01: Delivered Image Quality1
}
& Not Covered & Scott Daniel
        & 
        \hyperref[lvv-t587]{LVV-T587} 
        & Defined & Scott Daniel
        \\
            \hdashline
47 &
    \href{https://jira.lsstcorp.org/browse/LVV-254}{LVV-254 - LSR-REQ-0007-V-02: Delivered Image Quality2
}
& Not Covered & Scott Daniel
        & 
        \hyperref[lvv-t588]{LVV-T588} 
        & Defined & Scott Daniel
        \\
            \hdashline
48 &
    \href{https://jira.lsstcorp.org/browse/LVV-255}{LVV-255 - LSR-REQ-0007-V-03: Delivered Image Quality3
}
& Not Covered & Scott Daniel
        & 
        \hyperref[lvv-t589]{LVV-T589} 
        & Defined & Scott Daniel
        \\
            \hdashline
49 &
    \href{https://jira.lsstcorp.org/browse/LVV-256}{LVV-256 - LSR-REQ-0007-V-04: Delivered Image Quality4
}
& Not Covered & Scott Daniel
        & 
        \hyperref[lvv-t590]{LVV-T590} 
        & Defined & Scott Daniel
        \\
            \hdashline
50 &
    \href{https://jira.lsstcorp.org/browse/LVV-257}{LVV-257 - LSR-REQ-0007-V-05: Delivered Image Quality5
}
& Not Covered & Scott Daniel
        & 
        \hyperref[lvv-t591]{LVV-T591} 
        & Defined & Scott Daniel
        \\
            \hdashline
51 &
    \href{https://jira.lsstcorp.org/browse/LVV-9804}{LVV-9804 - LSR-REQ-0007-V-06: Delivered Image Quality\_6
}
& Not Covered & Scott Daniel
        & 
        \hyperref[lvv-t592]{LVV-T592} 
        & Defined & Scott Daniel
        \\
            \hdashline
52 &
    \href{https://jira.lsstcorp.org/browse/LVV-9805}{LVV-9805 - LSR-REQ-0007-V-07: Delivered Image Quality\_7
}
& Not Covered & Scott Daniel
        & 
        \hyperref[lvv-t593]{LVV-T593} 
        & Defined & Scott Daniel
        \\
            \hdashline
53 &
    \href{https://jira.lsstcorp.org/browse/LVV-268}{LVV-268 - LSR-REQ-0087-V-01: Off Zenith Degradation1
}
& Not Covered & Scott Daniel
        & 
        \hyperref[lvv-t594]{LVV-T594} 
        & Defined & Scott Daniel
        \\
            \hdashline
54 &
    \href{https://jira.lsstcorp.org/browse/LVV-248}{LVV-248 - LSR-REQ-0092-V-01: Delivered Image Ellipticity1
}
& Not Covered & Scott Daniel
        & 
        \hyperref[lvv-t595]{LVV-T595} 
        & Defined & Scott Daniel
        \\
            \hdashline
55 &
    \href{https://jira.lsstcorp.org/browse/LVV-3751}{LVV-3751 - LSR-REQ-0089-V-01: r-band Reference Depth1
}
& Not Covered & Scott Daniel
        & 
        \hyperref[lvv-t596]{LVV-T596} 
        & Defined & Scott Daniel
        \\
            \hdashline
56 &
    \href{https://jira.lsstcorp.org/browse/LVV-258}{LVV-258 - LSR-REQ-0109-V-01: Depth Variation Over FOV1
}
& Not Covered & Scott Daniel
        & 
        \hyperref[lvv-t597]{LVV-T597} 
        & Defined & Scott Daniel
        \\
            \hdashline
57 &
    \href{https://jira.lsstcorp.org/browse/LVV-1237}{LVV-1237 - OSS-REQ-0123-V-01: Reproducibility
}
& Not Covered & Scott Daniel
        & 
        \hyperref[lvv-t938]{LVV-T938} 
        & Defined & Scott Daniel
        \\
            &&&
        & 
        \hyperref[lvv-t939]{LVV-T939} 
        & Defined & Scott Daniel
        \\
            &&&
        & 
        \hyperref[lvv-t940]{LVV-T940} 
        & Defined & Scott Daniel
        \\
            &&&
        & 
        \hyperref[lvv-t941]{LVV-T941} 
        & Defined & Scott Daniel
        \\
            \hdashline
58 &
    \href{https://jira.lsstcorp.org/browse/LVV-165}{LVV-165 - DMS-REQ-0334-V-01: Persisting Data Products
}
& Not Covered & Colin Slater
        & 
        \hyperref[lvv-t12]{LVV-T12} 
        & Approved & Jim Bosch
        \\
            \hdashline
59 &
    \href{https://jira.lsstcorp.org/browse/LVV-98}{LVV-98 - DMS-REQ-0267-V-01: Source Catalog
}
& Not Covered & Jim Bosch
        & 
        \hyperref[lvv-t12]{LVV-T12} 
        & Approved & Jim Bosch
        \\
            \hdashline
60 &
    \href{https://jira.lsstcorp.org/browse/LVV-99}{LVV-99 - DMS-REQ-0268-V-01: Forced-Source Catalog
}
& Not Covered & Jim Bosch
        & 
        \hyperref[lvv-t12]{LVV-T12} 
        & Approved & Jim Bosch
        \\
            \hdashline
61 &
    \href{https://jira.lsstcorp.org/browse/LVV-106}{LVV-106 - DMS-REQ-0275-V-01: Object Catalog
}
& Not Covered & Jim Bosch
        & 
        \hyperref[lvv-t12]{LVV-T12} 
        & Approved & Jim Bosch
        \\
            \hdashline
62 &
    \href{https://jira.lsstcorp.org/browse/LVV-110}{LVV-110 - DMS-REQ-0279-V-01: Deep Detection Coadds
}
& Not Covered & Jim Bosch
        & 
        \hyperref[lvv-t12]{LVV-T12} 
        & Approved & Jim Bosch
        \\
            \hdashline
63 &
    \href{https://jira.lsstcorp.org/browse/LVV-125}{LVV-125 - DMS-REQ-0294-V-01: Processing of Datasets
}
& Not Covered & Robert Lupton
        & 
        \hyperref[lvv-t12]{LVV-T12} 
        & Approved & Jim Bosch
        \\
            \hdashline
64 &
    \href{https://jira.lsstcorp.org/browse/LVV-46}{LVV-46 - DMS-REQ-0106-V-01: Coadded Image Provenance
}
& Not Covered & Robert Gruendl
        & 
        \hyperref[lvv-t64]{LVV-T64} 
        & Draft & Jim Bosch
        \\
            \hdashline
65 &
    \href{https://jira.lsstcorp.org/browse/LVV-1234}{LVV-1234 - OSS-REQ-0122-V-01: Provenance
}
& Not Covered & Scott Daniel
        & 
        \hyperref[lvv-t64]{LVV-T64} 
        & Draft & Jim Bosch
        \\
            &&&
        & 
        \hyperref[lvv-t942]{LVV-T942} 
        & Defined & Scott Daniel
        \\
            &&&
        & 
        \hyperref[lvv-t33]{LVV-T33} 
        & Draft & Kian-Tat Lim
        \\
            &&&
        & 
        \hyperref[lvv-t943]{LVV-T943} 
        & Defined & Scott Daniel
        \\
            \hdashline
66 &
    \href{https://jira.lsstcorp.org/browse/LVV-8}{LVV-8 - DMS-REQ-0018-V-01: Raw Science Image Data Acquisition
}
& Not Covered & Robert Gruendl
        & 
        \hyperref[lvv-t29]{LVV-T29} 
        & Draft & Kian-Tat Lim
        \\
            \hdashline
67 &
    \href{https://jira.lsstcorp.org/browse/LVV-11}{LVV-11 - DMS-REQ-0024-V-01: Raw Image Assembly
}
& Not Covered & Gregory Dubois-Felsmann
        & 
        \hyperref[lvv-t32]{LVV-T32} 
        & Draft & Kian-Tat Lim
        \\
            \hdashline
68 &
    \href{https://jira.lsstcorp.org/browse/LVV-28}{LVV-28 - DMS-REQ-0068-V-01: Raw Science Image Metadata
}
& Not Covered & Gregory Dubois-Felsmann
        & 
        \hyperref[lvv-t33]{LVV-T33} 
        & Draft & Kian-Tat Lim
        \\
            \hdashline
69 &
    \href{https://jira.lsstcorp.org/browse/LVV-1285}{LVV-1285 - OSS-REQ-0160-V-01: Level 1 Difference Source - Difference Object
Association Quality
}
& Not Covered & Scott Daniel
        & 
        \hyperref[lvv-t950]{LVV-T950} 
        & Defined & Scott Daniel
        \\
            \hdashline
\tabularnewline
\bottomrule
\end{longtable}
} % end scriptsize
