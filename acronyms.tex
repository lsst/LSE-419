\addtocounter{table}{-1}
\begin{longtable}{|l|p{0.8\textwidth}|}\hline
\textbf{Acronym} & \textbf{Description}  \\\hline

AP & Alert Production \\\hline
ATM & Adaptavist Test Management \\\hline
Alert & A packet of information for each source detected with signal-to-noise ratio > 5 in a difference image during Prompt Processing, containing measurement and characterization parameters based on the past 12 months of LSST observations plus small cutouts of the single-visit, template, and difference images, distributed via the internet. \\\hline
Alert Production & The principal component of Prompt Processing that processes and calibrates incoming images, performs Difference Image Analysis to identify DIASources and DIAObjects, packages and distributes the resulting Alerts, and runs the Moving Object Processing System. \\\hline
B & Byte (8 bit) \\\hline
Butler & A middleware component for persisting and retrieving image datasets (raw or processed), calibration reference data, and catalogs. \\\hline
CCD & Charge-Coupled Device \\\hline
CCOB & Camera Calibration Optical Bench \\\hline
Collimated Beam Projector & The hardware to project a field of sources onto discrete sections of the telescope optics in order to characterize spatial variations in the telescope and instrument transmission function, and to monitor filter throughput evolution during the survey. Images obtained using the CBP will be used in calibration. \\\hline
Commissioning & A two-year phase at the end of the Construction project during which a technical team a) integrates the various technical components of the three subsystems; b) shows their compliance with ICDs and system-level requirements as detailed in the LSST Observatory System Specifications document (OSS, LSE-30); and c) performs science verification to show compliance with the survey performance specifications as detailed in the LSST Science Requirements Document (SRD, LPM-17). \\\hline
DEC & Declination \\\hline
DIASource & A DIASource is a detection with signal-to-noise ratio greater than 5 in a difference image. \\\hline
DIMM & Differential Image Motion Monitor \\\hline
DM & Data Management \\\hline
DMS & Data Management Subsystem \\\hline
DRP & Data Release Production \\\hline
Data Management & The LSST Subsystem responsible for the Data Management System (DMS), which will capture, store, catalog, and serve the LSST dataset to the scientific community and public. The DM team is responsible for the DMS architecture, applications, middleware, infrastructure, algorithms, and Observatory Network Design. DM is a distributed team working at LSST and partner institutions, with the DM Subsystem Manager located at LSST headquarters in Tucson. \\\hline
Data Release Processing & Deprecated term; see Data Release Production. \\\hline
Document & Any object (in any application supported by DocuShare or design archives such as PDMWorks or GIT) that supports project management or records milestones and deliverables of the LSST Project \\\hline
FWHM & Full Width at Half-Maximum \\\hline
Filter & A filter in astronomy is an optical element used to restrict the passband of light reaching the focal plane, it transmits a selected range of wavelengths. Filters elements are often named after standard photometric passbands, such as those used in the SDSS survey: u, g, r, i, z. \\\hline
HSC & Hyper Suprime-Cam \\\hline
HST & Hubble Space Telescope \\\hline
ISR & Instrument Signal Removal \\\hline
Instrument Signature Removal & Instrument Signature Removal is a pipeline that applies calibration reference data in the course of raw data processing, to remove artifacts of the instrument or detector electronics, such as removal of overscan pixels, bias correction, and the application of a flat-field to correct for pixel-to-pixel variations in sensitivity. \\\hline
LDM & LSST Data Management (Document Handle) \\\hline
LPM & LSST Project Management (Document Handle) \\\hline
LSE & LSST Systems Engineering (Document Handle) \\\hline
LSR & LSST System Requirements; LSE-29 \\\hline
LSST & Large Synoptic Survey Telescope \\\hline
MOPS & Moving Object Processing System \\\hline
OSS & Observatory System Specifications; LSE-30 \\\hline
Object & In LSST nomenclature this refers to an astronomical object, such as a star, galaxy, or other physical entity. E.g., comets, asteroids are also Objects but typically called a Moving Object or a Solar System Object (SSObject). One of the DRP data products is a table of Objects detected by LSST which can be static, or change brightness or position with time. \\\hline
OpSim & Operations Simulation \\\hline
PSF & Point Spread Function \\\hline
RA & Right Ascension \\\hline
RMS & Root-Mean-Square \\\hline
Raft & The sensors in the LSST camera are packaged into replaceable electronic assemblies, called rafts, consisting of 9 butted sensors (CCDs) in a 3x3 mosaic. Each raft is a replaceable unit in the LSST camera. There are 21 science rafts in the camera plus 4 additional corner rafts with specialized, non-science sensors, making for a total of 189 CCDs per focal plane image. The 21 science rafts are numbered from "0,1" through "0,3", "1,0" through "3,4", and "4,1" through "4,3". (In other words, the 25 combinations from "0,0" through "4,4" minus the four corners which are non-science.) \\\hline
SED & Spectral Energy Distribution \\\hline
SRD & LSST Science Requirements; LPM-17 \\\hline
SV & Science Validation \\\hline
Science Verification & The second phase of Commissioning for the LSST Construction Project, Science Verification demonstrates the system's compliance with the survey performance specifications detailed in the LSST Science Requirements Document (SRD, LPM-17). These activities are based solely on the measured 'on-sky' performance of the LSST system. \\\hline
Source & A single detection of an astrophysical object in an image, the characteristics for which are stored in the Source Catalog of the DRP database. The association of Sources that are non-moving lead to Objects; the association of moving Sources leads to Solar System Objects. (Note that in non-LSST usage "source" is often used for what LSST calls an Object.) \\\hline
Source Association & The process of associating source detections on multiple images taken at different epochs, or in multiple passbands, with a single astronomical Object. \\\hline
Specification & One or more performance parameter(s) being established by a requirement that the delivered system or subsystem must meet \\\hline
Summit & The site on the Cerro Pachón, Chile mountaintop where the LSST observatory, support facilities, and infrastructure will be built. \\\hline
Template & A co-added, single-band image of the sky that is deep, and created in a manner to remove transient or fast moving objects from the final image. Constituent images for templates may be selected from a limited range of quality parameters, such as PSF size or airmass. Such images are used to perform Difference Image Analysis in order to detect variable, transient, and Solar System astrophysical objects. \\\hline
Verification & The process of evaluating the design, including hardware and software - to ensure the requirements have been met;  verification (of requirements) is performed by test, analysis, inspection, and/or demonstration \\\hline
Visit & A sequence of one or more consecutive exposures at a given position, orientation, and filter within the LSST cadence. See Standard Visit, Alternative Standard Visit, and Non-Standard Visit,DM TS Sims,,
Education and Public Outreach (EPO),The LSST subsystem responsible for the cyberinfrastructure \\\hline
WCS & World Coordinate System \\\hline
World Coordinate System & a mapping from image pixel coordinates to physical coordinates; in the case of images the mapping is to sky coordinates, generally in an equatorial (RA, Dec) system. The WCS is expressed in FITS file extensions as a collection of header keyword=value pairs (basically, the values of parameters for a selected functional representation of the mapping) that are specified in the FITS Standard. \\\hline
airmass & The pathlength of light from an astrophysical source through the Earth's atmosphere. It is given approximately by sec z, where z is the angular distance from the zenith (the point directly overhead, where airmass = 1.0) to the source. \\\hline
algorithm & A computational implementation of a calculation or some method of processing. \\\hline
arcmin & arcminute minute of arc (unit of angle) \\\hline
arcsec & arcsecond second of arc (unit of angle) \\\hline
astrometry & In astronomy, the sub-discipline of astrometry concerns precision measurement of positions (at a reference epoch), and real and apparent motions of astrophysical objects. Real motion means 3-D motions of the object with respect to an inertial reference frame; apparent motions are an artifact of the motion of the Earth. Astrometry per se is sometimes confused with the act of determining a World Coordinate System (WCS), which is a functional characterization of the mapping from pixels in an image or spectrum to world coordinate such as (RA, Dec) or wavelength. \\\hline
background & In an image, the background consists of contributions from the sky (e.g., clouds or scattered moonlight), and from the telescope and camera optics, which must be distinguished from the astrophysical background. The sky and instrumental backgrounds are characterized and removed by the LSST processing software using a low-order spatial function whose coefficients are recorded in the image metadata. \\\hline
calibration & The process of translating signals produced by a measuring instrument such as a telescope and camera into physical units such as flux, which are used for scientific analysis. Calibration removes most of the contributions to the signal from environmental and instrumental factors, such that only the astronomical component remains. \\\hline
camera & An imaging device mounted at a telescope focal plane, composed of optics, a shutter, a set of filters, and one or more sensors arranged in a focal plane array. \\\hline
deg & degree; unit of angle \\\hline
epoch & Sky coordinate reference frame, e.g., J2000. Alternatively refers to a single observation (usually photometric, can be multi-band) of a variable source. \\\hline
flux & Shorthand for radiative flux, it is a measure of the transport of radiant energy per unit area per unit time. In astronomy this is usually expressed in cgs units: erg/cm2/s. \\\hline
footprint & See 'source footprint', 'instrumental footprint', or 'survey footprint', `Footprint` is a Python class representing a source footprint. \\\hline
forced photometry & A measurement of the photometric properties of a source, or expected source, with one or more parameters held fixed. Most often this means fixing the location of the center of the brightness profile (which may be known or predicted in advance), and measuring other properties such as total brightness, shape, and orientation. Forced photometry will be done for all Objects in the Data Release Production. \\\hline
jointcal & The jointcal package optimizes the astrometric and photometric calibrations of a set of astronomical images that cover a sky tract and were obtained as a series of visits, which may be spread out in time. The jointcal algorithms incorporates object matching both between visits and to reference star catalogs, and produces more accurate distortion and throughput models than if the astrometry and photometry were fit independently. Jointcal is a part of the Science Pipelines. \\\hline
metadata & General term for data about data, e.g., attributes of astronomical objects (e.g. images, sources, astroObjects, etc.) that are characteristics of the objects themselves, and facilitate the organization, preservation, and query of data sets. (E.g., a FITS header contains metadata). \\\hline
metric & A measurable quantity which may be tracked. A metric has a name, description, unit, references, and tags (which are used for grouping). A metric is a scalar by definition. See also: aggregate metric, model metric, point metric. \\\hline
patch & An quadrilateral sub-region of a sky tract, with a size in pixels chosen to fit easily into memory on desktop computers. \\\hline
pipeline & A configured sequence of software tasks (Stages) to process data and generate data products. Example: Association Pipeline. \\\hline
provenance & Information about how LSST images, Sources, and Objects were created (e.g., versions of pipelines, algorithmic components, or templates) and how to recreate them. \\\hline
seeing & An astronomical term for characterizing the stability of the atmosphere, as measured by the width of the point-spread function on images. The PSF width is also affected by a number of other factors, including the airmass, passband, and the telescope and camera optics. \\\hline
shape & In reference to a Source or Object, the shape is a functional characterization of its spatial intensity distribution, and the integral of the shape is the flux. Shape characterizations are a data product in the DIASource, DIAObject, Source, and Object catalogs. \\\hline
sky map & A sky tessellation for LSST. The Stack includes software to define a geometric mapping from the representation of World Coordinates in input images to the LSST sky map. This tessellation is comprised of individual tracts which are, in turn, comprised of patches. \\\hline
stack & A record of all versions of a document uploaded to a particular DocuShare handle \\\hline
tract & A portion of sky, a spherical convex polygon, within the LSST all-sky tessellation (sky map). Each tract is subdivided into sky patches. \\\hline
transient & A transient source is one that has been detected on a difference image, but has not been associated with either an astronomical object or a solar system body. \\\hline
\end{longtable}
